%
% API Documentation for Peach - Computational Intelligence for Python
% Module peach.ga.base
%
% Generated by epydoc 3.0.1
% [Thu Jul 28 16:37:46 2011]
%

%%%%%%%%%%%%%%%%%%%%%%%%%%%%%%%%%%%%%%%%%%%%%%%%%%%%%%%%%%%%%%%%%%%%%%%%%%%
%%                          Module Description                           %%
%%%%%%%%%%%%%%%%%%%%%%%%%%%%%%%%%%%%%%%%%%%%%%%%%%%%%%%%%%%%%%%%%%%%%%%%%%%

    \index{peach \textit{(package)}!peach.ga \textit{(package)}!peach.ga.base \textit{(module)}|(}
\section{Module peach.ga.base}

    \label{peach:ga:base}

Basic Genetic Algorithm (GA)

This sub-package implements a traditional genetic algorithm as described in
books and papers. It consists of selecting, breeding and mutating a population
of chromosomes (arrays of bits) and reinserting the fittest individual from the
previous generation if the GA is elitist. Please, consult a good reference on
the subject, for the subject is way too complicated to be explained here.

Within the algorithm implemented here, it is possible to specify and configure
the selection, crossover and mutation methods using the classes in the
respective sub-modules and custom methods can be implemented (check
\texttt{selection}, \texttt{crossover} and \texttt{mutation} modules).

A GA object is actually a list of chromosomes. Please, refer to the
documentation of the class below for more information.

%%%%%%%%%%%%%%%%%%%%%%%%%%%%%%%%%%%%%%%%%%%%%%%%%%%%%%%%%%%%%%%%%%%%%%%%%%%
%%                               Variables                               %%
%%%%%%%%%%%%%%%%%%%%%%%%%%%%%%%%%%%%%%%%%%%%%%%%%%%%%%%%%%%%%%%%%%%%%%%%%%%

  \subsection{Variables}

    \vspace{-1cm}
\hspace{\varindent}\begin{longtable}{|p{\varnamewidth}|p{\vardescrwidth}|l}
\cline{1-2}
\cline{1-2} \centering \textbf{Name} & \centering \textbf{Description}& \\
\cline{1-2}
\endhead\cline{1-2}\multicolumn{3}{r}{\small\textit{continued on next page}}\\\endfoot\cline{1-2}
\endlastfoot\raggedright \_\-\_\-d\-o\-c\-\_\-\_\- & \raggedright \textbf{Value:} 
{\tt \texttt{...}}&\\
\cline{1-2}
\raggedright \_\-\_\-p\-a\-c\-k\-a\-g\-e\-\_\-\_\- & \raggedright \textbf{Value:} 
{\tt \texttt{'}\texttt{peach.ga}\texttt{'}}&\\
\cline{1-2}
\raggedright a\-d\-d\- & \raggedright \textbf{Value:} 
{\tt {\textless}ufunc 'add'{\textgreater}}&\\
\cline{1-2}
\end{longtable}


%%%%%%%%%%%%%%%%%%%%%%%%%%%%%%%%%%%%%%%%%%%%%%%%%%%%%%%%%%%%%%%%%%%%%%%%%%%
%%                           Class Description                           %%
%%%%%%%%%%%%%%%%%%%%%%%%%%%%%%%%%%%%%%%%%%%%%%%%%%%%%%%%%%%%%%%%%%%%%%%%%%%

    \index{peach \textit{(package)}!peach.ga \textit{(package)}!peach.ga.base \textit{(module)}!peach.ga.base.GeneticAlgorithm \textit{(class)}|(}
\subsection{Class GeneticAlgorithm}

    \label{peach:ga:base:GeneticAlgorithm}
\begin{tabular}{cccccccc}
% Line for object, linespec=[False, False]
\multicolumn{2}{r}{\settowidth{\BCL}{object}\multirow{2}{\BCL}{object}}
&&
&&
  \\\cline{3-3}
  &&\multicolumn{1}{c|}{}
&&
&&
  \\
% Line for list, linespec=[False]
\multicolumn{4}{r}{\settowidth{\BCL}{list}\multirow{2}{\BCL}{list}}
&&
  \\\cline{5-5}
  &&&&\multicolumn{1}{c|}{}
&&
  \\
&&&&\multicolumn{2}{l}{\textbf{peach.ga.base.GeneticAlgorithm}}
\end{tabular}

\textbf{Known Subclasses:} peach.ga.base.GA


A standard Genetic Algorithm

This class implements the methods to generate, initialize and evolve a
population of chromosomes according to a given fitness function. A standard
GA implements, in this order:
%
\begin{quote}
%
\begin{itemize}

\item A selection method, to choose, from this generation, which individuals
will be present in the next generation;

\item A crossover method, to exchange information between selected individuals
to add diversity to the population;

\item A mutation method, to change information in a selected individual, also
to add diversity to the population;

\item The reinsertion of the fittest individual, if the population is elitist
(which is almost always the case).

\end{itemize}

\end{quote}

A population is actually a list of chromosomes, and individuals can be
read and set as in a normal list. Use the \texttt{{[} {]}} operators to access
individual chromosomes but please be aware that modifying the information on
the list before the end of convergence can cause unpredictable results. The
population and the algorithm have also other properties, check below to see
more information on them.

%%%%%%%%%%%%%%%%%%%%%%%%%%%%%%%%%%%%%%%%%%%%%%%%%%%%%%%%%%%%%%%%%%%%%%%%%%%
%%                                Methods                                %%
%%%%%%%%%%%%%%%%%%%%%%%%%%%%%%%%%%%%%%%%%%%%%%%%%%%%%%%%%%%%%%%%%%%%%%%%%%%

  \subsubsection{Methods}

    \vspace{0.5ex}

\hspace{.8\funcindent}\begin{boxedminipage}{\funcwidth}

    \raggedright \textbf{\_\_init\_\_}(\textit{self}, \textit{f}, \textit{x0}, \textit{ranges}={\tt \texttt{[}\texttt{]}}, \textit{fmt}={\tt \texttt{'}\texttt{f}\texttt{'}}, \textit{fitness}={\tt {\textless}class 'peach.ga.fitness.Fitness'{\textgreater}}, \textit{selection}={\tt {\textless}class 'peach.ga.selection.RouletteWheel'{\textgreater}}, \textit{crossover}={\tt {\textless}class 'peach.ga.crossover.TwoPoint'{\textgreater}}, \textit{mutation}={\tt {\textless}class 'peach.ga.mutation.BitToBit'{\textgreater}}, \textit{elitist}={\tt True})

    \vspace{-1.5ex}

    \rule{\textwidth}{0.5\fboxrule}
\setlength{\parskip}{2ex}

Initializes the population and the algorithm.

On the initialization of the population, a lot of parameters can be set.
Those will deeply affect the results. The parameters are:
\setlength{\parskip}{1ex}
      \textbf{Parameters}
      \vspace{-1ex}

      \begin{quote}
        \begin{Ventry}{xxxxxxxxx}

          \item[f]


A multivariable function to be evaluated. The nature of the
parameters in the objective function will depend of the way you want
the genetic algorithm to process. It can be a standard function that
receives a one-dimensional array of values and computes the value of
the function. In this case, the values will be passed as a tuple,
instead of an array. This is so that integer, floats and other types
of values can be passed and processed. In this case, the values will
depend of the format string (see below)

If you don't supply a format, your objective function will receive a
\texttt{Chromosome} instance, and it is the responsability of the
function to decode the array of bits in any way. Notice that, while
it is more flexible, it is certainly more difficult to deal with.
Your function should process the bits and compute the return value
which, in any case, should be a scalar.

Please, note that genetic algorithms maximize functions, so project
your objective function accordingly. If you want to minimize a
function, return its negated value.
          \item[x0]


A population of first estimates. This is a list, array or tuple of
one-dimension arrays, each one corresponding to an estimate of the
position of the minimum. The population size of the algorithm will
be the same as the number of estimates in this list. Each component
of the vectors in this list are one of the variables in the function
to be optimized.
          \item[ranges]


Since messing with the bits can change substantially the values
obtained can diverge a lot from the maximum point. To avoid this,
you can specify a range for each of the variables. \texttt{range}
defaults to \texttt{{[} {]}}, this means that no range checkin will be done.
If given, then every variable will be checked. There are two ways to
specify the ranges.

It might be a tuple of two values, \texttt{(x0, x1)}, where \texttt{x0} is the
start of the interval, and \texttt{x1} its end. Obviously, \texttt{x0} should
be smaller than \texttt{x1}. If \texttt{range} is given in this way, then this
range will be used for every variable.

It can be specified as a list of tuples with the same format as
given above. In that case, the list must have one range for every
variable specified in the format and the ranges must appear in the
same order as there. That is, every variable must have a range
associated to it.
          \item[fmt]


A \texttt{struct}-format string. The \texttt{struct} module is a standard
Python module that packs and unpacks informations in bits. These
are used to inform the algorithm what types of data are to be used.
For example, if you are maximizing a function of three real
variables, the format should be something like \texttt{"fff"}. Any type
supported by the \texttt{struct} module can be used. The GA will decode
the bit array according to this format and send it as is to your
fitness function -{}- your function \emph{must} know what to do with them.

Alternatively, the format can be an integer. In that case, the GA
will not try to decode the bit sequence. Instead, the bits are
passed without modification to the objective function, which must
deal with them. Notice that, if this is used this way, the
\texttt{ranges} property (see below) makes no sense, so it is set to
\texttt{None}. Also, no sanity checks will be performed.

It defaults to \texttt{``f''}, that is, a single floating point variable.
          \item[fitness]


A fitness method to be applied over the objective function. This
parameter must be a \texttt{Fitness} instance or subclass. It will be
applied over the objective function to compute the fitness of every
individual in the population. Please, see the documentation on the
\texttt{Fitness} class.
          \item[selection]


This specifies the selection method. You can use one given in the
\texttt{selection} sub-module, or you can implement your own. In any
case, the \texttt{selection} parameter must be an instance of
\texttt{Selection} or of a subclass. Please, see the documentation on the
\texttt{selection} module for more information. Defaults to
\texttt{RouletteWheel}. If made \texttt{None}, then selection will not be
present in the GA.
          \item[crossover]


This specifies the crossover method. You can use one given in the
\texttt{crossover} sub-module, or you can implement your own. In any
case, the \texttt{crossover} parameter must be an instance of
\texttt{Crossover} or of a subclass. Please, see the documentation on the
\texttt{crossover} module for more information. Defaults to
\texttt{TwoPoint}. If made \texttt{None}, then crossover will not be
present in the GA.
          \item[mutation]


This specifies the mutation method. You can use one given in the
\texttt{mutation} sub-module, or you can implement your own. In any
case, the \texttt{mutation} parameter must be an instance of \texttt{Mutation}
or of a subclass. Please, see the documentation on the \texttt{mutation}
module for more information. Defaults to \texttt{BitToBit}.  If made
\texttt{None}, then mutation will not be present in the GA.
          \item[elitist]


Defines if the population is elitist or not. An elitist population
will never discard the fittest individual when a new generation is
computed. Defaults to \texttt{True}.
        \end{Ventry}

      \end{quote}

      \textbf{Return Value}
    \vspace{-1ex}

      \begin{quote}

new empty list
      \end{quote}

      Overrides: object.\_\_init\_\_

    \end{boxedminipage}

    \label{peach:ga:base:GeneticAlgorithm:sanity}
    \index{peach \textit{(package)}!peach.ga \textit{(package)}!peach.ga.base \textit{(module)}!peach.ga.base.GeneticAlgorithm \textit{(class)}!peach.ga.base.GeneticAlgorithm.sanity \textit{(method)}}

    \vspace{0.5ex}

\hspace{.8\funcindent}\begin{boxedminipage}{\funcwidth}

    \raggedright \textbf{sanity}(\textit{self})

    \vspace{-1.5ex}

    \rule{\textwidth}{0.5\fboxrule}
\setlength{\parskip}{2ex}

Sanitizes the chromosomes in the population.

Since not every individual generated by the crossover and mutation
operations might be a valid result, this method verifies if they are
inside the allowed ranges (or if it is a number at all). Each invalid
individual is discarded and a new one is generated.

This method has no parameters and returns no values.
\setlength{\parskip}{1ex}
    \end{boxedminipage}

    \label{peach:ga:base:GeneticAlgorithm:restart}
    \index{peach \textit{(package)}!peach.ga \textit{(package)}!peach.ga.base \textit{(module)}!peach.ga.base.GeneticAlgorithm \textit{(class)}!peach.ga.base.GeneticAlgorithm.restart \textit{(method)}}

    \vspace{0.5ex}

\hspace{.8\funcindent}\begin{boxedminipage}{\funcwidth}

    \raggedright \textbf{restart}(\textit{self}, \textit{x0})

    \vspace{-1.5ex}

    \rule{\textwidth}{0.5\fboxrule}
\setlength{\parskip}{2ex}

Resets the optimizer, allowing the use of a new set of estimates. This
can be used to avoid stagnation.
\setlength{\parskip}{1ex}
      \textbf{Parameters}
      \vspace{-1ex}

      \begin{quote}
        \begin{Ventry}{xx}

          \item[x0]


A new set of estimates. It doesn't need to have the same size of the
original population, but it must be a list of estimates in the same
format as in the object instantiation. Please, see the documentation
on the instantiation of the class.
        \end{Ventry}

      \end{quote}

    \end{boxedminipage}

    \label{peach:ga:base:GeneticAlgorithm:step}
    \index{peach \textit{(package)}!peach.ga \textit{(package)}!peach.ga.base \textit{(module)}!peach.ga.base.GeneticAlgorithm \textit{(class)}!peach.ga.base.GeneticAlgorithm.step \textit{(method)}}

    \vspace{0.5ex}

\hspace{.8\funcindent}\begin{boxedminipage}{\funcwidth}

    \raggedright \textbf{step}(\textit{self})

    \vspace{-1.5ex}

    \rule{\textwidth}{0.5\fboxrule}
\setlength{\parskip}{2ex}

Computes a new generation of the population, a step of the adaptation.

This method goes through all the steps of the GA, as described above. If
the selection, crossover and mutation operators are defined, they are
applied over the population. If the population is elitist, then the
fittest individual of the past generation is reinserted.

This method has no parameters and returns no values. The GA itself can
be consulted (using \texttt{{[} {]}}) to find the fittest individual which is the
result of the process.
\setlength{\parskip}{1ex}
    \end{boxedminipage}

    \label{peach:ga:base:GeneticAlgorithm:__call__}
    \index{peach \textit{(package)}!peach.ga \textit{(package)}!peach.ga.base \textit{(module)}!peach.ga.base.GeneticAlgorithm \textit{(class)}!peach.ga.base.GeneticAlgorithm.\_\_call\_\_ \textit{(method)}}

    \vspace{0.5ex}

\hspace{.8\funcindent}\begin{boxedminipage}{\funcwidth}

    \raggedright \textbf{\_\_call\_\_}(\textit{self})

    \vspace{-1.5ex}

    \rule{\textwidth}{0.5\fboxrule}
\setlength{\parskip}{2ex}

Transparently executes the search until the minimum is found. The stop
criteria are the maximum error or the maximum number of iterations,
whichever is reached first. Note that this is a \texttt{\_\_call\_\_} method, so
the object is called as a function. This method returns a tuple
\texttt{(x, e)}, with the best estimate of the minimum and the error.
\setlength{\parskip}{1ex}
      \textbf{Return Value}
    \vspace{-1ex}

      \begin{quote}

This method returns a tuple \texttt{(x, e)}, where \texttt{x} is the best
estimate of the minimum, and \texttt{e} is the estimated error.
      \end{quote}

    \end{boxedminipage}


\large{\textbf{\textit{Inherited from list}}}

\begin{quote}
\_\_add\_\_(), \_\_contains\_\_(), \_\_delitem\_\_(), \_\_delslice\_\_(), \_\_eq\_\_(), \_\_ge\_\_(), \_\_getattribute\_\_(), \_\_getitem\_\_(), \_\_getslice\_\_(), \_\_gt\_\_(), \_\_iadd\_\_(), \_\_imul\_\_(), \_\_iter\_\_(), \_\_le\_\_(), \_\_len\_\_(), \_\_lt\_\_(), \_\_mul\_\_(), \_\_ne\_\_(), \_\_new\_\_(), \_\_repr\_\_(), \_\_reversed\_\_(), \_\_rmul\_\_(), \_\_setitem\_\_(), \_\_setslice\_\_(), \_\_sizeof\_\_(), append(), count(), extend(), index(), insert(), pop(), remove(), reverse(), sort()
\end{quote}

\large{\textbf{\textit{Inherited from object}}}

\begin{quote}
\_\_delattr\_\_(), \_\_format\_\_(), \_\_reduce\_\_(), \_\_reduce\_ex\_\_(), \_\_setattr\_\_(), \_\_str\_\_(), \_\_subclasshook\_\_()
\end{quote}

%%%%%%%%%%%%%%%%%%%%%%%%%%%%%%%%%%%%%%%%%%%%%%%%%%%%%%%%%%%%%%%%%%%%%%%%%%%
%%                              Properties                               %%
%%%%%%%%%%%%%%%%%%%%%%%%%%%%%%%%%%%%%%%%%%%%%%%%%%%%%%%%%%%%%%%%%%%%%%%%%%%

  \subsubsection{Properties}

    \vspace{-1cm}
\hspace{\varindent}\begin{longtable}{|p{\varnamewidth}|p{\vardescrwidth}|l}
\cline{1-2}
\cline{1-2} \centering \textbf{Name} & \centering \textbf{Description}& \\
\cline{1-2}
\endhead\cline{1-2}\multicolumn{3}{r}{\small\textit{continued on next page}}\\\endfoot\cline{1-2}
\endlastfoot\raggedright c\-h\-r\-o\-m\-o\-s\-o\-m\-e\-\_\-s\-i\-z\-e\- & &\\
\cline{1-2}
\raggedright f\-x\- & &\\
\cline{1-2}
\raggedright b\-e\-s\-t\- & &\\
\cline{1-2}
\raggedright f\-b\-e\-s\-t\- & &\\
\cline{1-2}
\raggedright f\-i\-t\-n\-e\-s\-s\- & &\\
\cline{1-2}
\multicolumn{2}{|l|}{\textit{Inherited from object}}\\
\multicolumn{2}{|p{\varwidth}|}{\raggedright \_\_class\_\_}\\
\cline{1-2}
\end{longtable}


%%%%%%%%%%%%%%%%%%%%%%%%%%%%%%%%%%%%%%%%%%%%%%%%%%%%%%%%%%%%%%%%%%%%%%%%%%%
%%                            Class Variables                            %%
%%%%%%%%%%%%%%%%%%%%%%%%%%%%%%%%%%%%%%%%%%%%%%%%%%%%%%%%%%%%%%%%%%%%%%%%%%%

  \subsubsection{Class Variables}

    \vspace{-1cm}
\hspace{\varindent}\begin{longtable}{|p{\varnamewidth}|p{\vardescrwidth}|l}
\cline{1-2}
\cline{1-2} \centering \textbf{Name} & \centering \textbf{Description}& \\
\cline{1-2}
\endhead\cline{1-2}\multicolumn{3}{r}{\small\textit{continued on next page}}\\\endfoot\cline{1-2}
\endlastfoot\multicolumn{2}{|l|}{\textit{Inherited from list}}\\
\multicolumn{2}{|p{\varwidth}|}{\raggedright \_\_hash\_\_}\\
\cline{1-2}
\end{longtable}


%%%%%%%%%%%%%%%%%%%%%%%%%%%%%%%%%%%%%%%%%%%%%%%%%%%%%%%%%%%%%%%%%%%%%%%%%%%
%%                          Instance Variables                           %%
%%%%%%%%%%%%%%%%%%%%%%%%%%%%%%%%%%%%%%%%%%%%%%%%%%%%%%%%%%%%%%%%%%%%%%%%%%%

  \subsubsection{Instance Variables}

    \vspace{-1cm}
\hspace{\varindent}\begin{longtable}{|p{\varnamewidth}|p{\vardescrwidth}|l}
\cline{1-2}
\cline{1-2} \centering \textbf{Name} & \centering \textbf{Description}& \\
\cline{1-2}
\endhead\cline{1-2}\multicolumn{3}{r}{\small\textit{continued on next page}}\\\endfoot\cline{1-2}
\endlastfoot\raggedright e\-l\-i\-t\-i\-s\-t\- & If \texttt{True}, then the population is elitist.&\\
\cline{1-2}
\raggedright r\-a\-n\-g\-e\-s\- & Holds the ranges for every variable. Although it is a
writable property, care should be taken in changing parameters
before ending the convergence.&\\
\cline{1-2}
\end{longtable}

    \index{peach \textit{(package)}!peach.ga \textit{(package)}!peach.ga.base \textit{(module)}!peach.ga.base.GeneticAlgorithm \textit{(class)}|)}

%%%%%%%%%%%%%%%%%%%%%%%%%%%%%%%%%%%%%%%%%%%%%%%%%%%%%%%%%%%%%%%%%%%%%%%%%%%
%%                           Class Description                           %%
%%%%%%%%%%%%%%%%%%%%%%%%%%%%%%%%%%%%%%%%%%%%%%%%%%%%%%%%%%%%%%%%%%%%%%%%%%%

    \index{peach \textit{(package)}!peach.ga \textit{(package)}!peach.ga.base \textit{(module)}!peach.ga.base.GA \textit{(class)}|(}
\subsection{Class GA}

    \label{peach:ga:base:GA}
\begin{tabular}{cccccccccc}
% Line for object, linespec=[False, False, False]
\multicolumn{2}{r}{\settowidth{\BCL}{object}\multirow{2}{\BCL}{object}}
&&
&&
&&
  \\\cline{3-3}
  &&\multicolumn{1}{c|}{}
&&
&&
&&
  \\
% Line for list, linespec=[False, False]
\multicolumn{4}{r}{\settowidth{\BCL}{list}\multirow{2}{\BCL}{list}}
&&
&&
  \\\cline{5-5}
  &&&&\multicolumn{1}{c|}{}
&&
&&
  \\
% Line for peach.ga.base.GeneticAlgorithm, linespec=[False]
\multicolumn{6}{r}{\settowidth{\BCL}{peach.ga.base.GeneticAlgorithm}\multirow{2}{\BCL}{peach.ga.base.GeneticAlgorithm}}
&&
  \\\cline{7-7}
  &&&&&&\multicolumn{1}{c|}{}
&&
  \\
&&&&&&\multicolumn{2}{l}{\textbf{peach.ga.base.GA}}
\end{tabular}


GA is an alias to \texttt{GeneticAlgorithm}

%%%%%%%%%%%%%%%%%%%%%%%%%%%%%%%%%%%%%%%%%%%%%%%%%%%%%%%%%%%%%%%%%%%%%%%%%%%
%%                                Methods                                %%
%%%%%%%%%%%%%%%%%%%%%%%%%%%%%%%%%%%%%%%%%%%%%%%%%%%%%%%%%%%%%%%%%%%%%%%%%%%

  \subsubsection{Methods}


\large{\textbf{\textit{Inherited from peach.ga.base.GeneticAlgorithm\textit{(Section \ref{peach:ga:base:GeneticAlgorithm})}}}}

\begin{quote}
\_\_call\_\_(), \_\_init\_\_(), restart(), sanity(), step()
\end{quote}

\large{\textbf{\textit{Inherited from list}}}

\begin{quote}
\_\_add\_\_(), \_\_contains\_\_(), \_\_delitem\_\_(), \_\_delslice\_\_(), \_\_eq\_\_(), \_\_ge\_\_(), \_\_getattribute\_\_(), \_\_getitem\_\_(), \_\_getslice\_\_(), \_\_gt\_\_(), \_\_iadd\_\_(), \_\_imul\_\_(), \_\_iter\_\_(), \_\_le\_\_(), \_\_len\_\_(), \_\_lt\_\_(), \_\_mul\_\_(), \_\_ne\_\_(), \_\_new\_\_(), \_\_repr\_\_(), \_\_reversed\_\_(), \_\_rmul\_\_(), \_\_setitem\_\_(), \_\_setslice\_\_(), \_\_sizeof\_\_(), append(), count(), extend(), index(), insert(), pop(), remove(), reverse(), sort()
\end{quote}

\large{\textbf{\textit{Inherited from object}}}

\begin{quote}
\_\_delattr\_\_(), \_\_format\_\_(), \_\_reduce\_\_(), \_\_reduce\_ex\_\_(), \_\_setattr\_\_(), \_\_str\_\_(), \_\_subclasshook\_\_()
\end{quote}

%%%%%%%%%%%%%%%%%%%%%%%%%%%%%%%%%%%%%%%%%%%%%%%%%%%%%%%%%%%%%%%%%%%%%%%%%%%
%%                              Properties                               %%
%%%%%%%%%%%%%%%%%%%%%%%%%%%%%%%%%%%%%%%%%%%%%%%%%%%%%%%%%%%%%%%%%%%%%%%%%%%

  \subsubsection{Properties}

    \vspace{-1cm}
\hspace{\varindent}\begin{longtable}{|p{\varnamewidth}|p{\vardescrwidth}|l}
\cline{1-2}
\cline{1-2} \centering \textbf{Name} & \centering \textbf{Description}& \\
\cline{1-2}
\endhead\cline{1-2}\multicolumn{3}{r}{\small\textit{continued on next page}}\\\endfoot\cline{1-2}
\endlastfoot\multicolumn{2}{|l|}{\textit{Inherited from peach.ga.base.GeneticAlgorithm \textit{(Section \ref{peach:ga:base:GeneticAlgorithm})}}}\\
\multicolumn{2}{|p{\varwidth}|}{\raggedright best, chromosome\_size, fbest, fitness, fx}\\
\cline{1-2}
\multicolumn{2}{|l|}{\textit{Inherited from object}}\\
\multicolumn{2}{|p{\varwidth}|}{\raggedright \_\_class\_\_}\\
\cline{1-2}
\end{longtable}


%%%%%%%%%%%%%%%%%%%%%%%%%%%%%%%%%%%%%%%%%%%%%%%%%%%%%%%%%%%%%%%%%%%%%%%%%%%
%%                            Class Variables                            %%
%%%%%%%%%%%%%%%%%%%%%%%%%%%%%%%%%%%%%%%%%%%%%%%%%%%%%%%%%%%%%%%%%%%%%%%%%%%

  \subsubsection{Class Variables}

    \vspace{-1cm}
\hspace{\varindent}\begin{longtable}{|p{\varnamewidth}|p{\vardescrwidth}|l}
\cline{1-2}
\cline{1-2} \centering \textbf{Name} & \centering \textbf{Description}& \\
\cline{1-2}
\endhead\cline{1-2}\multicolumn{3}{r}{\small\textit{continued on next page}}\\\endfoot\cline{1-2}
\endlastfoot\multicolumn{2}{|l|}{\textit{Inherited from list}}\\
\multicolumn{2}{|p{\varwidth}|}{\raggedright \_\_hash\_\_}\\
\cline{1-2}
\end{longtable}


%%%%%%%%%%%%%%%%%%%%%%%%%%%%%%%%%%%%%%%%%%%%%%%%%%%%%%%%%%%%%%%%%%%%%%%%%%%
%%                          Instance Variables                           %%
%%%%%%%%%%%%%%%%%%%%%%%%%%%%%%%%%%%%%%%%%%%%%%%%%%%%%%%%%%%%%%%%%%%%%%%%%%%

  \subsubsection{Instance Variables}

    \vspace{-1cm}
\hspace{\varindent}\begin{longtable}{|p{\varnamewidth}|p{\vardescrwidth}|l}
\cline{1-2}
\cline{1-2} \centering \textbf{Name} & \centering \textbf{Description}& \\
\cline{1-2}
\endhead\cline{1-2}\multicolumn{3}{r}{\small\textit{continued on next page}}\\\endfoot\cline{1-2}
\endlastfoot\multicolumn{2}{|l|}{\textit{Inherited from peach.ga.base.GeneticAlgorithm \textit{(Section \ref{peach:ga:base:GeneticAlgorithm})}}}\\
\multicolumn{2}{|p{\varwidth}|}{\raggedright elitist, ranges}\\
\cline{1-2}
\end{longtable}

    \index{peach \textit{(package)}!peach.ga \textit{(package)}!peach.ga.base \textit{(module)}!peach.ga.base.GA \textit{(class)}|)}
    \index{peach \textit{(package)}!peach.ga \textit{(package)}!peach.ga.base \textit{(module)}|)}
