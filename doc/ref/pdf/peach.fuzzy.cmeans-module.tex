%
% API Documentation for Peach - Computational Intelligence for Python
% Module peach.fuzzy.cmeans
%
% Generated by epydoc 3.0.1
% [Fri Feb  4 17:21:18 2011]
%

%%%%%%%%%%%%%%%%%%%%%%%%%%%%%%%%%%%%%%%%%%%%%%%%%%%%%%%%%%%%%%%%%%%%%%%%%%%
%%                          Module Description                           %%
%%%%%%%%%%%%%%%%%%%%%%%%%%%%%%%%%%%%%%%%%%%%%%%%%%%%%%%%%%%%%%%%%%%%%%%%%%%

    \index{peach \textit{(package)}!peach.fuzzy \textit{(package)}!peach.fuzzy.cmeans \textit{(module)}|(}
\section{Module peach.fuzzy.cmeans}

    \label{peach:fuzzy:cmeans}

Fuzzy C-Means

Fuzzy C-Means is a clustering algorithm based no fuzzy logic.

This package implements the fuzzy c-means algorithm for clustering and
classification. This algorithm is very simple, yet very efficient. From a
training set and an initial condition which gives the membership values of each
example in the training set to the clusters, it converges very fastly to crisper
sets.

The initial conditions, ie, the starting membership, must follow some rules.
Please, refer to any bibliography about the subject to see why. Those rules are:
no example might have membership 1 in every class, and the sum of the membership
of every component must be equal to 1. This means that the initial condition is
a fuzzy partition of the universe.

%%%%%%%%%%%%%%%%%%%%%%%%%%%%%%%%%%%%%%%%%%%%%%%%%%%%%%%%%%%%%%%%%%%%%%%%%%%
%%                               Variables                               %%
%%%%%%%%%%%%%%%%%%%%%%%%%%%%%%%%%%%%%%%%%%%%%%%%%%%%%%%%%%%%%%%%%%%%%%%%%%%

  \subsection{Variables}

    \vspace{-1cm}
\hspace{\varindent}\begin{longtable}{|p{\varnamewidth}|p{\vardescrwidth}|l}
\cline{1-2}
\cline{1-2} \centering \textbf{Name} & \centering \textbf{Description}& \\
\cline{1-2}
\endhead\cline{1-2}\multicolumn{3}{r}{\small\textit{continued on next page}}\\\endfoot\cline{1-2}
\endlastfoot\raggedright \_\-\_\-d\-o\-c\-\_\-\_\- & \raggedright \textbf{Value:} 
{\tt \texttt{...}}&\\
\cline{1-2}
\raggedright \_\-\_\-p\-a\-c\-k\-a\-g\-e\-\_\-\_\- & \raggedright \textbf{Value:} 
{\tt \texttt{'}\texttt{peach.fuzzy}\texttt{'}}&\\
\cline{1-2}
\end{longtable}


%%%%%%%%%%%%%%%%%%%%%%%%%%%%%%%%%%%%%%%%%%%%%%%%%%%%%%%%%%%%%%%%%%%%%%%%%%%
%%                           Class Description                           %%
%%%%%%%%%%%%%%%%%%%%%%%%%%%%%%%%%%%%%%%%%%%%%%%%%%%%%%%%%%%%%%%%%%%%%%%%%%%

    \index{peach \textit{(package)}!peach.fuzzy \textit{(package)}!peach.fuzzy.cmeans \textit{(module)}!peach.fuzzy.cmeans.FuzzyCMeans \textit{(class)}|(}
\subsection{Class FuzzyCMeans}

    \label{peach:fuzzy:cmeans:FuzzyCMeans}
\begin{tabular}{cccccc}
% Line for object, linespec=[False]
\multicolumn{2}{r}{\settowidth{\BCL}{object}\multirow{2}{\BCL}{object}}
&&
  \\\cline{3-3}
  &&\multicolumn{1}{c|}{}
&&
  \\
&&\multicolumn{2}{l}{\textbf{peach.fuzzy.cmeans.FuzzyCMeans}}
\end{tabular}


Fuzzy C-Means convergence.

Use this class to instantiate a fuzzy c-means object. The object must be
given a training set and initial conditions. The training set is a list or
an array of N-dimensional vectors; the initial conditions are a list of the
initial membership values for every vector in the training set -{}- thus, the
length of both lists must be the same. The number of columns in the initial
conditions must be the same number of classes. That is, if you are, for
example, classifying in \texttt{C} classes, then the initial conditions must have
\texttt{C} columns.

There are restrictions in the initial conditions: first, no column can be
all zeros or all ones -{}- if that happened, then the class described by this
column is unnecessary; second, the sum of the memberships of every example
must be one -{}- that is, the sum of the membership in every column in each
line must be one. This means that the initial condition is a perfect
partition of \texttt{C} subsets.

Notice, however, that \emph{no checking} is done. If your algorithm seems to be
behaving strangely, try to check these conditions.

%%%%%%%%%%%%%%%%%%%%%%%%%%%%%%%%%%%%%%%%%%%%%%%%%%%%%%%%%%%%%%%%%%%%%%%%%%%
%%                                Methods                                %%
%%%%%%%%%%%%%%%%%%%%%%%%%%%%%%%%%%%%%%%%%%%%%%%%%%%%%%%%%%%%%%%%%%%%%%%%%%%

  \subsubsection{Methods}

    \vspace{0.5ex}

\hspace{.8\funcindent}\begin{boxedminipage}{\funcwidth}

    \raggedright \textbf{\_\_init\_\_}(\textit{self}, \textit{training\_set}, \textit{initial\_conditions}, \textit{m}={\tt 2.0})

    \vspace{-1.5ex}

    \rule{\textwidth}{0.5\fboxrule}
\setlength{\parskip}{2ex}

Initializes the algorithm.
\setlength{\parskip}{1ex}
      \textbf{Parameters}
      \vspace{-1ex}

      \begin{quote}
        \begin{Ventry}{xxxxxxxxxxxxxxxxxx}

          \item[training\_set]


A list or array of vectors containing the data to be classified.
Each of the vectors in this list \emph{must} have the same dimension, or
the algorithm won't behave correctly. Notice that each vector can be
given as a tuple -{}- internally, everything is converted to arrays.
          \item[initial\_conditions]


A list or array of vectors containing the initial membership values
associated to each example in the training set. Each column of this
array contains the membership assigned to the corresponding class
for that vector. Notice that each vector can be given as a tuple -{}-
internally, everything is converted to arrays.
          \item[m]


This is the aggregation value. The bigger it is, the smoother will
be the classification. Please, consult the bibliography about the
subject. \texttt{m} must be bigger than 1. Its default value is 2
        \end{Ventry}

      \end{quote}

      Overrides: object.\_\_init\_\_

    \end{boxedminipage}

    \label{peach:fuzzy:cmeans:FuzzyCMeans:centers}
    \index{peach \textit{(package)}!peach.fuzzy \textit{(package)}!peach.fuzzy.cmeans \textit{(module)}!peach.fuzzy.cmeans.FuzzyCMeans \textit{(class)}!peach.fuzzy.cmeans.FuzzyCMeans.centers \textit{(method)}}

    \vspace{0.5ex}

\hspace{.8\funcindent}\begin{boxedminipage}{\funcwidth}

    \raggedright \textbf{centers}(\textit{self})

    \vspace{-1.5ex}

    \rule{\textwidth}{0.5\fboxrule}
\setlength{\parskip}{2ex}

Given the present state of the algorithm, recalculates the centers, that
is, the position of the vectors representing each of the classes. Notice
that this method modifies the state of the algorithm if any change was
made to any parameter. This method receives no arguments and will seldom
be used externally. It can be useful if you want to step over the
algorithm. \emph{This method has a colateral effect!} If you use it, the
\texttt{c} property (see above) will be modified.
\setlength{\parskip}{1ex}
      \textbf{Return Value}
    \vspace{-1ex}

      \begin{quote}

A vector containing, in each line, the position of the centers of the
algorithm.
      \end{quote}

    \end{boxedminipage}

    \label{peach:fuzzy:cmeans:FuzzyCMeans:membership}
    \index{peach \textit{(package)}!peach.fuzzy \textit{(package)}!peach.fuzzy.cmeans \textit{(module)}!peach.fuzzy.cmeans.FuzzyCMeans \textit{(class)}!peach.fuzzy.cmeans.FuzzyCMeans.membership \textit{(method)}}

    \vspace{0.5ex}

\hspace{.8\funcindent}\begin{boxedminipage}{\funcwidth}

    \raggedright \textbf{membership}(\textit{self})

    \vspace{-1.5ex}

    \rule{\textwidth}{0.5\fboxrule}
\setlength{\parskip}{2ex}

Given the present state of the algorithm, recalculates the membership of
each example on each class. That is, it modifies the initial conditions
to represent an evolved state of the algorithm. Notice that this method
modifies the state of the algorithm if any change was made to any
parameter.
\setlength{\parskip}{1ex}
      \textbf{Return Value}
    \vspace{-1ex}

      \begin{quote}

A vector containing, in each line, the membership of the corresponding
example in each class.
      \end{quote}

    \end{boxedminipage}

    \label{peach:fuzzy:cmeans:FuzzyCMeans:step}
    \index{peach \textit{(package)}!peach.fuzzy \textit{(package)}!peach.fuzzy.cmeans \textit{(module)}!peach.fuzzy.cmeans.FuzzyCMeans \textit{(class)}!peach.fuzzy.cmeans.FuzzyCMeans.step \textit{(method)}}

    \vspace{0.5ex}

\hspace{.8\funcindent}\begin{boxedminipage}{\funcwidth}

    \raggedright \textbf{step}(\textit{self})

    \vspace{-1.5ex}

    \rule{\textwidth}{0.5\fboxrule}
\setlength{\parskip}{2ex}

This method runs one step of the algorithm. It might be useful to track
the changes in the parameters.
\setlength{\parskip}{1ex}
      \textbf{Return Value}
    \vspace{-1ex}

      \begin{quote}

The norm of the change in the membership values of the examples. It
can be used to track convergence and as an estimate of the error.
      \end{quote}

    \end{boxedminipage}

    \label{peach:fuzzy:cmeans:FuzzyCMeans:__call__}
    \index{peach \textit{(package)}!peach.fuzzy \textit{(package)}!peach.fuzzy.cmeans \textit{(module)}!peach.fuzzy.cmeans.FuzzyCMeans \textit{(class)}!peach.fuzzy.cmeans.FuzzyCMeans.\_\_call\_\_ \textit{(method)}}

    \vspace{0.5ex}

\hspace{.8\funcindent}\begin{boxedminipage}{\funcwidth}

    \raggedright \textbf{\_\_call\_\_}(\textit{self}, \textit{emax}={\tt 1e-10}, \textit{imax}={\tt 20})

    \vspace{-1.5ex}

    \rule{\textwidth}{0.5\fboxrule}
\setlength{\parskip}{2ex}

The \texttt{\_\_call\_\_} interface is used to run the algorithm until
convergence is found.
\setlength{\parskip}{1ex}
      \textbf{Parameters}
      \vspace{-1ex}

      \begin{quote}
        \begin{Ventry}{xxxx}

          \item[emax]


Specifies the maximum error admitted in the execution of the
algorithm. It defaults to 1.e-10. The error is tracked according to
the norm returned by the \texttt{step()} method.
          \item[imax]


Specifies the maximum number of iterations admitted in the execution
of the algorithm. It defaults to 20.
        \end{Ventry}

      \end{quote}

      \textbf{Return Value}
    \vspace{-1ex}

      \begin{quote}

An array containing, at each line, the vectors representing the
centers of the clustered regions.
      \end{quote}

    \end{boxedminipage}


\large{\textbf{\textit{Inherited from object}}}

\begin{quote}
\_\_delattr\_\_(), \_\_format\_\_(), \_\_getattribute\_\_(), \_\_hash\_\_(), \_\_new\_\_(), \_\_reduce\_\_(), \_\_reduce\_ex\_\_(), \_\_repr\_\_(), \_\_setattr\_\_(), \_\_sizeof\_\_(), \_\_str\_\_(), \_\_subclasshook\_\_()
\end{quote}

%%%%%%%%%%%%%%%%%%%%%%%%%%%%%%%%%%%%%%%%%%%%%%%%%%%%%%%%%%%%%%%%%%%%%%%%%%%
%%                              Properties                               %%
%%%%%%%%%%%%%%%%%%%%%%%%%%%%%%%%%%%%%%%%%%%%%%%%%%%%%%%%%%%%%%%%%%%%%%%%%%%

  \subsubsection{Properties}

    \vspace{-1cm}
\hspace{\varindent}\begin{longtable}{|p{\varnamewidth}|p{\vardescrwidth}|l}
\cline{1-2}
\cline{1-2} \centering \textbf{Name} & \centering \textbf{Description}& \\
\cline{1-2}
\endhead\cline{1-2}\multicolumn{3}{r}{\small\textit{continued on next page}}\\\endfoot\cline{1-2}
\endlastfoot\raggedright c\- & &\\
\cline{1-2}
\raggedright m\-u\- & &\\
\cline{1-2}
\raggedright x\- & &\\
\cline{1-2}
\multicolumn{2}{|l|}{\textit{Inherited from object}}\\
\multicolumn{2}{|p{\varwidth}|}{\raggedright \_\_class\_\_}\\
\cline{1-2}
\end{longtable}


%%%%%%%%%%%%%%%%%%%%%%%%%%%%%%%%%%%%%%%%%%%%%%%%%%%%%%%%%%%%%%%%%%%%%%%%%%%
%%                          Instance Variables                           %%
%%%%%%%%%%%%%%%%%%%%%%%%%%%%%%%%%%%%%%%%%%%%%%%%%%%%%%%%%%%%%%%%%%%%%%%%%%%

  \subsubsection{Instance Variables}

    \vspace{-1cm}
\hspace{\varindent}\begin{longtable}{|p{\varnamewidth}|p{\vardescrwidth}|l}
\cline{1-2}
\cline{1-2} \centering \textbf{Name} & \centering \textbf{Description}& \\
\cline{1-2}
\endhead\cline{1-2}\multicolumn{3}{r}{\small\textit{continued on next page}}\\\endfoot\cline{1-2}
\endlastfoot\raggedright m\- & The fuzzyness coefficient. Must be bigger than 1, the closest it is
to 1, the smoother the membership curves will be.&\\
\cline{1-2}
\end{longtable}

    \index{peach \textit{(package)}!peach.fuzzy \textit{(package)}!peach.fuzzy.cmeans \textit{(module)}!peach.fuzzy.cmeans.FuzzyCMeans \textit{(class)}|)}
    \index{peach \textit{(package)}!peach.fuzzy \textit{(package)}!peach.fuzzy.cmeans \textit{(module)}|)}
