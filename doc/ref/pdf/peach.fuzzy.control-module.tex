%
% API Documentation for Peach - Computational Intelligence for Python
% Module peach.fuzzy.control
%
% Generated by epydoc 3.0.1
% [Thu Jul 28 16:37:45 2011]
%

%%%%%%%%%%%%%%%%%%%%%%%%%%%%%%%%%%%%%%%%%%%%%%%%%%%%%%%%%%%%%%%%%%%%%%%%%%%
%%                          Module Description                           %%
%%%%%%%%%%%%%%%%%%%%%%%%%%%%%%%%%%%%%%%%%%%%%%%%%%%%%%%%%%%%%%%%%%%%%%%%%%%

    \index{peach \textit{(package)}!peach.fuzzy \textit{(package)}!peach.fuzzy.control \textit{(module)}|(}
\section{Module peach.fuzzy.control}

    \label{peach:fuzzy:control}

This package implements fuzzy controllers, of fuzzy inference systems.

There are two types of controllers implemented in this package. The Mamdani
controller is the traditional approach, where input (or controlled) variables
are fuzzified, a set of decision rules determine the outcome in a fuzzified way,
and a defuzzification method is applied to obtain the numerical result.

The Sugeno controller operates in a similar way, but there is no defuzzification
step. Instead, the value of the output (or manipulated) variable is determined
by parametric models, and the final result is determined by a weighted average
based on the decision rules. This type of controller is also known as parametric
controller.

%%%%%%%%%%%%%%%%%%%%%%%%%%%%%%%%%%%%%%%%%%%%%%%%%%%%%%%%%%%%%%%%%%%%%%%%%%%
%%                               Variables                               %%
%%%%%%%%%%%%%%%%%%%%%%%%%%%%%%%%%%%%%%%%%%%%%%%%%%%%%%%%%%%%%%%%%%%%%%%%%%%

  \subsection{Variables}

    \vspace{-1cm}
\hspace{\varindent}\begin{longtable}{|p{\varnamewidth}|p{\vardescrwidth}|l}
\cline{1-2}
\cline{1-2} \centering \textbf{Name} & \centering \textbf{Description}& \\
\cline{1-2}
\endhead\cline{1-2}\multicolumn{3}{r}{\small\textit{continued on next page}}\\\endfoot\cline{1-2}
\endlastfoot\raggedright \_\-\_\-d\-o\-c\-\_\-\_\- & \raggedright \textbf{Value:} 
{\tt \texttt{...}}&\\
\cline{1-2}
\raggedright D\-R\-A\-S\-T\-I\-C\-\_\-N\-O\-R\-M\-S\- & \raggedright \textbf{Value:} 
{\tt DrasticProduct, DrasticSum, ZadehNot}&\\
\cline{1-2}
\raggedright E\-I\-N\-S\-T\-E\-I\-N\-\_\-N\-O\-R\-M\-S\- & \raggedright \textbf{Value:} 
{\tt EinsteinProduct, EinsteinSum, ZadehNot}&\\
\cline{1-2}
\raggedright M\-A\-M\-D\-A\-N\-I\-\_\-I\-N\-F\-E\-R\-E\-N\-C\-E\- & \raggedright \textbf{Value:} 
{\tt MamdaniImplication, MamdaniAglutination}&\\
\cline{1-2}
\raggedright P\-R\-O\-B\-\_\-I\-N\-F\-E\-R\-E\-N\-C\-E\- & \raggedright \textbf{Value:} 
{\tt ProbabilisticImplication, ProbabilisticAglutination}&\\
\cline{1-2}
\raggedright P\-R\-O\-B\-\_\-N\-O\-R\-M\-S\- & \raggedright \textbf{Value:} 
{\tt ProbabilisticAnd, ProbabilisticOr, ProbabilisticNot}&\\
\cline{1-2}
\raggedright Z\-A\-D\-E\-H\-\_\-N\-O\-R\-M\-S\- & \raggedright \textbf{Value:} 
{\tt ZadehAnd, ZadehOr, ZadehNot}&\\
\cline{1-2}
\raggedright \_\-\_\-p\-a\-c\-k\-a\-g\-e\-\_\-\_\- & \raggedright \textbf{Value:} 
{\tt \texttt{'}\texttt{peach.fuzzy}\texttt{'}}&\\
\cline{1-2}
\raggedright c\-o\-s\- & \raggedright \textbf{Value:} 
{\tt {\textless}ufunc 'cos'{\textgreater}}&\\
\cline{1-2}
\raggedright e\-x\-p\- & \raggedright \textbf{Value:} 
{\tt {\textless}ufunc 'exp'{\textgreater}}&\\
\cline{1-2}
\raggedright p\-i\- & \raggedright \textbf{Value:} 
{\tt 3.14159265359}&\\
\cline{1-2}
\end{longtable}


%%%%%%%%%%%%%%%%%%%%%%%%%%%%%%%%%%%%%%%%%%%%%%%%%%%%%%%%%%%%%%%%%%%%%%%%%%%
%%                           Class Description                           %%
%%%%%%%%%%%%%%%%%%%%%%%%%%%%%%%%%%%%%%%%%%%%%%%%%%%%%%%%%%%%%%%%%%%%%%%%%%%

    \index{peach \textit{(package)}!peach.fuzzy \textit{(package)}!peach.fuzzy.control \textit{(module)}!peach.fuzzy.control.Controller \textit{(class)}|(}
\subsection{Class Controller}

    \label{peach:fuzzy:control:Controller}
\begin{tabular}{cccccc}
% Line for object, linespec=[False]
\multicolumn{2}{r}{\settowidth{\BCL}{object}\multirow{2}{\BCL}{object}}
&&
  \\\cline{3-3}
  &&\multicolumn{1}{c|}{}
&&
  \\
&&\multicolumn{2}{l}{\textbf{peach.fuzzy.control.Controller}}
\end{tabular}

\textbf{Known Subclasses:} peach.fuzzy.control.Mamdani


Basic Mamdani controller

This class implements a standard Mamdani controller. A controller based on
fuzzy logic has a somewhat complex behaviour, so it is not explained here.
There are numerous references that can be consulted.

It is essential to understand the format that decision rules must follow to
obtain correct behaviour of the controller. A rule is a tuple given by:
%
\begin{quote}{\ttfamily \raggedright \noindent
((mx0,~mx1,~...,~mxn),~my)
}
\end{quote}

where \texttt{mx0} is a membership function of the first input variable, \texttt{mx1}
is a membership function of the second input variable and so on; and \texttt{my}
is a membership function or a fuzzy set of the output variable.

Notice that \texttt{mx}'s are \emph{functions} not fuzzy sets! They will be applied to
the values of the input variables given in the function call, so, if they
are anything different from a membership function, an exception will be
raised. Please, consult the examples to see how they must be used.

%%%%%%%%%%%%%%%%%%%%%%%%%%%%%%%%%%%%%%%%%%%%%%%%%%%%%%%%%%%%%%%%%%%%%%%%%%%
%%                                Methods                                %%
%%%%%%%%%%%%%%%%%%%%%%%%%%%%%%%%%%%%%%%%%%%%%%%%%%%%%%%%%%%%%%%%%%%%%%%%%%%

  \subsubsection{Methods}

    \vspace{0.5ex}

\hspace{.8\funcindent}\begin{boxedminipage}{\funcwidth}

    \raggedright \textbf{\_\_init\_\_}(\textit{self}, \textit{yrange}, \textit{rules}={\tt \texttt{[}\texttt{]}}, \textit{defuzzy}={\tt {\textless}function Centroid at 0x9891bfc{\textgreater}}, \textit{norm}={\tt {\textless}function ZadehAnd at 0x97d7a3c{\textgreater}}, \textit{conorm}={\tt {\textless}function ZadehOr at 0x97d7a74{\textgreater}}, \textit{negation}={\tt {\textless}function ZadehNot at 0x97d7aac{\textgreater}}, \textit{imply}={\tt {\textless}function MamdaniImplication at 0x97d7bfc{\textgreater}}, \textit{aglutinate}={\tt {\textless}function MamdaniAglutination at 0x97d7c34{\textgreater}})

    \vspace{-1.5ex}

    \rule{\textwidth}{0.5\fboxrule}
\setlength{\parskip}{2ex}

Creates and initialize the controller.
\setlength{\parskip}{1ex}
      \textbf{Parameters}
      \vspace{-1ex}

      \begin{quote}
        \begin{Ventry}{xxxxxxxx}

          \item[yrange]


The range of the output variable. This must be given as a set of
points belonging to the interval where the output variable is
defined, not only the start and end points. It is strongly suggested
that the interval is divided in some (eg.: 100) points equally
spaced;
          \item[rules]


The set of decision rules, as defined above. If none is given, an
empty set of rules is assumed;
          \item[defuzzy]


The defuzzification method to be used. If none is given, the
Centroid method is used;
          \item[norm]


The norm (\texttt{and} operation) to be used. Defaults to Zadeh and.
          \item[conorm]


The conorm (\texttt{or} operation) to be used. Defaults to Zadeh or.
          \item[negation]


The negation (\texttt{not} operation) to be used. Defaults to Zadeh not.
          \item[imply]


The implication method to be used. Defaults to Mamdani implication.          aglutinate
The aglutination method to be used. Defaults to Mamdani
aglutination.
        \end{Ventry}

      \end{quote}

      Overrides: object.\_\_init\_\_

    \end{boxedminipage}

    \label{peach:fuzzy:control:Controller:set_norm}
    \index{peach \textit{(package)}!peach.fuzzy \textit{(package)}!peach.fuzzy.control \textit{(module)}!peach.fuzzy.control.Controller \textit{(class)}!peach.fuzzy.control.Controller.set\_norm \textit{(method)}}

    \vspace{0.5ex}

\hspace{.8\funcindent}\begin{boxedminipage}{\funcwidth}

    \raggedright \textbf{set\_norm}(\textit{self}, \textit{f})

    \vspace{-1.5ex}

    \rule{\textwidth}{0.5\fboxrule}
\setlength{\parskip}{2ex}

Sets the norm (\texttt{and}) to be used.

This method must be used to change the behavior of the \texttt{and} operation
of the controller.
\setlength{\parskip}{1ex}
      \textbf{Parameters}
      \vspace{-1ex}

      \begin{quote}
        \begin{Ventry}{x}

          \item[f]


The function can be any function that takes two numerical values and
return one numerical value, that corresponds to the \texttt{and} result.
        \end{Ventry}

      \end{quote}

    \end{boxedminipage}

    \label{peach:fuzzy:control:Controller:set_conorm}
    \index{peach \textit{(package)}!peach.fuzzy \textit{(package)}!peach.fuzzy.control \textit{(module)}!peach.fuzzy.control.Controller \textit{(class)}!peach.fuzzy.control.Controller.set\_conorm \textit{(method)}}

    \vspace{0.5ex}

\hspace{.8\funcindent}\begin{boxedminipage}{\funcwidth}

    \raggedright \textbf{set\_conorm}(\textit{self}, \textit{f})

    \vspace{-1.5ex}

    \rule{\textwidth}{0.5\fboxrule}
\setlength{\parskip}{2ex}

Sets the conorm (\texttt{or}) to be used.

This method must be used to change the behavior of the \texttt{or} operation
of the controller.
\setlength{\parskip}{1ex}
      \textbf{Parameters}
      \vspace{-1ex}

      \begin{quote}
        \begin{Ventry}{x}

          \item[f]


The function can be any function that takes two numerical values and
return one numerical value, that corresponds to the \texttt{or} result.
        \end{Ventry}

      \end{quote}

    \end{boxedminipage}

    \label{peach:fuzzy:control:Controller:set_negation}
    \index{peach \textit{(package)}!peach.fuzzy \textit{(package)}!peach.fuzzy.control \textit{(module)}!peach.fuzzy.control.Controller \textit{(class)}!peach.fuzzy.control.Controller.set\_negation \textit{(method)}}

    \vspace{0.5ex}

\hspace{.8\funcindent}\begin{boxedminipage}{\funcwidth}

    \raggedright \textbf{set\_negation}(\textit{self}, \textit{f})

    \vspace{-1.5ex}

    \rule{\textwidth}{0.5\fboxrule}
\setlength{\parskip}{2ex}

Sets the negation (\texttt{not}) to be used.

This method must be used to change the behavior of the \texttt{not} operation
of the controller.
\setlength{\parskip}{1ex}
      \textbf{Parameters}
      \vspace{-1ex}

      \begin{quote}
        \begin{Ventry}{x}

          \item[f]


The function can be any function that takes one numerical value and
return one numerical value, that corresponds to the \texttt{not} result.
        \end{Ventry}

      \end{quote}

    \end{boxedminipage}

    \label{peach:fuzzy:control:Controller:set_implication}
    \index{peach \textit{(package)}!peach.fuzzy \textit{(package)}!peach.fuzzy.control \textit{(module)}!peach.fuzzy.control.Controller \textit{(class)}!peach.fuzzy.control.Controller.set\_implication \textit{(method)}}

    \vspace{0.5ex}

\hspace{.8\funcindent}\begin{boxedminipage}{\funcwidth}

    \raggedright \textbf{set\_implication}(\textit{self}, \textit{f})

    \vspace{-1.5ex}

    \rule{\textwidth}{0.5\fboxrule}
\setlength{\parskip}{2ex}

Sets the implication to be used.

This method must be used to change the behavior of the implication
operation of the controller.
\setlength{\parskip}{1ex}
      \textbf{Parameters}
      \vspace{-1ex}

      \begin{quote}
        \begin{Ventry}{x}

          \item[f]


The function can be any function that takes two numerical values and
return one numerical value, that corresponds to the implication
result.
        \end{Ventry}

      \end{quote}

    \end{boxedminipage}

    \label{peach:fuzzy:control:Controller:set_aglutination}
    \index{peach \textit{(package)}!peach.fuzzy \textit{(package)}!peach.fuzzy.control \textit{(module)}!peach.fuzzy.control.Controller \textit{(class)}!peach.fuzzy.control.Controller.set\_aglutination \textit{(method)}}

    \vspace{0.5ex}

\hspace{.8\funcindent}\begin{boxedminipage}{\funcwidth}

    \raggedright \textbf{set\_aglutination}(\textit{self}, \textit{f})

    \vspace{-1.5ex}

    \rule{\textwidth}{0.5\fboxrule}
\setlength{\parskip}{2ex}

Sets the aglutination to be used.

This method must be used to change the behavior of the aglutination
operation of the controller.
\setlength{\parskip}{1ex}
      \textbf{Parameters}
      \vspace{-1ex}

      \begin{quote}
        \begin{Ventry}{x}

          \item[f]


The function can be any function that takes two numerical values and
return one numerical value, that corresponds to the aglutination
result.
        \end{Ventry}

      \end{quote}

    \end{boxedminipage}

    \label{peach:fuzzy:control:Controller:add_rule}
    \index{peach \textit{(package)}!peach.fuzzy \textit{(package)}!peach.fuzzy.control \textit{(module)}!peach.fuzzy.control.Controller \textit{(class)}!peach.fuzzy.control.Controller.add\_rule \textit{(method)}}

    \vspace{0.5ex}

\hspace{.8\funcindent}\begin{boxedminipage}{\funcwidth}

    \raggedright \textbf{add\_rule}(\textit{self}, \textit{rule})

    \vspace{-1.5ex}

    \rule{\textwidth}{0.5\fboxrule}
\setlength{\parskip}{2ex}

Adds a decision rule to the knowledge base.

It is essential to understand the format that decision rules must follow
to obtain correct behaviour of the controller. A rule is a tuple must
have the following format:
%
\begin{quote}{\ttfamily \raggedright \noindent
((mx0,~mx1,~...,~mxn),~my)
}
\end{quote}

where \texttt{mx0} is a membership function of the first input variable,
\texttt{mx1} is a membership function of the second input variable and so on;
and \texttt{my} is a membership function or a fuzzy set of the output
variable.

Notice that \texttt{mx}'s are \emph{functions} not fuzzy sets! They will be
applied to the values of the input variables given in the function call,
so, if they are anything different from a membership function, an
exception will be raised when the controller is used. Please, consult
the examples to see how they must be used.
\setlength{\parskip}{1ex}
    \end{boxedminipage}

    \label{peach:fuzzy:control:Controller:add_table}
    \index{peach \textit{(package)}!peach.fuzzy \textit{(package)}!peach.fuzzy.control \textit{(module)}!peach.fuzzy.control.Controller \textit{(class)}!peach.fuzzy.control.Controller.add\_table \textit{(method)}}

    \vspace{0.5ex}

\hspace{.8\funcindent}\begin{boxedminipage}{\funcwidth}

    \raggedright \textbf{add\_table}(\textit{self}, \textit{lx1}, \textit{lx2}, \textit{table})

    \vspace{-1.5ex}

    \rule{\textwidth}{0.5\fboxrule}
\setlength{\parskip}{2ex}

Adds a table of decision rules in a two variable controller.

Typically, fuzzy controllers are used to control two variables. In that
case, the set of decision rules are given in the form of a table, since
that is a more compact format and very easy to visualize. This is a
convenience function that allows to add decision rules in the form of a
table. Notice that the resulting knowledge base will be the same if this
function is used or the \texttt{add\_rule} method is used with every single
rule. The second method is in general easier to read in a script, so
consider well.
\setlength{\parskip}{1ex}
      \textbf{Parameters}
      \vspace{-1ex}

      \begin{quote}
        \begin{Ventry}{xxxxx}

          \item[lx1]


The set of membership functions to the variable \texttt{x1}, or the
lines of the table
          \item[lx2]


The set of membership functions to the variable \texttt{x2}, or the
columns of the table
          \item[table]


The consequent of the rule where the condition is the line \texttt{and}
the column. These can be the membership functions or fuzzy sets.
        \end{Ventry}

      \end{quote}

    \end{boxedminipage}

    \label{peach:fuzzy:control:Controller:eval}
    \index{peach \textit{(package)}!peach.fuzzy \textit{(package)}!peach.fuzzy.control \textit{(module)}!peach.fuzzy.control.Controller \textit{(class)}!peach.fuzzy.control.Controller.eval \textit{(method)}}

    \vspace{0.5ex}

\hspace{.8\funcindent}\begin{boxedminipage}{\funcwidth}

    \raggedright \textbf{eval}(\textit{self}, \textit{r}, \textit{xs})

    \vspace{-1.5ex}

    \rule{\textwidth}{0.5\fboxrule}
\setlength{\parskip}{2ex}

Evaluates one decision rule in this controller

Takes a rule from the controller and evaluates it given the values of
the input variables.
\setlength{\parskip}{1ex}
      \textbf{Parameters}
      \vspace{-1ex}

      \begin{quote}
        \begin{Ventry}{xx}

          \item[r]


The rule in the standard format, or an integer number. If \texttt{r} is
an integer, then the \texttt{r} th rule in the knowledge base will be
evaluated.
          \item[xs]


A tuple, a list or an array containing the values of the input
variables. The dimension must be coherent with the given rule.
        \end{Ventry}

      \end{quote}

      \textbf{Return Value}
    \vspace{-1ex}

      \begin{quote}

This method evaluates each membership function in the rule for each
given value, and \texttt{and} 's the results to obtain the condition. If
the condition is zero, a tuple \texttt{(0.0, None) is returned. Otherwise,
the condition is `{}`imply} ed in the membership function of the output
variable. A tuple containing \texttt{(condition, imply)} (the membership
value associated to the condition and the result of the implication)
is returned.
      \end{quote}

    \end{boxedminipage}

    \label{peach:fuzzy:control:Controller:eval_all}
    \index{peach \textit{(package)}!peach.fuzzy \textit{(package)}!peach.fuzzy.control \textit{(module)}!peach.fuzzy.control.Controller \textit{(class)}!peach.fuzzy.control.Controller.eval\_all \textit{(method)}}

    \vspace{0.5ex}

\hspace{.8\funcindent}\begin{boxedminipage}{\funcwidth}

    \raggedright \textbf{eval\_all}(\textit{self}, *\textit{xs})

    \vspace{-1.5ex}

    \rule{\textwidth}{0.5\fboxrule}
\setlength{\parskip}{2ex}

Evaluates all the rules and aglutinates the results.

Given the values of the input variables, evaluate and apply every rule
in the knowledge base (with the \texttt{eval} method) and aglutinates the
results.
\setlength{\parskip}{1ex}
      \textbf{Parameters}
      \vspace{-1ex}

      \begin{quote}
        \begin{Ventry}{xx}

          \item[xs]


A tuple, a list or an array with the values of the input variables.
        \end{Ventry}

      \end{quote}

      \textbf{Return Value}
    \vspace{-1ex}

      \begin{quote}

A fuzzy set containing the result of the evaluation of every rule in
the knowledge base, with the results aglutinated.
      \end{quote}

    \end{boxedminipage}

    \label{peach:fuzzy:control:Controller:__call__}
    \index{peach \textit{(package)}!peach.fuzzy \textit{(package)}!peach.fuzzy.control \textit{(module)}!peach.fuzzy.control.Controller \textit{(class)}!peach.fuzzy.control.Controller.\_\_call\_\_ \textit{(method)}}

    \vspace{0.5ex}

\hspace{.8\funcindent}\begin{boxedminipage}{\funcwidth}

    \raggedright \textbf{\_\_call\_\_}(\textit{self}, *\textit{xs})

    \vspace{-1.5ex}

    \rule{\textwidth}{0.5\fboxrule}
\setlength{\parskip}{2ex}

Apply the controller to the set of input variables

Given the values of the input variables, evaluates every decision rule,
aglutinates the results and defuzzify it. Returns the response of the
controller.
\setlength{\parskip}{1ex}
      \textbf{Parameters}
      \vspace{-1ex}

      \begin{quote}
        \begin{Ventry}{xx}

          \item[xs]


A tuple, a list or an array with the values of the input variables.
        \end{Ventry}

      \end{quote}

      \textbf{Return Value}
    \vspace{-1ex}

      \begin{quote}

The response of the controller.
      \end{quote}

    \end{boxedminipage}


\large{\textbf{\textit{Inherited from object}}}

\begin{quote}
\_\_delattr\_\_(), \_\_format\_\_(), \_\_getattribute\_\_(), \_\_hash\_\_(), \_\_new\_\_(), \_\_reduce\_\_(), \_\_reduce\_ex\_\_(), \_\_repr\_\_(), \_\_setattr\_\_(), \_\_sizeof\_\_(), \_\_str\_\_(), \_\_subclasshook\_\_()
\end{quote}

%%%%%%%%%%%%%%%%%%%%%%%%%%%%%%%%%%%%%%%%%%%%%%%%%%%%%%%%%%%%%%%%%%%%%%%%%%%
%%                              Properties                               %%
%%%%%%%%%%%%%%%%%%%%%%%%%%%%%%%%%%%%%%%%%%%%%%%%%%%%%%%%%%%%%%%%%%%%%%%%%%%

  \subsubsection{Properties}

    \vspace{-1cm}
\hspace{\varindent}\begin{longtable}{|p{\varnamewidth}|p{\vardescrwidth}|l}
\cline{1-2}
\cline{1-2} \centering \textbf{Name} & \centering \textbf{Description}& \\
\cline{1-2}
\endhead\cline{1-2}\multicolumn{3}{r}{\small\textit{continued on next page}}\\\endfoot\cline{1-2}
\endlastfoot\raggedright y\- & &\\
\cline{1-2}
\raggedright r\-u\-l\-e\-s\- & &\\
\cline{1-2}
\multicolumn{2}{|l|}{\textit{Inherited from object}}\\
\multicolumn{2}{|p{\varwidth}|}{\raggedright \_\_class\_\_}\\
\cline{1-2}
\end{longtable}

    \index{peach \textit{(package)}!peach.fuzzy \textit{(package)}!peach.fuzzy.control \textit{(module)}!peach.fuzzy.control.Controller \textit{(class)}|)}

%%%%%%%%%%%%%%%%%%%%%%%%%%%%%%%%%%%%%%%%%%%%%%%%%%%%%%%%%%%%%%%%%%%%%%%%%%%
%%                           Class Description                           %%
%%%%%%%%%%%%%%%%%%%%%%%%%%%%%%%%%%%%%%%%%%%%%%%%%%%%%%%%%%%%%%%%%%%%%%%%%%%

    \index{peach \textit{(package)}!peach.fuzzy \textit{(package)}!peach.fuzzy.control \textit{(module)}!peach.fuzzy.control.Mamdani \textit{(class)}|(}
\subsection{Class Mamdani}

    \label{peach:fuzzy:control:Mamdani}
\begin{tabular}{cccccccc}
% Line for object, linespec=[False, False]
\multicolumn{2}{r}{\settowidth{\BCL}{object}\multirow{2}{\BCL}{object}}
&&
&&
  \\\cline{3-3}
  &&\multicolumn{1}{c|}{}
&&
&&
  \\
% Line for peach.fuzzy.control.Controller, linespec=[False]
\multicolumn{4}{r}{\settowidth{\BCL}{peach.fuzzy.control.Controller}\multirow{2}{\BCL}{peach.fuzzy.control.Controller}}
&&
  \\\cline{5-5}
  &&&&\multicolumn{1}{c|}{}
&&
  \\
&&&&\multicolumn{2}{l}{\textbf{peach.fuzzy.control.Mamdani}}
\end{tabular}


\texttt{Mandani} is an alias to \texttt{Controller}

%%%%%%%%%%%%%%%%%%%%%%%%%%%%%%%%%%%%%%%%%%%%%%%%%%%%%%%%%%%%%%%%%%%%%%%%%%%
%%                                Methods                                %%
%%%%%%%%%%%%%%%%%%%%%%%%%%%%%%%%%%%%%%%%%%%%%%%%%%%%%%%%%%%%%%%%%%%%%%%%%%%

  \subsubsection{Methods}


\large{\textbf{\textit{Inherited from peach.fuzzy.control.Controller\textit{(Section \ref{peach:fuzzy:control:Controller})}}}}

\begin{quote}
\_\_call\_\_(), \_\_init\_\_(), add\_rule(), add\_table(), eval(), eval\_all(), set\_aglutination(), set\_conorm(), set\_implication(), set\_negation(), set\_norm()
\end{quote}

\large{\textbf{\textit{Inherited from object}}}

\begin{quote}
\_\_delattr\_\_(), \_\_format\_\_(), \_\_getattribute\_\_(), \_\_hash\_\_(), \_\_new\_\_(), \_\_reduce\_\_(), \_\_reduce\_ex\_\_(), \_\_repr\_\_(), \_\_setattr\_\_(), \_\_sizeof\_\_(), \_\_str\_\_(), \_\_subclasshook\_\_()
\end{quote}

%%%%%%%%%%%%%%%%%%%%%%%%%%%%%%%%%%%%%%%%%%%%%%%%%%%%%%%%%%%%%%%%%%%%%%%%%%%
%%                              Properties                               %%
%%%%%%%%%%%%%%%%%%%%%%%%%%%%%%%%%%%%%%%%%%%%%%%%%%%%%%%%%%%%%%%%%%%%%%%%%%%

  \subsubsection{Properties}

    \vspace{-1cm}
\hspace{\varindent}\begin{longtable}{|p{\varnamewidth}|p{\vardescrwidth}|l}
\cline{1-2}
\cline{1-2} \centering \textbf{Name} & \centering \textbf{Description}& \\
\cline{1-2}
\endhead\cline{1-2}\multicolumn{3}{r}{\small\textit{continued on next page}}\\\endfoot\cline{1-2}
\endlastfoot\multicolumn{2}{|l|}{\textit{Inherited from peach.fuzzy.control.Controller \textit{(Section \ref{peach:fuzzy:control:Controller})}}}\\
\multicolumn{2}{|p{\varwidth}|}{\raggedright rules, y}\\
\cline{1-2}
\multicolumn{2}{|l|}{\textit{Inherited from object}}\\
\multicolumn{2}{|p{\varwidth}|}{\raggedright \_\_class\_\_}\\
\cline{1-2}
\end{longtable}

    \index{peach \textit{(package)}!peach.fuzzy \textit{(package)}!peach.fuzzy.control \textit{(module)}!peach.fuzzy.control.Mamdani \textit{(class)}|)}

%%%%%%%%%%%%%%%%%%%%%%%%%%%%%%%%%%%%%%%%%%%%%%%%%%%%%%%%%%%%%%%%%%%%%%%%%%%
%%                           Class Description                           %%
%%%%%%%%%%%%%%%%%%%%%%%%%%%%%%%%%%%%%%%%%%%%%%%%%%%%%%%%%%%%%%%%%%%%%%%%%%%

    \index{peach \textit{(package)}!peach.fuzzy \textit{(package)}!peach.fuzzy.control \textit{(module)}!peach.fuzzy.control.Parametric \textit{(class)}|(}
\subsection{Class Parametric}

    \label{peach:fuzzy:control:Parametric}
\begin{tabular}{cccccc}
% Line for object, linespec=[False]
\multicolumn{2}{r}{\settowidth{\BCL}{object}\multirow{2}{\BCL}{object}}
&&
  \\\cline{3-3}
  &&\multicolumn{1}{c|}{}
&&
  \\
&&\multicolumn{2}{l}{\textbf{peach.fuzzy.control.Parametric}}
\end{tabular}

\textbf{Known Subclasses:} peach.fuzzy.control.Sugeno


Basic Parametric controller

This class implements a standard parametric (or Takagi-Sugeno) controller. A
controller based on fuzzy logic has a somewhat complex behaviour, so it is
not explained here. There are numerous references that can be consulted.

It is essential to understand the format that decision rules must follow to
obtain correct behaviour of the controller. A rule is a tuple given by:
%
\begin{quote}{\ttfamily \raggedright \noindent
((mx0,~mx1,~...,~mxn),~(a0,~a1,~...,~an))
}
\end{quote}

where \texttt{mx0} is a membership function of the first input variable, \texttt{mx1}
is a membership function of the second input variable and so on; and \texttt{a0}
is the linear parameter, \texttt{a1} is the parameter associated with the first
input variable, \texttt{a2} is the parameter associated with the second input
variable and so on. The response to the rule is calculated by:
%
\begin{quote}{\ttfamily \raggedright \noindent
y~=~a0~+~a1*x1~+~a2*x2~+~...~+~an*xn
}
\end{quote}

Notice that \texttt{mx}'s are \emph{functions} not fuzzy sets! They will be applied to
the values of the input variables given in the function call, so, if they
are anything different from a membership function, an exception will be
raised. Please, consult the examples to see how they must be used.

%%%%%%%%%%%%%%%%%%%%%%%%%%%%%%%%%%%%%%%%%%%%%%%%%%%%%%%%%%%%%%%%%%%%%%%%%%%
%%                                Methods                                %%
%%%%%%%%%%%%%%%%%%%%%%%%%%%%%%%%%%%%%%%%%%%%%%%%%%%%%%%%%%%%%%%%%%%%%%%%%%%

  \subsubsection{Methods}

    \vspace{0.5ex}

\hspace{.8\funcindent}\begin{boxedminipage}{\funcwidth}

    \raggedright \textbf{\_\_init\_\_}(\textit{self}, \textit{rules}={\tt \texttt{[}\texttt{]}}, \textit{norm}={\tt {\textless}function ProbabilisticAnd at 0x97d7c6c{\textgreater}}, \textit{conorm}={\tt {\textless}function ProbabilisticOr at 0x97d7ca4{\textgreater}}, \textit{negation}={\tt {\textless}function ProbabilisticNot at 0x97d7cdc{\textgreater}})

    \vspace{-1.5ex}

    \rule{\textwidth}{0.5\fboxrule}
\setlength{\parskip}{2ex}

Creates and initializes the controller.
\setlength{\parskip}{1ex}
      \textbf{Parameters}
      \vspace{-1ex}

      \begin{quote}
        \begin{Ventry}{xxxxxxxx}

          \item[rules]


List containing the decision rules for the controller. If not given,
an empty set of decision rules is used.
          \item[norm]


The norm (\texttt{and} operation) to be used. Defaults to Probabilistic
and.
          \item[conorm]


The conorm (\texttt{or} operation) to be used. Defaults to Probabilistic
or.
          \item[negation]


The negation (\texttt{not} operation) to be used. Defaults to
Probabilistic not.
        \end{Ventry}

      \end{quote}

      Overrides: object.\_\_init\_\_

    \end{boxedminipage}

    \label{peach:fuzzy:control:Parametric:add_rule}
    \index{peach \textit{(package)}!peach.fuzzy \textit{(package)}!peach.fuzzy.control \textit{(module)}!peach.fuzzy.control.Parametric \textit{(class)}!peach.fuzzy.control.Parametric.add\_rule \textit{(method)}}

    \vspace{0.5ex}

\hspace{.8\funcindent}\begin{boxedminipage}{\funcwidth}

    \raggedright \textbf{add\_rule}(\textit{self}, \textit{rule})

    \vspace{-1.5ex}

    \rule{\textwidth}{0.5\fboxrule}
\setlength{\parskip}{2ex}

Adds a decision rule to the knowledge base.

It is essential to understand the format that decision rules must follow
to obtain correct behaviour of the controller. A rule is a tuple given
by:
%
\begin{quote}{\ttfamily \raggedright \noindent
((mx0,~mx1,~...,~mxn),~(a0,~a1,~...,~an))
}
\end{quote}

where \texttt{mx0} is a membership function of the first input variable,
\texttt{mx1} is a membership function of the second input variable and so on;
and \texttt{a0} is the linear parameter, \texttt{a1} is the parameter associated
with the first input variable, \texttt{a2} is the parameter associated with
the second input variable and so on.

Notice that \texttt{mx}'s are \emph{functions} not fuzzy sets! They will be
applied to the values of the input variables given in the function call,
so, if they are anything different from a membership function, an
exception will be raised. Please, consult the examples to see how they
must be used.
\setlength{\parskip}{1ex}
    \end{boxedminipage}

    \label{peach:fuzzy:control:Parametric:eval}
    \index{peach \textit{(package)}!peach.fuzzy \textit{(package)}!peach.fuzzy.control \textit{(module)}!peach.fuzzy.control.Parametric \textit{(class)}!peach.fuzzy.control.Parametric.eval \textit{(method)}}

    \vspace{0.5ex}

\hspace{.8\funcindent}\begin{boxedminipage}{\funcwidth}

    \raggedright \textbf{eval}(\textit{self}, \textit{r}, \textit{xs})

    \vspace{-1.5ex}

    \rule{\textwidth}{0.5\fboxrule}
\setlength{\parskip}{2ex}

Evaluates one decision rule in this controller

Takes a rule from the controller and evaluates it given the values of
the input variables. The format of the rule is as given, and the
response to the rule is calculated by:
%
\begin{quote}{\ttfamily \raggedright \noindent
y~=~a0~+~a1*x1~+~a2*x2~+~...~+~an*xn
}
\end{quote}
\setlength{\parskip}{1ex}
      \textbf{Parameters}
      \vspace{-1ex}

      \begin{quote}
        \begin{Ventry}{xx}

          \item[r]


The rule in the standard format, or an integer number. If \texttt{r} is
an integer, then the \texttt{r} th rule in the knowledge base will be
evaluated.
          \item[xs]


A tuple, a list or an array containing the values of the input
variables. The dimension must be coherent with the given rule.
        \end{Ventry}

      \end{quote}

      \textbf{Return Value}
    \vspace{-1ex}

      \begin{quote}

This method evaluates each membership function in the rule for each
given value, and \texttt{and} 's the results to obtain the condition. If
the condition is zero, a tuple \texttt{(0.0, 0.0) is returned. Otherwise,
the result as given above is calculate, and a tuple containing
`{}`(condition, result)} (the membership value associated to the
condition and the result of the calculation) is returned.
      \end{quote}

    \end{boxedminipage}

    \label{peach:fuzzy:control:Parametric:__call__}
    \index{peach \textit{(package)}!peach.fuzzy \textit{(package)}!peach.fuzzy.control \textit{(module)}!peach.fuzzy.control.Parametric \textit{(class)}!peach.fuzzy.control.Parametric.\_\_call\_\_ \textit{(method)}}

    \vspace{0.5ex}

\hspace{.8\funcindent}\begin{boxedminipage}{\funcwidth}

    \raggedright \textbf{\_\_call\_\_}(\textit{self}, *\textit{xs})

    \vspace{-1.5ex}

    \rule{\textwidth}{0.5\fboxrule}
\setlength{\parskip}{2ex}

Apply the controller to the set of input variables

Given the values of the input variables, evaluates every decision rule,
and calculates the weighted average of the results. Returns the response
of the controller.
\setlength{\parskip}{1ex}
      \textbf{Parameters}
      \vspace{-1ex}

      \begin{quote}
        \begin{Ventry}{xx}

          \item[xs]


A tuple, a list or an array with the values of the input variables.
        \end{Ventry}

      \end{quote}

      \textbf{Return Value}
    \vspace{-1ex}

      \begin{quote}

The response of the controller.
      \end{quote}

    \end{boxedminipage}


\large{\textbf{\textit{Inherited from object}}}

\begin{quote}
\_\_delattr\_\_(), \_\_format\_\_(), \_\_getattribute\_\_(), \_\_hash\_\_(), \_\_new\_\_(), \_\_reduce\_\_(), \_\_reduce\_ex\_\_(), \_\_repr\_\_(), \_\_setattr\_\_(), \_\_sizeof\_\_(), \_\_str\_\_(), \_\_subclasshook\_\_()
\end{quote}

%%%%%%%%%%%%%%%%%%%%%%%%%%%%%%%%%%%%%%%%%%%%%%%%%%%%%%%%%%%%%%%%%%%%%%%%%%%
%%                              Properties                               %%
%%%%%%%%%%%%%%%%%%%%%%%%%%%%%%%%%%%%%%%%%%%%%%%%%%%%%%%%%%%%%%%%%%%%%%%%%%%

  \subsubsection{Properties}

    \vspace{-1cm}
\hspace{\varindent}\begin{longtable}{|p{\varnamewidth}|p{\vardescrwidth}|l}
\cline{1-2}
\cline{1-2} \centering \textbf{Name} & \centering \textbf{Description}& \\
\cline{1-2}
\endhead\cline{1-2}\multicolumn{3}{r}{\small\textit{continued on next page}}\\\endfoot\cline{1-2}
\endlastfoot\raggedright r\-u\-l\-e\-s\- & &\\
\cline{1-2}
\multicolumn{2}{|l|}{\textit{Inherited from object}}\\
\multicolumn{2}{|p{\varwidth}|}{\raggedright \_\_class\_\_}\\
\cline{1-2}
\end{longtable}

    \index{peach \textit{(package)}!peach.fuzzy \textit{(package)}!peach.fuzzy.control \textit{(module)}!peach.fuzzy.control.Parametric \textit{(class)}|)}

%%%%%%%%%%%%%%%%%%%%%%%%%%%%%%%%%%%%%%%%%%%%%%%%%%%%%%%%%%%%%%%%%%%%%%%%%%%
%%                           Class Description                           %%
%%%%%%%%%%%%%%%%%%%%%%%%%%%%%%%%%%%%%%%%%%%%%%%%%%%%%%%%%%%%%%%%%%%%%%%%%%%

    \index{peach \textit{(package)}!peach.fuzzy \textit{(package)}!peach.fuzzy.control \textit{(module)}!peach.fuzzy.control.Sugeno \textit{(class)}|(}
\subsection{Class Sugeno}

    \label{peach:fuzzy:control:Sugeno}
\begin{tabular}{cccccccc}
% Line for object, linespec=[False, False]
\multicolumn{2}{r}{\settowidth{\BCL}{object}\multirow{2}{\BCL}{object}}
&&
&&
  \\\cline{3-3}
  &&\multicolumn{1}{c|}{}
&&
&&
  \\
% Line for peach.fuzzy.control.Parametric, linespec=[False]
\multicolumn{4}{r}{\settowidth{\BCL}{peach.fuzzy.control.Parametric}\multirow{2}{\BCL}{peach.fuzzy.control.Parametric}}
&&
  \\\cline{5-5}
  &&&&\multicolumn{1}{c|}{}
&&
  \\
&&&&\multicolumn{2}{l}{\textbf{peach.fuzzy.control.Sugeno}}
\end{tabular}


\texttt{Sugeno} is an alias to \texttt{Parametric}

%%%%%%%%%%%%%%%%%%%%%%%%%%%%%%%%%%%%%%%%%%%%%%%%%%%%%%%%%%%%%%%%%%%%%%%%%%%
%%                                Methods                                %%
%%%%%%%%%%%%%%%%%%%%%%%%%%%%%%%%%%%%%%%%%%%%%%%%%%%%%%%%%%%%%%%%%%%%%%%%%%%

  \subsubsection{Methods}


\large{\textbf{\textit{Inherited from peach.fuzzy.control.Parametric\textit{(Section \ref{peach:fuzzy:control:Parametric})}}}}

\begin{quote}
\_\_call\_\_(), \_\_init\_\_(), add\_rule(), eval()
\end{quote}

\large{\textbf{\textit{Inherited from object}}}

\begin{quote}
\_\_delattr\_\_(), \_\_format\_\_(), \_\_getattribute\_\_(), \_\_hash\_\_(), \_\_new\_\_(), \_\_reduce\_\_(), \_\_reduce\_ex\_\_(), \_\_repr\_\_(), \_\_setattr\_\_(), \_\_sizeof\_\_(), \_\_str\_\_(), \_\_subclasshook\_\_()
\end{quote}

%%%%%%%%%%%%%%%%%%%%%%%%%%%%%%%%%%%%%%%%%%%%%%%%%%%%%%%%%%%%%%%%%%%%%%%%%%%
%%                              Properties                               %%
%%%%%%%%%%%%%%%%%%%%%%%%%%%%%%%%%%%%%%%%%%%%%%%%%%%%%%%%%%%%%%%%%%%%%%%%%%%

  \subsubsection{Properties}

    \vspace{-1cm}
\hspace{\varindent}\begin{longtable}{|p{\varnamewidth}|p{\vardescrwidth}|l}
\cline{1-2}
\cline{1-2} \centering \textbf{Name} & \centering \textbf{Description}& \\
\cline{1-2}
\endhead\cline{1-2}\multicolumn{3}{r}{\small\textit{continued on next page}}\\\endfoot\cline{1-2}
\endlastfoot\multicolumn{2}{|l|}{\textit{Inherited from peach.fuzzy.control.Parametric \textit{(Section \ref{peach:fuzzy:control:Parametric})}}}\\
\multicolumn{2}{|p{\varwidth}|}{\raggedright rules}\\
\cline{1-2}
\multicolumn{2}{|l|}{\textit{Inherited from object}}\\
\multicolumn{2}{|p{\varwidth}|}{\raggedright \_\_class\_\_}\\
\cline{1-2}
\end{longtable}

    \index{peach \textit{(package)}!peach.fuzzy \textit{(package)}!peach.fuzzy.control \textit{(module)}!peach.fuzzy.control.Sugeno \textit{(class)}|)}
    \index{peach \textit{(package)}!peach.fuzzy \textit{(package)}!peach.fuzzy.control \textit{(module)}|)}
