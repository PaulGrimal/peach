%
% API Documentation for Peach - Computational Intelligence for Python
% Module peach.fuzzy.norms
%
% Generated by epydoc 3.0.1
% [Thu Jul 28 16:37:46 2011]
%

%%%%%%%%%%%%%%%%%%%%%%%%%%%%%%%%%%%%%%%%%%%%%%%%%%%%%%%%%%%%%%%%%%%%%%%%%%%
%%                          Module Description                           %%
%%%%%%%%%%%%%%%%%%%%%%%%%%%%%%%%%%%%%%%%%%%%%%%%%%%%%%%%%%%%%%%%%%%%%%%%%%%

    \index{peach \textit{(package)}!peach.fuzzy \textit{(package)}!peach.fuzzy.norms \textit{(module)}|(}
\section{Module peach.fuzzy.norms}

    \label{peach:fuzzy:norms}

This package implements operations of fuzzy logic.

Basic operations are \texttt{and (\&)}, \texttt{or (|)} and \texttt{not (\textasciitilde{})}. Those are
implemented as functions of, respectively, two, two and one values. The \texttt{and}
is the t-norm of the fuzzy logic, and it is a function that takes two values and
returns the result of the \texttt{and} operation. The \texttt{or} is a function that takes
two values and returns the result of the \texttt{or} operation. the \texttt{not} is a
function that takes one value and returns the result of the \texttt{not} operation.
To implement your own operations there is no need to subclass -{}- just create the
functions and use them where appropriate.

Also, implication and aglutination functions are defined here. Implication is
the result of the generalized modus ponens used in fuzzy inference systems.
Aglutination is the generalization from two different conclusions used in fuzzy
inference systems. Both are implemented as functions that take two values and
return the result of the operation. As above, to implement your own operations,
there is no need to subclass -{}- just create the functions and use them where
appropriate.

The functions here are provided as convenience.

%%%%%%%%%%%%%%%%%%%%%%%%%%%%%%%%%%%%%%%%%%%%%%%%%%%%%%%%%%%%%%%%%%%%%%%%%%%
%%                               Functions                               %%
%%%%%%%%%%%%%%%%%%%%%%%%%%%%%%%%%%%%%%%%%%%%%%%%%%%%%%%%%%%%%%%%%%%%%%%%%%%

  \subsection{Functions}

    \label{peach:fuzzy:norms:ZadehAnd}
    \index{peach \textit{(package)}!peach.fuzzy \textit{(package)}!peach.fuzzy.norms \textit{(module)}!peach.fuzzy.norms.ZadehAnd \textit{(function)}}

    \vspace{0.5ex}

\hspace{.8\funcindent}\begin{boxedminipage}{\funcwidth}

    \raggedright \textbf{ZadehAnd}(\textit{x}, \textit{y})

    \vspace{-1.5ex}

    \rule{\textwidth}{0.5\fboxrule}
\setlength{\parskip}{2ex}

And operation as defined by Lofti Zadeh.

And operation is the minimum of the two values.
\setlength{\parskip}{1ex}
      \textbf{Return Value}
    \vspace{-1ex}

      \begin{quote}

The result of the and operation.
      \end{quote}

    \end{boxedminipage}

    \label{peach:fuzzy:norms:ZadehOr}
    \index{peach \textit{(package)}!peach.fuzzy \textit{(package)}!peach.fuzzy.norms \textit{(module)}!peach.fuzzy.norms.ZadehOr \textit{(function)}}

    \vspace{0.5ex}

\hspace{.8\funcindent}\begin{boxedminipage}{\funcwidth}

    \raggedright \textbf{ZadehOr}(\textit{x}, \textit{y})

    \vspace{-1.5ex}

    \rule{\textwidth}{0.5\fboxrule}
\setlength{\parskip}{2ex}

Or operation as defined by Lofti Zadeh.

Or operation is the maximum of the two values.
\setlength{\parskip}{1ex}
      \textbf{Return Value}
    \vspace{-1ex}

      \begin{quote}

The result of the or operation.
      \end{quote}

    \end{boxedminipage}

    \label{peach:fuzzy:norms:ZadehNot}
    \index{peach \textit{(package)}!peach.fuzzy \textit{(package)}!peach.fuzzy.norms \textit{(module)}!peach.fuzzy.norms.ZadehNot \textit{(function)}}

    \vspace{0.5ex}

\hspace{.8\funcindent}\begin{boxedminipage}{\funcwidth}

    \raggedright \textbf{ZadehNot}(\textit{x})

    \vspace{-1.5ex}

    \rule{\textwidth}{0.5\fboxrule}
\setlength{\parskip}{2ex}

Not operation as defined by Lofti Zadeh.

Not operation is the complement to 1 of the given value, that is, \texttt{1 - x}.
\setlength{\parskip}{1ex}
      \textbf{Return Value}
    \vspace{-1ex}

      \begin{quote}

The result of the not operation.
      \end{quote}

    \end{boxedminipage}

    \label{peach:fuzzy:norms:ZadehImplication}
    \index{peach \textit{(package)}!peach.fuzzy \textit{(package)}!peach.fuzzy.norms \textit{(module)}!peach.fuzzy.norms.ZadehImplication \textit{(function)}}

    \vspace{0.5ex}

\hspace{.8\funcindent}\begin{boxedminipage}{\funcwidth}

    \raggedright \textbf{ZadehImplication}(\textit{x}, \textit{y})

    \vspace{-1.5ex}

    \rule{\textwidth}{0.5\fboxrule}
\setlength{\parskip}{2ex}

Implication operation as defined by Zadeh.
\setlength{\parskip}{1ex}
      \textbf{Return Value}
    \vspace{-1ex}

      \begin{quote}

The result of the implication.
      \end{quote}

    \end{boxedminipage}

    \label{peach:fuzzy:norms:DrasticProduct}
    \index{peach \textit{(package)}!peach.fuzzy \textit{(package)}!peach.fuzzy.norms \textit{(module)}!peach.fuzzy.norms.DrasticProduct \textit{(function)}}

    \vspace{0.5ex}

\hspace{.8\funcindent}\begin{boxedminipage}{\funcwidth}

    \raggedright \textbf{DrasticProduct}(\textit{x}, \textit{y})

    \vspace{-1.5ex}

    \rule{\textwidth}{0.5\fboxrule}
\setlength{\parskip}{2ex}

Drastic product that can be used as and operation
\setlength{\parskip}{1ex}
      \textbf{Return Value}
    \vspace{-1ex}

      \begin{quote}

The result of the and operation
      \end{quote}

    \end{boxedminipage}

    \label{peach:fuzzy:norms:DrasticSum}
    \index{peach \textit{(package)}!peach.fuzzy \textit{(package)}!peach.fuzzy.norms \textit{(module)}!peach.fuzzy.norms.DrasticSum \textit{(function)}}

    \vspace{0.5ex}

\hspace{.8\funcindent}\begin{boxedminipage}{\funcwidth}

    \raggedright \textbf{DrasticSum}(\textit{x}, \textit{y})

    \vspace{-1.5ex}

    \rule{\textwidth}{0.5\fboxrule}
\setlength{\parskip}{2ex}

Drastic sum that can be used as or operation
\setlength{\parskip}{1ex}
      \textbf{Return Value}
    \vspace{-1ex}

      \begin{quote}

The result of the or operation
      \end{quote}

    \end{boxedminipage}

    \label{peach:fuzzy:norms:EinsteinProduct}
    \index{peach \textit{(package)}!peach.fuzzy \textit{(package)}!peach.fuzzy.norms \textit{(module)}!peach.fuzzy.norms.EinsteinProduct \textit{(function)}}

    \vspace{0.5ex}

\hspace{.8\funcindent}\begin{boxedminipage}{\funcwidth}

    \raggedright \textbf{EinsteinProduct}(\textit{x}, \textit{y})

    \vspace{-1.5ex}

    \rule{\textwidth}{0.5\fboxrule}
\setlength{\parskip}{2ex}

Einstein product that can be used as and operation.
\setlength{\parskip}{1ex}
      \textbf{Return Value}
    \vspace{-1ex}

      \begin{quote}

The result of the and operation.
      \end{quote}

    \end{boxedminipage}

    \label{peach:fuzzy:norms:EinsteinSum}
    \index{peach \textit{(package)}!peach.fuzzy \textit{(package)}!peach.fuzzy.norms \textit{(module)}!peach.fuzzy.norms.EinsteinSum \textit{(function)}}

    \vspace{0.5ex}

\hspace{.8\funcindent}\begin{boxedminipage}{\funcwidth}

    \raggedright \textbf{EinsteinSum}(\textit{x}, \textit{y})

    \vspace{-1.5ex}

    \rule{\textwidth}{0.5\fboxrule}
\setlength{\parskip}{2ex}

Einstein sum that can be used as or operation.
\setlength{\parskip}{1ex}
      \textbf{Return Value}
    \vspace{-1ex}

      \begin{quote}

The result of the or operation.
      \end{quote}

    \end{boxedminipage}

    \label{peach:fuzzy:norms:MamdaniImplication}
    \index{peach \textit{(package)}!peach.fuzzy \textit{(package)}!peach.fuzzy.norms \textit{(module)}!peach.fuzzy.norms.MamdaniImplication \textit{(function)}}

    \vspace{0.5ex}

\hspace{.8\funcindent}\begin{boxedminipage}{\funcwidth}

    \raggedright \textbf{MamdaniImplication}(\textit{x}, \textit{y})

    \vspace{-1.5ex}

    \rule{\textwidth}{0.5\fboxrule}
\setlength{\parskip}{2ex}

Implication operation as defined by Mamdani.

Implication is the minimum of the two values.
\setlength{\parskip}{1ex}
      \textbf{Return Value}
    \vspace{-1ex}

      \begin{quote}

The result of the implication.
      \end{quote}

    \end{boxedminipage}

    \label{peach:fuzzy:norms:MamdaniAglutination}
    \index{peach \textit{(package)}!peach.fuzzy \textit{(package)}!peach.fuzzy.norms \textit{(module)}!peach.fuzzy.norms.MamdaniAglutination \textit{(function)}}

    \vspace{0.5ex}

\hspace{.8\funcindent}\begin{boxedminipage}{\funcwidth}

    \raggedright \textbf{MamdaniAglutination}(\textit{x}, \textit{y})

    \vspace{-1.5ex}

    \rule{\textwidth}{0.5\fboxrule}
\setlength{\parskip}{2ex}

Aglutination as defined by Mamdani.

Aglutination is the maximum of the two values.
\setlength{\parskip}{1ex}
      \textbf{Return Value}
    \vspace{-1ex}

      \begin{quote}

The result of the aglutination.
      \end{quote}

    \end{boxedminipage}

    \label{peach:fuzzy:norms:ProbabilisticAnd}
    \index{peach \textit{(package)}!peach.fuzzy \textit{(package)}!peach.fuzzy.norms \textit{(module)}!peach.fuzzy.norms.ProbabilisticAnd \textit{(function)}}

    \vspace{0.5ex}

\hspace{.8\funcindent}\begin{boxedminipage}{\funcwidth}

    \raggedright \textbf{ProbabilisticAnd}(\textit{x}, \textit{y})

    \vspace{-1.5ex}

    \rule{\textwidth}{0.5\fboxrule}
\setlength{\parskip}{2ex}

And operation as a probabilistic operation.

And operation is the product of the two values.
\setlength{\parskip}{1ex}
      \textbf{Return Value}
    \vspace{-1ex}

      \begin{quote}

The result of the and operation.
      \end{quote}

    \end{boxedminipage}

    \label{peach:fuzzy:norms:ProbabilisticOr}
    \index{peach \textit{(package)}!peach.fuzzy \textit{(package)}!peach.fuzzy.norms \textit{(module)}!peach.fuzzy.norms.ProbabilisticOr \textit{(function)}}

    \vspace{0.5ex}

\hspace{.8\funcindent}\begin{boxedminipage}{\funcwidth}

    \raggedright \textbf{ProbabilisticOr}(\textit{x}, \textit{y})

    \vspace{-1.5ex}

    \rule{\textwidth}{0.5\fboxrule}
\setlength{\parskip}{2ex}

Or operation as a probabilistic operation.

Or operation is given as the probability of the intersection of two events,
that is, x + y - xy.
\setlength{\parskip}{1ex}
      \textbf{Return Value}
    \vspace{-1ex}

      \begin{quote}

The result of the or operation.
      \end{quote}

    \end{boxedminipage}

    \label{peach:fuzzy:norms:ProbabilisticNot}
    \index{peach \textit{(package)}!peach.fuzzy \textit{(package)}!peach.fuzzy.norms \textit{(module)}!peach.fuzzy.norms.ProbabilisticNot \textit{(function)}}

    \vspace{0.5ex}

\hspace{.8\funcindent}\begin{boxedminipage}{\funcwidth}

    \raggedright \textbf{ProbabilisticNot}(\textit{x})

    \vspace{-1.5ex}

    \rule{\textwidth}{0.5\fboxrule}
\setlength{\parskip}{2ex}

Not operation as a probabilistic operation.

Not operation is the complement to 1 of the given value, that is, \texttt{1 - x}.
\setlength{\parskip}{1ex}
      \textbf{Return Value}
    \vspace{-1ex}

      \begin{quote}

The result of the not operation.
      \end{quote}

    \end{boxedminipage}

    \label{peach:fuzzy:norms:ProbabilisticImplication}
    \index{peach \textit{(package)}!peach.fuzzy \textit{(package)}!peach.fuzzy.norms \textit{(module)}!peach.fuzzy.norms.ProbabilisticImplication \textit{(function)}}

    \vspace{0.5ex}

\hspace{.8\funcindent}\begin{boxedminipage}{\funcwidth}

    \raggedright \textbf{ProbabilisticImplication}(\textit{x}, \textit{y})

    \vspace{-1.5ex}

    \rule{\textwidth}{0.5\fboxrule}
\setlength{\parskip}{2ex}

Implication as a probabilistic operation.

Implication is the product of the two values.
\setlength{\parskip}{1ex}
      \textbf{Return Value}
    \vspace{-1ex}

      \begin{quote}

The result of the and implication.
      \end{quote}

    \end{boxedminipage}

    \label{peach:fuzzy:norms:ProbabilisticAglutination}
    \index{peach \textit{(package)}!peach.fuzzy \textit{(package)}!peach.fuzzy.norms \textit{(module)}!peach.fuzzy.norms.ProbabilisticAglutination \textit{(function)}}

    \vspace{0.5ex}

\hspace{.8\funcindent}\begin{boxedminipage}{\funcwidth}

    \raggedright \textbf{ProbabilisticAglutination}(\textit{x}, \textit{y})

    \vspace{-1.5ex}

    \rule{\textwidth}{0.5\fboxrule}
\setlength{\parskip}{2ex}

Implication as a probabilistic operation.

Implication is given as the probability of the intersection of two events,
that is, x + y - xy.
\setlength{\parskip}{1ex}
      \textbf{Return Value}
    \vspace{-1ex}

      \begin{quote}

The result of the and algutination.
      \end{quote}

    \end{boxedminipage}

    \label{peach:fuzzy:norms:DienesRescherImplication}
    \index{peach \textit{(package)}!peach.fuzzy \textit{(package)}!peach.fuzzy.norms \textit{(module)}!peach.fuzzy.norms.DienesRescherImplication \textit{(function)}}

    \vspace{0.5ex}

\hspace{.8\funcindent}\begin{boxedminipage}{\funcwidth}

    \raggedright \textbf{DienesRescherImplication}(\textit{x}, \textit{y})

    \vspace{-1.5ex}

    \rule{\textwidth}{0.5\fboxrule}
\setlength{\parskip}{2ex}

Natural implication as in truth table, defined by Dienes-Rescher
\setlength{\parskip}{1ex}
      \textbf{Return Value}
    \vspace{-1ex}

      \begin{quote}

The result of the implication.
      \end{quote}

    \end{boxedminipage}

    \label{peach:fuzzy:norms:LukasiewiczImplication}
    \index{peach \textit{(package)}!peach.fuzzy \textit{(package)}!peach.fuzzy.norms \textit{(module)}!peach.fuzzy.norms.LukasiewiczImplication \textit{(function)}}

    \vspace{0.5ex}

\hspace{.8\funcindent}\begin{boxedminipage}{\funcwidth}

    \raggedright \textbf{LukasiewiczImplication}(\textit{x}, \textit{y})

    \vspace{-1.5ex}

    \rule{\textwidth}{0.5\fboxrule}
\setlength{\parskip}{2ex}

Implication of the Lukasiewicz three-valued logic.
\setlength{\parskip}{1ex}
      \textbf{Return Value}
    \vspace{-1ex}

      \begin{quote}

The result of the implication.
      \end{quote}

    \end{boxedminipage}

    \label{peach:fuzzy:norms:GodelImplication}
    \index{peach \textit{(package)}!peach.fuzzy \textit{(package)}!peach.fuzzy.norms \textit{(module)}!peach.fuzzy.norms.GodelImplication \textit{(function)}}

    \vspace{0.5ex}

\hspace{.8\funcindent}\begin{boxedminipage}{\funcwidth}

    \raggedright \textbf{GodelImplication}(\textit{x}, \textit{y})

    \vspace{-1.5ex}

    \rule{\textwidth}{0.5\fboxrule}
\setlength{\parskip}{2ex}

Implication as defined by Godel.
\setlength{\parskip}{1ex}
      \textbf{Return Value}
    \vspace{-1ex}

      \begin{quote}

The result of the implication.
      \end{quote}

    \end{boxedminipage}


%%%%%%%%%%%%%%%%%%%%%%%%%%%%%%%%%%%%%%%%%%%%%%%%%%%%%%%%%%%%%%%%%%%%%%%%%%%
%%                               Variables                               %%
%%%%%%%%%%%%%%%%%%%%%%%%%%%%%%%%%%%%%%%%%%%%%%%%%%%%%%%%%%%%%%%%%%%%%%%%%%%

  \subsection{Variables}

    \vspace{-1cm}
\hspace{\varindent}\begin{longtable}{|p{\varnamewidth}|p{\vardescrwidth}|l}
\cline{1-2}
\cline{1-2} \centering \textbf{Name} & \centering \textbf{Description}& \\
\cline{1-2}
\endhead\cline{1-2}\multicolumn{3}{r}{\small\textit{continued on next page}}\\\endfoot\cline{1-2}
\endlastfoot\raggedright \_\-\_\-d\-o\-c\-\_\-\_\- & \raggedright \textbf{Value:} 
{\tt \texttt{...}}&\\
\cline{1-2}
\raggedright Z\-A\-D\-E\-H\-\_\-N\-O\-R\-M\-S\- & \raggedright Tuple containing, in order, Zadeh and, or and not operations

\textbf{Value:} 
{\tt ZadehAnd, ZadehOr, ZadehNot}&\\
\cline{1-2}
\raggedright D\-R\-A\-S\-T\-I\-C\-\_\-N\-O\-R\-M\-S\- & \raggedright Tuple containing, in order, Drastic product (and), Drastic sum (or) and Zadeh
not operations

\textbf{Value:} 
{\tt DrasticProduct, DrasticSum, ZadehNot}&\\
\cline{1-2}
\raggedright E\-I\-N\-S\-T\-E\-I\-N\-\_\-N\-O\-R\-M\-S\- & \raggedright Tuple containing, in order, Einstein product (and), Einstein sum (or) and
Zadeh not operations

\textbf{Value:} 
{\tt EinsteinProduct, EinsteinSum, ZadehNot}&\\
\cline{1-2}
\raggedright M\-A\-M\-D\-A\-N\-I\-\_\-I\-N\-F\-E\-R\-E\-N\-C\-E\- & \raggedright Tuple containing, in order, Mamdani implication and algutination

\textbf{Value:} 
{\tt MamdaniImplication, MamdaniAglutination}&\\
\cline{1-2}
\raggedright P\-R\-O\-B\-\_\-N\-O\-R\-M\-S\- & \raggedright Tuple containing, in order, probabilistic and, or and not operations

\textbf{Value:} 
{\tt ProbabilisticAnd, ProbabilisticOr, ProbabilisticNot}&\\
\cline{1-2}
\raggedright P\-R\-O\-B\-\_\-I\-N\-F\-E\-R\-E\-N\-C\-E\- & \raggedright Tuple containing, in order, probabilistic implication and algutination

\textbf{Value:} 
{\tt ProbabilisticImplication, ProbabilisticAglutination}&\\
\cline{1-2}
\raggedright \_\-\_\-p\-a\-c\-k\-a\-g\-e\-\_\-\_\- & \raggedright \textbf{Value:} 
{\tt \texttt{'}\texttt{peach.fuzzy}\texttt{'}}&\\
\cline{1-2}
\end{longtable}

    \index{peach \textit{(package)}!peach.fuzzy \textit{(package)}!peach.fuzzy.norms \textit{(module)}|)}
