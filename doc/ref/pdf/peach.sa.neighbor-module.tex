%
% API Documentation for Peach - Computational Intelligence for Python
% Module peach.sa.neighbor
%
% Generated by epydoc 3.0.1
% [Fri Feb  4 17:21:22 2011]
%

%%%%%%%%%%%%%%%%%%%%%%%%%%%%%%%%%%%%%%%%%%%%%%%%%%%%%%%%%%%%%%%%%%%%%%%%%%%
%%                          Module Description                           %%
%%%%%%%%%%%%%%%%%%%%%%%%%%%%%%%%%%%%%%%%%%%%%%%%%%%%%%%%%%%%%%%%%%%%%%%%%%%

    \index{peach \textit{(package)}!peach.sa \textit{(package)}!peach.sa.neighbor \textit{(module)}|(}
\section{Module peach.sa.neighbor}

    \label{peach:sa:neighbor}

This module implements a general class to compute neighbors for continuous and
binary simulated annealing algorithms. The continuous neighbor functions return
an array with a neighbor of a given estimate; the binary neighbor functions
return a \texttt{bitarray} object.

%%%%%%%%%%%%%%%%%%%%%%%%%%%%%%%%%%%%%%%%%%%%%%%%%%%%%%%%%%%%%%%%%%%%%%%%%%%
%%                               Variables                               %%
%%%%%%%%%%%%%%%%%%%%%%%%%%%%%%%%%%%%%%%%%%%%%%%%%%%%%%%%%%%%%%%%%%%%%%%%%%%

  \subsection{Variables}

    \vspace{-1cm}
\hspace{\varindent}\begin{longtable}{|p{\varnamewidth}|p{\vardescrwidth}|l}
\cline{1-2}
\cline{1-2} \centering \textbf{Name} & \centering \textbf{Description}& \\
\cline{1-2}
\endhead\cline{1-2}\multicolumn{3}{r}{\small\textit{continued on next page}}\\\endfoot\cline{1-2}
\endlastfoot\raggedright \_\-\_\-d\-o\-c\-\_\-\_\- & \raggedright \textbf{Value:} 
{\tt \texttt{...}}&\\
\cline{1-2}
\raggedright \_\-\_\-p\-a\-c\-k\-a\-g\-e\-\_\-\_\- & \raggedright \textbf{Value:} 
{\tt \texttt{'}\texttt{peach.sa}\texttt{'}}&\\
\cline{1-2}
\end{longtable}


%%%%%%%%%%%%%%%%%%%%%%%%%%%%%%%%%%%%%%%%%%%%%%%%%%%%%%%%%%%%%%%%%%%%%%%%%%%
%%                           Class Description                           %%
%%%%%%%%%%%%%%%%%%%%%%%%%%%%%%%%%%%%%%%%%%%%%%%%%%%%%%%%%%%%%%%%%%%%%%%%%%%

    \index{peach \textit{(package)}!peach.sa \textit{(package)}!peach.sa.neighbor \textit{(module)}!peach.sa.neighbor.ContinuousNeighbor \textit{(class)}|(}
\subsection{Class ContinuousNeighbor}

    \label{peach:sa:neighbor:ContinuousNeighbor}
\begin{tabular}{cccccc}
% Line for object, linespec=[False]
\multicolumn{2}{r}{\settowidth{\BCL}{object}\multirow{2}{\BCL}{object}}
&&
  \\\cline{3-3}
  &&\multicolumn{1}{c|}{}
&&
  \\
&&\multicolumn{2}{l}{\textbf{peach.sa.neighbor.ContinuousNeighbor}}
\end{tabular}

\textbf{Known Subclasses:}
peach.sa.neighbor.GaussianNeighbor,
    peach.sa.neighbor.UniformNeighbor


Base class for continuous neighbor functions

This class should be derived to implement a function which computes the
neighbor of a given estimate. Every neighbor function should implement at
least two methods, defined below:
%
\begin{quote}
%
\begin{description}
\item[{\_\_init\_\_(self, %
\raisebox{1em}{\hypertarget{id2}{}}\hyperlink{id1}{\textbf{\color{red}*}}cnf, %
\raisebox{1em}{\hypertarget{id4}{}}\hyperlink{id3}{\textbf{\color{red}**}}kw)}] \leavevmode 
Initializes the object. There are no mandatory arguments, but any
parameters can be used here to configure the operator. For example, a
class can define a variance for randomly chose the neighbor -{}- this
should be defined here:
%
\begin{quote}{\ttfamily \raggedright \noindent
\_\_init\_\_(self,~variance=1.0)
}
\end{quote}

A default value should always be offered, if possible.

\item[{\_\_call\_\_(self, x):}] \leavevmode 
The \texttt{\_\_call\_\_} interface should be programmed to actually compute the
value of the neighbor. This method should receive an estimate in \texttt{x}
and use whatever parameters from the instantiation to compute the new
estimate. It should return the new estimate.

\end{description}

\end{quote}

Please, note that the SA implementations relies on this behaviour: it will
pass an estimate to your \texttt{\_\_call\_\_} method and expects to received the
result back.

This class can be used also to transform a simple function in a neighbor
function. In this case, the outside function must compute in an appropriate
way the new estimate.

%%%%%%%%%%%%%%%%%%%%%%%%%%%%%%%%%%%%%%%%%%%%%%%%%%%%%%%%%%%%%%%%%%%%%%%%%%%
%%                                Methods                                %%
%%%%%%%%%%%%%%%%%%%%%%%%%%%%%%%%%%%%%%%%%%%%%%%%%%%%%%%%%%%%%%%%%%%%%%%%%%%

  \subsubsection{Methods}

    \vspace{0.5ex}

\hspace{.8\funcindent}\begin{boxedminipage}{\funcwidth}

    \raggedright \textbf{\_\_init\_\_}(\textit{self}, \textit{f})

    \vspace{-1.5ex}

    \rule{\textwidth}{0.5\fboxrule}
\setlength{\parskip}{2ex}

Creates a neighbor function from a function.
\setlength{\parskip}{1ex}
      \textbf{Parameters}
      \vspace{-1ex}

      \begin{quote}
        \begin{Ventry}{x}

          \item[f]


The function to be transformed. This function must receive an array
of any size and shape as an estimate, and return an estimate of the
same size and shape as a result. A function that operates only over
a single number can be used -{}- in this case, the function operation
will propagate over all components of the estimate.
        \end{Ventry}

      \end{quote}

      Overrides: object.\_\_init\_\_

    \end{boxedminipage}

    \label{peach:sa:neighbor:ContinuousNeighbor:__call__}
    \index{peach \textit{(package)}!peach.sa \textit{(package)}!peach.sa.neighbor \textit{(module)}!peach.sa.neighbor.ContinuousNeighbor \textit{(class)}!peach.sa.neighbor.ContinuousNeighbor.\_\_call\_\_ \textit{(method)}}

    \vspace{0.5ex}

\hspace{.8\funcindent}\begin{boxedminipage}{\funcwidth}

    \raggedright \textbf{\_\_call\_\_}(\textit{self}, \textit{x})

    \vspace{-1.5ex}

    \rule{\textwidth}{0.5\fboxrule}
\setlength{\parskip}{2ex}

Computes the neighbor of the given estimate.
\setlength{\parskip}{1ex}
      \textbf{Parameters}
      \vspace{-1ex}

      \begin{quote}
        \begin{Ventry}{x}

          \item[x]


The estimate to which the neighbor must be computed.
        \end{Ventry}

      \end{quote}

    \end{boxedminipage}


\large{\textbf{\textit{Inherited from object}}}

\begin{quote}
\_\_delattr\_\_(), \_\_format\_\_(), \_\_getattribute\_\_(), \_\_hash\_\_(), \_\_new\_\_(), \_\_reduce\_\_(), \_\_reduce\_ex\_\_(), \_\_repr\_\_(), \_\_setattr\_\_(), \_\_sizeof\_\_(), \_\_str\_\_(), \_\_subclasshook\_\_()
\end{quote}

%%%%%%%%%%%%%%%%%%%%%%%%%%%%%%%%%%%%%%%%%%%%%%%%%%%%%%%%%%%%%%%%%%%%%%%%%%%
%%                              Properties                               %%
%%%%%%%%%%%%%%%%%%%%%%%%%%%%%%%%%%%%%%%%%%%%%%%%%%%%%%%%%%%%%%%%%%%%%%%%%%%

  \subsubsection{Properties}

    \vspace{-1cm}
\hspace{\varindent}\begin{longtable}{|p{\varnamewidth}|p{\vardescrwidth}|l}
\cline{1-2}
\cline{1-2} \centering \textbf{Name} & \centering \textbf{Description}& \\
\cline{1-2}
\endhead\cline{1-2}\multicolumn{3}{r}{\small\textit{continued on next page}}\\\endfoot\cline{1-2}
\endlastfoot\multicolumn{2}{|l|}{\textit{Inherited from object}}\\
\multicolumn{2}{|p{\varwidth}|}{\raggedright \_\_class\_\_}\\
\cline{1-2}
\end{longtable}

    \index{peach \textit{(package)}!peach.sa \textit{(package)}!peach.sa.neighbor \textit{(module)}!peach.sa.neighbor.ContinuousNeighbor \textit{(class)}|)}

%%%%%%%%%%%%%%%%%%%%%%%%%%%%%%%%%%%%%%%%%%%%%%%%%%%%%%%%%%%%%%%%%%%%%%%%%%%
%%                           Class Description                           %%
%%%%%%%%%%%%%%%%%%%%%%%%%%%%%%%%%%%%%%%%%%%%%%%%%%%%%%%%%%%%%%%%%%%%%%%%%%%

    \index{peach \textit{(package)}!peach.sa \textit{(package)}!peach.sa.neighbor \textit{(module)}!peach.sa.neighbor.GaussianNeighbor \textit{(class)}|(}
\subsection{Class GaussianNeighbor}

    \label{peach:sa:neighbor:GaussianNeighbor}
\begin{tabular}{cccccccc}
% Line for object, linespec=[False, False]
\multicolumn{2}{r}{\settowidth{\BCL}{object}\multirow{2}{\BCL}{object}}
&&
&&
  \\\cline{3-3}
  &&\multicolumn{1}{c|}{}
&&
&&
  \\
% Line for peach.sa.neighbor.ContinuousNeighbor, linespec=[False]
\multicolumn{4}{r}{\settowidth{\BCL}{peach.sa.neighbor.ContinuousNeighbor}\multirow{2}{\BCL}{peach.sa.neighbor.ContinuousNeighbor}}
&&
  \\\cline{5-5}
  &&&&\multicolumn{1}{c|}{}
&&
  \\
&&&&\multicolumn{2}{l}{\textbf{peach.sa.neighbor.GaussianNeighbor}}
\end{tabular}


A new estimate based on a gaussian distribution

This class creates a function that computes the neighbor of an estimate by
adding a gaussian distributed randomly choosen vector with the same shape
and size of the estimate.

%%%%%%%%%%%%%%%%%%%%%%%%%%%%%%%%%%%%%%%%%%%%%%%%%%%%%%%%%%%%%%%%%%%%%%%%%%%
%%                                Methods                                %%
%%%%%%%%%%%%%%%%%%%%%%%%%%%%%%%%%%%%%%%%%%%%%%%%%%%%%%%%%%%%%%%%%%%%%%%%%%%

  \subsubsection{Methods}

    \vspace{0.5ex}

\hspace{.8\funcindent}\begin{boxedminipage}{\funcwidth}

    \raggedright \textbf{\_\_init\_\_}(\textit{self}, \textit{variance}={\tt 0.05})

    \vspace{-1.5ex}

    \rule{\textwidth}{0.5\fboxrule}
\setlength{\parskip}{2ex}

Initializes the neighbor operator
\setlength{\parskip}{1ex}
      \textbf{Parameters}
      \vspace{-1ex}

      \begin{quote}
        \begin{Ventry}{xxxxxxxx}

          \item[variance]


This is the variance of the gaussian distribution used to randomize
the estimate. This can be given as a single value or as an array. In
the first case, the same value will be used for all the components
of the estimate; in the second case, \texttt{variance} should be an array
with the same number of components of the estimate, and each
component in this array is the variance of the corresponding
component in the estimate array.
        \end{Ventry}

      \end{quote}

      Overrides: object.\_\_init\_\_

    \end{boxedminipage}

    \vspace{0.5ex}

\hspace{.8\funcindent}\begin{boxedminipage}{\funcwidth}

    \raggedright \textbf{\_\_call\_\_}(\textit{self}, \textit{x})

    \vspace{-1.5ex}

    \rule{\textwidth}{0.5\fboxrule}
\setlength{\parskip}{2ex}

Computes the neighbor of the given estimate.
\setlength{\parskip}{1ex}
      \textbf{Parameters}
      \vspace{-1ex}

      \begin{quote}
        \begin{Ventry}{x}

          \item[x]


The estimate to which the neighbor must be computed.
        \end{Ventry}

      \end{quote}

      Overrides: peach.sa.neighbor.ContinuousNeighbor.\_\_call\_\_

    \end{boxedminipage}


\large{\textbf{\textit{Inherited from object}}}

\begin{quote}
\_\_delattr\_\_(), \_\_format\_\_(), \_\_getattribute\_\_(), \_\_hash\_\_(), \_\_new\_\_(), \_\_reduce\_\_(), \_\_reduce\_ex\_\_(), \_\_repr\_\_(), \_\_setattr\_\_(), \_\_sizeof\_\_(), \_\_str\_\_(), \_\_subclasshook\_\_()
\end{quote}

%%%%%%%%%%%%%%%%%%%%%%%%%%%%%%%%%%%%%%%%%%%%%%%%%%%%%%%%%%%%%%%%%%%%%%%%%%%
%%                              Properties                               %%
%%%%%%%%%%%%%%%%%%%%%%%%%%%%%%%%%%%%%%%%%%%%%%%%%%%%%%%%%%%%%%%%%%%%%%%%%%%

  \subsubsection{Properties}

    \vspace{-1cm}
\hspace{\varindent}\begin{longtable}{|p{\varnamewidth}|p{\vardescrwidth}|l}
\cline{1-2}
\cline{1-2} \centering \textbf{Name} & \centering \textbf{Description}& \\
\cline{1-2}
\endhead\cline{1-2}\multicolumn{3}{r}{\small\textit{continued on next page}}\\\endfoot\cline{1-2}
\endlastfoot\multicolumn{2}{|l|}{\textit{Inherited from object}}\\
\multicolumn{2}{|p{\varwidth}|}{\raggedright \_\_class\_\_}\\
\cline{1-2}
\end{longtable}


%%%%%%%%%%%%%%%%%%%%%%%%%%%%%%%%%%%%%%%%%%%%%%%%%%%%%%%%%%%%%%%%%%%%%%%%%%%
%%                          Instance Variables                           %%
%%%%%%%%%%%%%%%%%%%%%%%%%%%%%%%%%%%%%%%%%%%%%%%%%%%%%%%%%%%%%%%%%%%%%%%%%%%

  \subsubsection{Instance Variables}

    \vspace{-1cm}
\hspace{\varindent}\begin{longtable}{|p{\varnamewidth}|p{\vardescrwidth}|l}
\cline{1-2}
\cline{1-2} \centering \textbf{Name} & \centering \textbf{Description}& \\
\cline{1-2}
\endhead\cline{1-2}\multicolumn{3}{r}{\small\textit{continued on next page}}\\\endfoot\cline{1-2}
\endlastfoot\raggedright v\-a\-r\-i\-a\-n\-c\-e\- & Variance of the gaussian distribution.&\\
\cline{1-2}
\end{longtable}

    \index{peach \textit{(package)}!peach.sa \textit{(package)}!peach.sa.neighbor \textit{(module)}!peach.sa.neighbor.GaussianNeighbor \textit{(class)}|)}

%%%%%%%%%%%%%%%%%%%%%%%%%%%%%%%%%%%%%%%%%%%%%%%%%%%%%%%%%%%%%%%%%%%%%%%%%%%
%%                           Class Description                           %%
%%%%%%%%%%%%%%%%%%%%%%%%%%%%%%%%%%%%%%%%%%%%%%%%%%%%%%%%%%%%%%%%%%%%%%%%%%%

    \index{peach \textit{(package)}!peach.sa \textit{(package)}!peach.sa.neighbor \textit{(module)}!peach.sa.neighbor.UniformNeighbor \textit{(class)}|(}
\subsection{Class UniformNeighbor}

    \label{peach:sa:neighbor:UniformNeighbor}
\begin{tabular}{cccccccc}
% Line for object, linespec=[False, False]
\multicolumn{2}{r}{\settowidth{\BCL}{object}\multirow{2}{\BCL}{object}}
&&
&&
  \\\cline{3-3}
  &&\multicolumn{1}{c|}{}
&&
&&
  \\
% Line for peach.sa.neighbor.ContinuousNeighbor, linespec=[False]
\multicolumn{4}{r}{\settowidth{\BCL}{peach.sa.neighbor.ContinuousNeighbor}\multirow{2}{\BCL}{peach.sa.neighbor.ContinuousNeighbor}}
&&
  \\\cline{5-5}
  &&&&\multicolumn{1}{c|}{}
&&
  \\
&&&&\multicolumn{2}{l}{\textbf{peach.sa.neighbor.UniformNeighbor}}
\end{tabular}


A new estimate based on a uniform distribution

This class creates a function that computes the neighbor of an estimate by
adding a uniform distributed randomly choosen vector with the same shape
and size of the estimate.

%%%%%%%%%%%%%%%%%%%%%%%%%%%%%%%%%%%%%%%%%%%%%%%%%%%%%%%%%%%%%%%%%%%%%%%%%%%
%%                                Methods                                %%
%%%%%%%%%%%%%%%%%%%%%%%%%%%%%%%%%%%%%%%%%%%%%%%%%%%%%%%%%%%%%%%%%%%%%%%%%%%

  \subsubsection{Methods}

    \vspace{0.5ex}

\hspace{.8\funcindent}\begin{boxedminipage}{\funcwidth}

    \raggedright \textbf{\_\_init\_\_}(\textit{self}, \textit{xl}={\tt -1.0}, \textit{xh}={\tt 1.0})

    \vspace{-1.5ex}

    \rule{\textwidth}{0.5\fboxrule}
\setlength{\parskip}{2ex}

Initializes the neighbor operator
\setlength{\parskip}{1ex}
      \textbf{Parameters}
      \vspace{-1ex}

      \begin{quote}
        \begin{Ventry}{xx}

          \item[xl]


The lower limit of the distribution;
          \item[xh]


The upper limit of the distribution. Both values can be given as a
single value or as an array. In the first case, the same value will
be used for all the components of the estimate; in the second case,
they should be an array with the same number of components of the
estimate, and each component in this array is the variance of the
corresponding component in the estimate array.
        \end{Ventry}

      \end{quote}

      Overrides: object.\_\_init\_\_

    \end{boxedminipage}

    \vspace{0.5ex}

\hspace{.8\funcindent}\begin{boxedminipage}{\funcwidth}

    \raggedright \textbf{\_\_call\_\_}(\textit{self}, \textit{x})

    \vspace{-1.5ex}

    \rule{\textwidth}{0.5\fboxrule}
\setlength{\parskip}{2ex}

Computes the neighbor of the given estimate.
\setlength{\parskip}{1ex}
      \textbf{Parameters}
      \vspace{-1ex}

      \begin{quote}
        \begin{Ventry}{x}

          \item[x]


The estimate to which the neighbor must be computed.
        \end{Ventry}

      \end{quote}

      Overrides: peach.sa.neighbor.ContinuousNeighbor.\_\_call\_\_

    \end{boxedminipage}


\large{\textbf{\textit{Inherited from object}}}

\begin{quote}
\_\_delattr\_\_(), \_\_format\_\_(), \_\_getattribute\_\_(), \_\_hash\_\_(), \_\_new\_\_(), \_\_reduce\_\_(), \_\_reduce\_ex\_\_(), \_\_repr\_\_(), \_\_setattr\_\_(), \_\_sizeof\_\_(), \_\_str\_\_(), \_\_subclasshook\_\_()
\end{quote}

%%%%%%%%%%%%%%%%%%%%%%%%%%%%%%%%%%%%%%%%%%%%%%%%%%%%%%%%%%%%%%%%%%%%%%%%%%%
%%                              Properties                               %%
%%%%%%%%%%%%%%%%%%%%%%%%%%%%%%%%%%%%%%%%%%%%%%%%%%%%%%%%%%%%%%%%%%%%%%%%%%%

  \subsubsection{Properties}

    \vspace{-1cm}
\hspace{\varindent}\begin{longtable}{|p{\varnamewidth}|p{\vardescrwidth}|l}
\cline{1-2}
\cline{1-2} \centering \textbf{Name} & \centering \textbf{Description}& \\
\cline{1-2}
\endhead\cline{1-2}\multicolumn{3}{r}{\small\textit{continued on next page}}\\\endfoot\cline{1-2}
\endlastfoot\multicolumn{2}{|l|}{\textit{Inherited from object}}\\
\multicolumn{2}{|p{\varwidth}|}{\raggedright \_\_class\_\_}\\
\cline{1-2}
\end{longtable}


%%%%%%%%%%%%%%%%%%%%%%%%%%%%%%%%%%%%%%%%%%%%%%%%%%%%%%%%%%%%%%%%%%%%%%%%%%%
%%                          Instance Variables                           %%
%%%%%%%%%%%%%%%%%%%%%%%%%%%%%%%%%%%%%%%%%%%%%%%%%%%%%%%%%%%%%%%%%%%%%%%%%%%

  \subsubsection{Instance Variables}

    \vspace{-1cm}
\hspace{\varindent}\begin{longtable}{|p{\varnamewidth}|p{\vardescrwidth}|l}
\cline{1-2}
\cline{1-2} \centering \textbf{Name} & \centering \textbf{Description}& \\
\cline{1-2}
\endhead\cline{1-2}\multicolumn{3}{r}{\small\textit{continued on next page}}\\\endfoot\cline{1-2}
\endlastfoot\raggedright x\-l\- & Lower limit of the uniform distribution.&\\
\cline{1-2}
\raggedright x\-h\- & Upper limit of the uniform distribution.&\\
\cline{1-2}
\end{longtable}

    \index{peach \textit{(package)}!peach.sa \textit{(package)}!peach.sa.neighbor \textit{(module)}!peach.sa.neighbor.UniformNeighbor \textit{(class)}|)}

%%%%%%%%%%%%%%%%%%%%%%%%%%%%%%%%%%%%%%%%%%%%%%%%%%%%%%%%%%%%%%%%%%%%%%%%%%%
%%                           Class Description                           %%
%%%%%%%%%%%%%%%%%%%%%%%%%%%%%%%%%%%%%%%%%%%%%%%%%%%%%%%%%%%%%%%%%%%%%%%%%%%

    \index{peach \textit{(package)}!peach.sa \textit{(package)}!peach.sa.neighbor \textit{(module)}!peach.sa.neighbor.BinaryNeighbor \textit{(class)}|(}
\subsection{Class BinaryNeighbor}

    \label{peach:sa:neighbor:BinaryNeighbor}
\begin{tabular}{cccccc}
% Line for object, linespec=[False]
\multicolumn{2}{r}{\settowidth{\BCL}{object}\multirow{2}{\BCL}{object}}
&&
  \\\cline{3-3}
  &&\multicolumn{1}{c|}{}
&&
  \\
&&\multicolumn{2}{l}{\textbf{peach.sa.neighbor.BinaryNeighbor}}
\end{tabular}

\textbf{Known Subclasses:} peach.sa.neighbor.InvertBitsNeighbor


Base class for binary neighbor functions

This class should be derived to implement a function which computes the
neighbor of a given estimate. Every neighbor functions should implement at
least two methods, defined below:
%
\begin{quote}
%
\begin{description}
\item[{\_\_init\_\_(self, %
\raisebox{1em}{\hypertarget{id2}{}}\hyperlink{id1}{\textbf{\color{red}*}}cnf, %
\raisebox{1em}{\hypertarget{id4}{}}\hyperlink{id3}{\textbf{\color{red}**}}kw)}] \leavevmode 
Initializes the object. There are no mandatory arguments, but any
parameters can be used here to configure the operator. For example, a
class can define a bit change rate -{}- this should be defined here:
%
\begin{quote}{\ttfamily \raggedright \noindent
\_\_init\_\_(self,~rate=0.01)
}
\end{quote}

A default value should always be offered, if possible.

\item[{\_\_call\_\_(self, x):}] \leavevmode 
The \texttt{\_\_call\_\_} interface should be programmed to actually compute the
value of the neighbor. This method should receive an estimate in \texttt{x}
and use whatever parameters from the instantiation to compute the new
estimate. It should return the new estimate.

\end{description}

\end{quote}

Please, note that the SA implementations relies on this behaviour: it will
pass an estimate to your \texttt{\_\_call\_\_} method and expects to received the
result back. Notice, however, that the SA implementation does not expect
that the result is sane, ie, that it is in conformity with the
representation used in the algorithm. A sanity check is done inside the
binary SA class. Please, consult the documentation on \texttt{BinarySA} for
further details.

This class can be used also to transform a simple function in a neighbor
function. In this case, the outside function must compute in an appropriate
way the new estimate.

%%%%%%%%%%%%%%%%%%%%%%%%%%%%%%%%%%%%%%%%%%%%%%%%%%%%%%%%%%%%%%%%%%%%%%%%%%%
%%                                Methods                                %%
%%%%%%%%%%%%%%%%%%%%%%%%%%%%%%%%%%%%%%%%%%%%%%%%%%%%%%%%%%%%%%%%%%%%%%%%%%%

  \subsubsection{Methods}

    \vspace{0.5ex}

\hspace{.8\funcindent}\begin{boxedminipage}{\funcwidth}

    \raggedright \textbf{\_\_init\_\_}(\textit{self}, \textit{f})

    \vspace{-1.5ex}

    \rule{\textwidth}{0.5\fboxrule}
\setlength{\parskip}{2ex}

Creates a neighbor function from a function.
\setlength{\parskip}{1ex}
      \textbf{Parameters}
      \vspace{-1ex}

      \begin{quote}
        \begin{Ventry}{x}

          \item[f]


The function to be transformed. This function must receive a
bitarray of any length as an estimate, and return a new bitarray of
the same length as a result.
        \end{Ventry}

      \end{quote}

      Overrides: object.\_\_init\_\_

    \end{boxedminipage}

    \label{peach:sa:neighbor:BinaryNeighbor:__call__}
    \index{peach \textit{(package)}!peach.sa \textit{(package)}!peach.sa.neighbor \textit{(module)}!peach.sa.neighbor.BinaryNeighbor \textit{(class)}!peach.sa.neighbor.BinaryNeighbor.\_\_call\_\_ \textit{(method)}}

    \vspace{0.5ex}

\hspace{.8\funcindent}\begin{boxedminipage}{\funcwidth}

    \raggedright \textbf{\_\_call\_\_}(\textit{self}, \textit{x})

    \vspace{-1.5ex}

    \rule{\textwidth}{0.5\fboxrule}
\setlength{\parskip}{2ex}

Computes the neighbor of the given estimate.
\setlength{\parskip}{1ex}
      \textbf{Parameters}
      \vspace{-1ex}

      \begin{quote}
        \begin{Ventry}{x}

          \item[x]


The estimate to which the neighbor must be computed.
        \end{Ventry}

      \end{quote}

    \end{boxedminipage}


\large{\textbf{\textit{Inherited from object}}}

\begin{quote}
\_\_delattr\_\_(), \_\_format\_\_(), \_\_getattribute\_\_(), \_\_hash\_\_(), \_\_new\_\_(), \_\_reduce\_\_(), \_\_reduce\_ex\_\_(), \_\_repr\_\_(), \_\_setattr\_\_(), \_\_sizeof\_\_(), \_\_str\_\_(), \_\_subclasshook\_\_()
\end{quote}

%%%%%%%%%%%%%%%%%%%%%%%%%%%%%%%%%%%%%%%%%%%%%%%%%%%%%%%%%%%%%%%%%%%%%%%%%%%
%%                              Properties                               %%
%%%%%%%%%%%%%%%%%%%%%%%%%%%%%%%%%%%%%%%%%%%%%%%%%%%%%%%%%%%%%%%%%%%%%%%%%%%

  \subsubsection{Properties}

    \vspace{-1cm}
\hspace{\varindent}\begin{longtable}{|p{\varnamewidth}|p{\vardescrwidth}|l}
\cline{1-2}
\cline{1-2} \centering \textbf{Name} & \centering \textbf{Description}& \\
\cline{1-2}
\endhead\cline{1-2}\multicolumn{3}{r}{\small\textit{continued on next page}}\\\endfoot\cline{1-2}
\endlastfoot\multicolumn{2}{|l|}{\textit{Inherited from object}}\\
\multicolumn{2}{|p{\varwidth}|}{\raggedright \_\_class\_\_}\\
\cline{1-2}
\end{longtable}

    \index{peach \textit{(package)}!peach.sa \textit{(package)}!peach.sa.neighbor \textit{(module)}!peach.sa.neighbor.BinaryNeighbor \textit{(class)}|)}

%%%%%%%%%%%%%%%%%%%%%%%%%%%%%%%%%%%%%%%%%%%%%%%%%%%%%%%%%%%%%%%%%%%%%%%%%%%
%%                           Class Description                           %%
%%%%%%%%%%%%%%%%%%%%%%%%%%%%%%%%%%%%%%%%%%%%%%%%%%%%%%%%%%%%%%%%%%%%%%%%%%%

    \index{peach \textit{(package)}!peach.sa \textit{(package)}!peach.sa.neighbor \textit{(module)}!peach.sa.neighbor.InvertBitsNeighbor \textit{(class)}|(}
\subsection{Class InvertBitsNeighbor}

    \label{peach:sa:neighbor:InvertBitsNeighbor}
\begin{tabular}{cccccccc}
% Line for object, linespec=[False, False]
\multicolumn{2}{r}{\settowidth{\BCL}{object}\multirow{2}{\BCL}{object}}
&&
&&
  \\\cline{3-3}
  &&\multicolumn{1}{c|}{}
&&
&&
  \\
% Line for peach.sa.neighbor.BinaryNeighbor, linespec=[False]
\multicolumn{4}{r}{\settowidth{\BCL}{peach.sa.neighbor.BinaryNeighbor}\multirow{2}{\BCL}{peach.sa.neighbor.BinaryNeighbor}}
&&
  \\\cline{5-5}
  &&&&\multicolumn{1}{c|}{}
&&
  \\
&&&&\multicolumn{2}{l}{\textbf{peach.sa.neighbor.InvertBitsNeighbor}}
\end{tabular}


A simple neighborhood based on the change of a few bits.

This neighbor will be computed by randomly choosing a bit in the bitarray
representing the estimate and change a number of bits in the bitarray and
inverting their value.

%%%%%%%%%%%%%%%%%%%%%%%%%%%%%%%%%%%%%%%%%%%%%%%%%%%%%%%%%%%%%%%%%%%%%%%%%%%
%%                                Methods                                %%
%%%%%%%%%%%%%%%%%%%%%%%%%%%%%%%%%%%%%%%%%%%%%%%%%%%%%%%%%%%%%%%%%%%%%%%%%%%

  \subsubsection{Methods}

    \vspace{0.5ex}

\hspace{.8\funcindent}\begin{boxedminipage}{\funcwidth}

    \raggedright \textbf{\_\_init\_\_}(\textit{self}, \textit{nb}={\tt 2})

    \vspace{-1.5ex}

    \rule{\textwidth}{0.5\fboxrule}
\setlength{\parskip}{2ex}

Initializes the operator.
\setlength{\parskip}{1ex}
      \textbf{Parameters}
      \vspace{-1ex}

      \begin{quote}
        \begin{Ventry}{xx}

          \item[nb]


The number of bits to be randomly choosen to be inverted in the
calculation of the neighbor. Be very careful while choosing this
parameter. While very large optimizations can benefit from a big
value here, it is not recommended that more than one bit per
variable is inverted at each step -{}- otherwise, the neighbor might
fall very far from the present estimate, which can make the
algorithm not work accordingly. This defaults to 2, that is, at each
step, only one bit will be inverted at most.
        \end{Ventry}

      \end{quote}

      Overrides: object.\_\_init\_\_

    \end{boxedminipage}

    \vspace{0.5ex}

\hspace{.8\funcindent}\begin{boxedminipage}{\funcwidth}

    \raggedright \textbf{\_\_call\_\_}(\textit{self}, \textit{x})

    \vspace{-1.5ex}

    \rule{\textwidth}{0.5\fboxrule}
\setlength{\parskip}{2ex}

Computes the neighbor of the given estimate.
\setlength{\parskip}{1ex}
      \textbf{Parameters}
      \vspace{-1ex}

      \begin{quote}
        \begin{Ventry}{x}

          \item[x]


The estimate to which the neighbor must be computed.
        \end{Ventry}

      \end{quote}

      Overrides: peach.sa.neighbor.BinaryNeighbor.\_\_call\_\_

    \end{boxedminipage}


\large{\textbf{\textit{Inherited from object}}}

\begin{quote}
\_\_delattr\_\_(), \_\_format\_\_(), \_\_getattribute\_\_(), \_\_hash\_\_(), \_\_new\_\_(), \_\_reduce\_\_(), \_\_reduce\_ex\_\_(), \_\_repr\_\_(), \_\_setattr\_\_(), \_\_sizeof\_\_(), \_\_str\_\_(), \_\_subclasshook\_\_()
\end{quote}

%%%%%%%%%%%%%%%%%%%%%%%%%%%%%%%%%%%%%%%%%%%%%%%%%%%%%%%%%%%%%%%%%%%%%%%%%%%
%%                              Properties                               %%
%%%%%%%%%%%%%%%%%%%%%%%%%%%%%%%%%%%%%%%%%%%%%%%%%%%%%%%%%%%%%%%%%%%%%%%%%%%

  \subsubsection{Properties}

    \vspace{-1cm}
\hspace{\varindent}\begin{longtable}{|p{\varnamewidth}|p{\vardescrwidth}|l}
\cline{1-2}
\cline{1-2} \centering \textbf{Name} & \centering \textbf{Description}& \\
\cline{1-2}
\endhead\cline{1-2}\multicolumn{3}{r}{\small\textit{continued on next page}}\\\endfoot\cline{1-2}
\endlastfoot\multicolumn{2}{|l|}{\textit{Inherited from object}}\\
\multicolumn{2}{|p{\varwidth}|}{\raggedright \_\_class\_\_}\\
\cline{1-2}
\end{longtable}

    \index{peach \textit{(package)}!peach.sa \textit{(package)}!peach.sa.neighbor \textit{(module)}!peach.sa.neighbor.InvertBitsNeighbor \textit{(class)}|)}
    \index{peach \textit{(package)}!peach.sa \textit{(package)}!peach.sa.neighbor \textit{(module)}|)}
