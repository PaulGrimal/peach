%
% API Documentation for Peach - Computational Intelligence for Python
% Module peach.ga.selection
%
% Generated by epydoc 3.0beta1
% [Mon Dec 21 08:51:36 2009]
%

%%%%%%%%%%%%%%%%%%%%%%%%%%%%%%%%%%%%%%%%%%%%%%%%%%%%%%%%%%%%%%%%%%%%%%%%%%%
%%                          Module Description                           %%
%%%%%%%%%%%%%%%%%%%%%%%%%%%%%%%%%%%%%%%%%%%%%%%%%%%%%%%%%%%%%%%%%%%%%%%%%%%

    \index{peach \textit{(package)}!peach.ga \textit{(package)}!peach.ga.selection \textit{(module)}|(}
\section{Module peach.ga.selection}

    \label{peach:ga:selection}

Basic classes and definitions for selection operator.

The first step in a genetic algorithm is the selection of the fittest
individuals. The selection method typically uses the fitness of the population
to compute which individuals are closer to the best solution. However, instead
of deterministically deciding which individuals continue to the next generation,
they are randomly choosen, the chances of an individual being choosen given by
its fitness value. This sub-module implements selection methods.

%%%%%%%%%%%%%%%%%%%%%%%%%%%%%%%%%%%%%%%%%%%%%%%%%%%%%%%%%%%%%%%%%%%%%%%%%%%
%%                               Variables                               %%
%%%%%%%%%%%%%%%%%%%%%%%%%%%%%%%%%%%%%%%%%%%%%%%%%%%%%%%%%%%%%%%%%%%%%%%%%%%

  \subsection{Variables}

\begin{longtable}{|p{.30\textwidth}|p{.62\textwidth}|l}
\cline{1-2}
\cline{1-2} \centering \textbf{Name} & \centering \textbf{Description}& \\
\cline{1-2}
\endhead\cline{1-2}\multicolumn{3}{r}{\small\textit{continued on next page}}\\\endfoot\cline{1-2}
\endlastfoot\raggedright \_\-\_\-d\-o\-c\-\_\-\_\- & \raggedright \textbf{Value:} 
{\tt \texttt{...}}&\\
\cline{1-2}
\end{longtable}


%%%%%%%%%%%%%%%%%%%%%%%%%%%%%%%%%%%%%%%%%%%%%%%%%%%%%%%%%%%%%%%%%%%%%%%%%%%
%%                           Class Description                           %%
%%%%%%%%%%%%%%%%%%%%%%%%%%%%%%%%%%%%%%%%%%%%%%%%%%%%%%%%%%%%%%%%%%%%%%%%%%%

    \index{peach \textit{(package)}!peach.ga \textit{(package)}!peach.ga.selection \textit{(module)}!peach.ga.selection.Selection \textit{(class)}|(}
\subsection{Class Selection}

    \label{peach:ga:selection:Selection}
\begin{tabular}{cccccc}
% Line for object, linespec=[False]
\multicolumn{2}{r}{\settowidth{\BCL}{object}\multirow{2}{\BCL}{object}}
&&
  \\\cline{3-3}
  &&\multicolumn{1}{c|}{}
&&
  \\
&&\multicolumn{2}{l}{\textbf{peach.ga.selection.Selection}}
\end{tabular}

\textbf{Known Subclasses:}
peach.ga.selection.Baker,
    peach.ga.selection.BinaryTournament,
    peach.ga.selection.RouletteWheel


Base class for selection operators.

This class should be subclassed if you want to create your own selection
operator. The base class doesn't do much, it is only a prototype. As is done
with all the base classes within this library, use the \texttt{{\_}{\_}init{\_}{\_}} method
to configure your selection behaviour -{}- if needed -{}- and the \texttt{{\_}{\_}call{\_}{\_}}
method to operate over a population.

A class derived from this one should implement at least 2 methods, defined
below:
\begin{quote}
\begin{description}
%[visit_definition_list_item]
\item[{{\_}{\_}init{\_}{\_}(self, {\color{red}\bfseries{}*}cnf, {\color{red}\bfseries{}**}kw)}] %[visit_definition]

Initializes the object. There is no mandatory arguments, but any
parameters can be used here to configure the operator. A default value
should always be offered, if possible.

%[depart_definition]
%[depart_definition_list_item]
%[visit_definition_list_item]
\item[{{\_}{\_}call{\_}{\_}(self, population)}] %[visit_definition]

The \texttt{{\_}{\_}call{\_}{\_}} implementation should receive a population and operate
over it. Please, consult the \texttt{ga} module to see more information on
populations. It should return the processed population. No recomendation
on the internals of the method is made.

%[depart_definition]
%[depart_definition_list_item]
\end{description}
\end{quote}

Please, note that the GA implementations relies on this behaviour: it will
pass a population to your \texttt{{\_}{\_}call{\_}{\_}} method and expects to received the
result back.

%%%%%%%%%%%%%%%%%%%%%%%%%%%%%%%%%%%%%%%%%%%%%%%%%%%%%%%%%%%%%%%%%%%%%%%%%%%
%%                                Methods                                %%
%%%%%%%%%%%%%%%%%%%%%%%%%%%%%%%%%%%%%%%%%%%%%%%%%%%%%%%%%%%%%%%%%%%%%%%%%%%

  \subsubsection{Methods}

    \label{object:__delattr__}
    \index{object.\_\_delattr\_\_ \textit{(function)}}

    \vspace{0.5ex}

    \begin{boxedminipage}{\textwidth}

    \raggedright \textbf{\_\_delattr\_\_}(\textit{...})

    \vspace{-1.5ex}

    \rule{\textwidth}{0.5\fboxrule}

x.{\_}{\_}delattr{\_}{\_}('name') {\textless}=={\textgreater} del x.name
    \vspace{1ex}

    \end{boxedminipage}

    \label{object:__getattribute__}
    \index{object.\_\_getattribute\_\_ \textit{(function)}}

    \vspace{0.5ex}

    \begin{boxedminipage}{\textwidth}

    \raggedright \textbf{\_\_getattribute\_\_}(\textit{...})

    \vspace{-1.5ex}

    \rule{\textwidth}{0.5\fboxrule}

x.{\_}{\_}getattribute{\_}{\_}('name') {\textless}=={\textgreater} x.name
    \vspace{1ex}

    \end{boxedminipage}

    \label{object:__hash__}
    \index{object.\_\_hash\_\_ \textit{(function)}}

    \vspace{0.5ex}

    \begin{boxedminipage}{\textwidth}

    \raggedright \textbf{\_\_hash\_\_}(\textit{x})

    \vspace{-1.5ex}

    \rule{\textwidth}{0.5\fboxrule}

hash(x)
    \vspace{1ex}

    \end{boxedminipage}

    \label{object:__init__}
    \index{object.\_\_init\_\_ \textit{(function)}}

    \vspace{0.5ex}

    \begin{boxedminipage}{\textwidth}

    \raggedright \textbf{\_\_init\_\_}(\textit{...})

    \vspace{-1.5ex}

    \rule{\textwidth}{0.5\fboxrule}

x.{\_}{\_}init{\_}{\_}(...) initializes x; see x.{\_}{\_}class{\_}{\_}.{\_}{\_}doc{\_}{\_} for signature
    \vspace{1ex}

    \end{boxedminipage}

    \label{object:__new__}
    \index{object.\_\_new\_\_ \textit{(function)}}

    \vspace{0.5ex}

    \begin{boxedminipage}{\textwidth}

    \raggedright \textbf{\_\_new\_\_}(\textit{T}, \textit{S}, \textit{...})

      \textbf{Return Value}
      \begin{quote}
\begin{alltt}
a new object with type S, a subtype of T
\end{alltt}

      \end{quote}

    \vspace{1ex}

    \end{boxedminipage}

    \label{object:__reduce__}
    \index{object.\_\_reduce\_\_ \textit{(function)}}

    \vspace{0.5ex}

    \begin{boxedminipage}{\textwidth}

    \raggedright \textbf{\_\_reduce\_\_}(\textit{...})

    \vspace{-1.5ex}

    \rule{\textwidth}{0.5\fboxrule}

helper for pickle
    \vspace{1ex}

    \end{boxedminipage}

    \label{object:__reduce_ex__}
    \index{object.\_\_reduce\_ex\_\_ \textit{(function)}}

    \vspace{0.5ex}

    \begin{boxedminipage}{\textwidth}

    \raggedright \textbf{\_\_reduce\_ex\_\_}(\textit{...})

    \vspace{-1.5ex}

    \rule{\textwidth}{0.5\fboxrule}

helper for pickle
    \vspace{1ex}

    \end{boxedminipage}

    \label{object:__repr__}
    \index{object.\_\_repr\_\_ \textit{(function)}}

    \vspace{0.5ex}

    \begin{boxedminipage}{\textwidth}

    \raggedright \textbf{\_\_repr\_\_}(\textit{x})

    \vspace{-1.5ex}

    \rule{\textwidth}{0.5\fboxrule}

repr(x)
    \vspace{1ex}

    \end{boxedminipage}

    \label{object:__setattr__}
    \index{object.\_\_setattr\_\_ \textit{(function)}}

    \vspace{0.5ex}

    \begin{boxedminipage}{\textwidth}

    \raggedright \textbf{\_\_setattr\_\_}(\textit{...})

    \vspace{-1.5ex}

    \rule{\textwidth}{0.5\fboxrule}

x.{\_}{\_}setattr{\_}{\_}('name', value) {\textless}=={\textgreater} x.name = value
    \vspace{1ex}

    \end{boxedminipage}

    \label{object:__str__}
    \index{object.\_\_str\_\_ \textit{(function)}}

    \vspace{0.5ex}

    \begin{boxedminipage}{\textwidth}

    \raggedright \textbf{\_\_str\_\_}(\textit{x})

    \vspace{-1.5ex}

    \rule{\textwidth}{0.5\fboxrule}

str(x)
    \vspace{1ex}

    \end{boxedminipage}


%%%%%%%%%%%%%%%%%%%%%%%%%%%%%%%%%%%%%%%%%%%%%%%%%%%%%%%%%%%%%%%%%%%%%%%%%%%
%%                              Properties                               %%
%%%%%%%%%%%%%%%%%%%%%%%%%%%%%%%%%%%%%%%%%%%%%%%%%%%%%%%%%%%%%%%%%%%%%%%%%%%

  \subsubsection{Properties}

\begin{longtable}{|p{.30\textwidth}|p{.62\textwidth}|l}
\cline{1-2}
\cline{1-2} \centering \textbf{Name} & \centering \textbf{Description}& \\
\cline{1-2}
\endhead\cline{1-2}\multicolumn{3}{r}{\small\textit{continued on next page}}\\\endfoot\cline{1-2}
\endlastfoot\raggedright \_\-\_\-c\-l\-a\-s\-s\-\_\-\_\- & \raggedright \textbf{Value:} 
{\tt {\textless}attribute '\_\_class\_\_' of 'object' objects{\textgreater}}&\\
\cline{1-2}
\end{longtable}

    \index{peach \textit{(package)}!peach.ga \textit{(package)}!peach.ga.selection \textit{(module)}!peach.ga.selection.Selection \textit{(class)}|)}

%%%%%%%%%%%%%%%%%%%%%%%%%%%%%%%%%%%%%%%%%%%%%%%%%%%%%%%%%%%%%%%%%%%%%%%%%%%
%%                           Class Description                           %%
%%%%%%%%%%%%%%%%%%%%%%%%%%%%%%%%%%%%%%%%%%%%%%%%%%%%%%%%%%%%%%%%%%%%%%%%%%%

    \index{peach \textit{(package)}!peach.ga \textit{(package)}!peach.ga.selection \textit{(module)}!peach.ga.selection.RouletteWheel \textit{(class)}|(}
\subsection{Class RouletteWheel}

    \label{peach:ga:selection:RouletteWheel}
\begin{tabular}{cccccccc}
% Line for object, linespec=[False, False]
\multicolumn{2}{r}{\settowidth{\BCL}{object}\multirow{2}{\BCL}{object}}
&&
&&
  \\\cline{3-3}
  &&\multicolumn{1}{c|}{}
&&
&&
  \\
% Line for peach.ga.selection.Selection, linespec=[False]
\multicolumn{4}{r}{\settowidth{\BCL}{peach.ga.selection.Selection}\multirow{2}{\BCL}{peach.ga.selection.Selection}}
&&
  \\\cline{5-5}
  &&&&\multicolumn{1}{c|}{}
&&
  \\
&&&&\multicolumn{2}{l}{\textbf{peach.ga.selection.RouletteWheel}}
\end{tabular}


The Roulette Wheel selection method.

This method randomly chooses a new population with the same size of the
original population. An individual is choosen with a probability
proportional to its fitness value, independent of what fitness method was
used. This is usually abstracted as a roulette wheel in texts about the
subject. Please, note that the selection is done \emph{in loco}, that is,
although the new population is returned, it is not a new list -{}- it is the
same list as before, but with values changed.

%%%%%%%%%%%%%%%%%%%%%%%%%%%%%%%%%%%%%%%%%%%%%%%%%%%%%%%%%%%%%%%%%%%%%%%%%%%
%%                                Methods                                %%
%%%%%%%%%%%%%%%%%%%%%%%%%%%%%%%%%%%%%%%%%%%%%%%%%%%%%%%%%%%%%%%%%%%%%%%%%%%

  \subsubsection{Methods}

    \label{peach:ga:selection:RouletteWheel:__call__}
    \index{peach \textit{(package)}!peach.ga \textit{(package)}!peach.ga.selection \textit{(module)}!peach.ga.selection.RouletteWheel \textit{(class)}!peach.ga.selection.RouletteWheel.\_\_call\_\_ \textit{(method)}}

    \vspace{0.5ex}

    \begin{boxedminipage}{\textwidth}

    \raggedright \textbf{\_\_call\_\_}(\textit{self}, \textit{population})

    \vspace{-1.5ex}

    \rule{\textwidth}{0.5\fboxrule}

Selects the population.
    \vspace{1ex}

      \textbf{Parameters}
      \begin{quote}
        \begin{Ventry}{xxxxxxxxxx}

          \item[population]


The list of chromosomes that should be operated over. The given list
is modified, so be aware that the old generation will not be
available after stepping the GA.
        \end{Ventry}

      \end{quote}

    \vspace{1ex}

      \textbf{Return Value}
      \begin{quote}

The new population.
      \end{quote}

    \vspace{1ex}

    \end{boxedminipage}

    \label{object:__delattr__}
    \index{object.\_\_delattr\_\_ \textit{(function)}}

    \vspace{0.5ex}

    \begin{boxedminipage}{\textwidth}

    \raggedright \textbf{\_\_delattr\_\_}(\textit{...})

    \vspace{-1.5ex}

    \rule{\textwidth}{0.5\fboxrule}

x.{\_}{\_}delattr{\_}{\_}('name') {\textless}=={\textgreater} del x.name
    \vspace{1ex}

    \end{boxedminipage}

    \label{object:__getattribute__}
    \index{object.\_\_getattribute\_\_ \textit{(function)}}

    \vspace{0.5ex}

    \begin{boxedminipage}{\textwidth}

    \raggedright \textbf{\_\_getattribute\_\_}(\textit{...})

    \vspace{-1.5ex}

    \rule{\textwidth}{0.5\fboxrule}

x.{\_}{\_}getattribute{\_}{\_}('name') {\textless}=={\textgreater} x.name
    \vspace{1ex}

    \end{boxedminipage}

    \label{object:__hash__}
    \index{object.\_\_hash\_\_ \textit{(function)}}

    \vspace{0.5ex}

    \begin{boxedminipage}{\textwidth}

    \raggedright \textbf{\_\_hash\_\_}(\textit{x})

    \vspace{-1.5ex}

    \rule{\textwidth}{0.5\fboxrule}

hash(x)
    \vspace{1ex}

    \end{boxedminipage}

    \label{object:__init__}
    \index{object.\_\_init\_\_ \textit{(function)}}

    \vspace{0.5ex}

    \begin{boxedminipage}{\textwidth}

    \raggedright \textbf{\_\_init\_\_}(\textit{...})

    \vspace{-1.5ex}

    \rule{\textwidth}{0.5\fboxrule}

x.{\_}{\_}init{\_}{\_}(...) initializes x; see x.{\_}{\_}class{\_}{\_}.{\_}{\_}doc{\_}{\_} for signature
    \vspace{1ex}

    \end{boxedminipage}

    \label{object:__new__}
    \index{object.\_\_new\_\_ \textit{(function)}}

    \vspace{0.5ex}

    \begin{boxedminipage}{\textwidth}

    \raggedright \textbf{\_\_new\_\_}(\textit{T}, \textit{S}, \textit{...})

      \textbf{Return Value}
      \begin{quote}
\begin{alltt}
a new object with type S, a subtype of T
\end{alltt}

      \end{quote}

    \vspace{1ex}

    \end{boxedminipage}

    \label{object:__reduce__}
    \index{object.\_\_reduce\_\_ \textit{(function)}}

    \vspace{0.5ex}

    \begin{boxedminipage}{\textwidth}

    \raggedright \textbf{\_\_reduce\_\_}(\textit{...})

    \vspace{-1.5ex}

    \rule{\textwidth}{0.5\fboxrule}

helper for pickle
    \vspace{1ex}

    \end{boxedminipage}

    \label{object:__reduce_ex__}
    \index{object.\_\_reduce\_ex\_\_ \textit{(function)}}

    \vspace{0.5ex}

    \begin{boxedminipage}{\textwidth}

    \raggedright \textbf{\_\_reduce\_ex\_\_}(\textit{...})

    \vspace{-1.5ex}

    \rule{\textwidth}{0.5\fboxrule}

helper for pickle
    \vspace{1ex}

    \end{boxedminipage}

    \label{object:__repr__}
    \index{object.\_\_repr\_\_ \textit{(function)}}

    \vspace{0.5ex}

    \begin{boxedminipage}{\textwidth}

    \raggedright \textbf{\_\_repr\_\_}(\textit{x})

    \vspace{-1.5ex}

    \rule{\textwidth}{0.5\fboxrule}

repr(x)
    \vspace{1ex}

    \end{boxedminipage}

    \label{object:__setattr__}
    \index{object.\_\_setattr\_\_ \textit{(function)}}

    \vspace{0.5ex}

    \begin{boxedminipage}{\textwidth}

    \raggedright \textbf{\_\_setattr\_\_}(\textit{...})

    \vspace{-1.5ex}

    \rule{\textwidth}{0.5\fboxrule}

x.{\_}{\_}setattr{\_}{\_}('name', value) {\textless}=={\textgreater} x.name = value
    \vspace{1ex}

    \end{boxedminipage}

    \label{object:__str__}
    \index{object.\_\_str\_\_ \textit{(function)}}

    \vspace{0.5ex}

    \begin{boxedminipage}{\textwidth}

    \raggedright \textbf{\_\_str\_\_}(\textit{x})

    \vspace{-1.5ex}

    \rule{\textwidth}{0.5\fboxrule}

str(x)
    \vspace{1ex}

    \end{boxedminipage}


%%%%%%%%%%%%%%%%%%%%%%%%%%%%%%%%%%%%%%%%%%%%%%%%%%%%%%%%%%%%%%%%%%%%%%%%%%%
%%                              Properties                               %%
%%%%%%%%%%%%%%%%%%%%%%%%%%%%%%%%%%%%%%%%%%%%%%%%%%%%%%%%%%%%%%%%%%%%%%%%%%%

  \subsubsection{Properties}

\begin{longtable}{|p{.30\textwidth}|p{.62\textwidth}|l}
\cline{1-2}
\cline{1-2} \centering \textbf{Name} & \centering \textbf{Description}& \\
\cline{1-2}
\endhead\cline{1-2}\multicolumn{3}{r}{\small\textit{continued on next page}}\\\endfoot\cline{1-2}
\endlastfoot\raggedright \_\-\_\-c\-l\-a\-s\-s\-\_\-\_\- & \raggedright \textbf{Value:} 
{\tt {\textless}attribute '\_\_class\_\_' of 'object' objects{\textgreater}}&\\
\cline{1-2}
\end{longtable}

    \index{peach \textit{(package)}!peach.ga \textit{(package)}!peach.ga.selection \textit{(module)}!peach.ga.selection.RouletteWheel \textit{(class)}|)}

%%%%%%%%%%%%%%%%%%%%%%%%%%%%%%%%%%%%%%%%%%%%%%%%%%%%%%%%%%%%%%%%%%%%%%%%%%%
%%                           Class Description                           %%
%%%%%%%%%%%%%%%%%%%%%%%%%%%%%%%%%%%%%%%%%%%%%%%%%%%%%%%%%%%%%%%%%%%%%%%%%%%

    \index{peach \textit{(package)}!peach.ga \textit{(package)}!peach.ga.selection \textit{(module)}!peach.ga.selection.BinaryTournament \textit{(class)}|(}
\subsection{Class BinaryTournament}

    \label{peach:ga:selection:BinaryTournament}
\begin{tabular}{cccccccc}
% Line for object, linespec=[False, False]
\multicolumn{2}{r}{\settowidth{\BCL}{object}\multirow{2}{\BCL}{object}}
&&
&&
  \\\cline{3-3}
  &&\multicolumn{1}{c|}{}
&&
&&
  \\
% Line for peach.ga.selection.Selection, linespec=[False]
\multicolumn{4}{r}{\settowidth{\BCL}{peach.ga.selection.Selection}\multirow{2}{\BCL}{peach.ga.selection.Selection}}
&&
  \\\cline{5-5}
  &&&&\multicolumn{1}{c|}{}
&&
  \\
&&&&\multicolumn{2}{l}{\textbf{peach.ga.selection.BinaryTournament}}
\end{tabular}


The Binary Tournament selection method.

This method randomly chooses a new population with the same size of the
original population. Two individuals are choosen at random and they
``battle'', the fittest surviving for the next generation. Please, note that
the selection is done \emph{in loco}, that is, although the new population is
returned, it is not a new list -{}- it is the same list as before, but with
values changed.

%%%%%%%%%%%%%%%%%%%%%%%%%%%%%%%%%%%%%%%%%%%%%%%%%%%%%%%%%%%%%%%%%%%%%%%%%%%
%%                                Methods                                %%
%%%%%%%%%%%%%%%%%%%%%%%%%%%%%%%%%%%%%%%%%%%%%%%%%%%%%%%%%%%%%%%%%%%%%%%%%%%

  \subsubsection{Methods}

    \label{peach:ga:selection:BinaryTournament:__call__}
    \index{peach \textit{(package)}!peach.ga \textit{(package)}!peach.ga.selection \textit{(module)}!peach.ga.selection.BinaryTournament \textit{(class)}!peach.ga.selection.BinaryTournament.\_\_call\_\_ \textit{(method)}}

    \vspace{0.5ex}

    \begin{boxedminipage}{\textwidth}

    \raggedright \textbf{\_\_call\_\_}(\textit{self}, \textit{population})

    \vspace{-1.5ex}

    \rule{\textwidth}{0.5\fboxrule}

Selects the population.
    \vspace{1ex}

      \textbf{Parameters}
      \begin{quote}
        \begin{Ventry}{xxxxxxxxxx}

          \item[population]


The list of chromosomes that should be operated over. The given list
is modified, so be aware that the old generation will not be
available after stepping the GA.
        \end{Ventry}

      \end{quote}

    \vspace{1ex}

      \textbf{Return Value}
      \begin{quote}

The new population.
      \end{quote}

    \vspace{1ex}

    \end{boxedminipage}

    \label{object:__delattr__}
    \index{object.\_\_delattr\_\_ \textit{(function)}}

    \vspace{0.5ex}

    \begin{boxedminipage}{\textwidth}

    \raggedright \textbf{\_\_delattr\_\_}(\textit{...})

    \vspace{-1.5ex}

    \rule{\textwidth}{0.5\fboxrule}

x.{\_}{\_}delattr{\_}{\_}('name') {\textless}=={\textgreater} del x.name
    \vspace{1ex}

    \end{boxedminipage}

    \label{object:__getattribute__}
    \index{object.\_\_getattribute\_\_ \textit{(function)}}

    \vspace{0.5ex}

    \begin{boxedminipage}{\textwidth}

    \raggedright \textbf{\_\_getattribute\_\_}(\textit{...})

    \vspace{-1.5ex}

    \rule{\textwidth}{0.5\fboxrule}

x.{\_}{\_}getattribute{\_}{\_}('name') {\textless}=={\textgreater} x.name
    \vspace{1ex}

    \end{boxedminipage}

    \label{object:__hash__}
    \index{object.\_\_hash\_\_ \textit{(function)}}

    \vspace{0.5ex}

    \begin{boxedminipage}{\textwidth}

    \raggedright \textbf{\_\_hash\_\_}(\textit{x})

    \vspace{-1.5ex}

    \rule{\textwidth}{0.5\fboxrule}

hash(x)
    \vspace{1ex}

    \end{boxedminipage}

    \label{object:__init__}
    \index{object.\_\_init\_\_ \textit{(function)}}

    \vspace{0.5ex}

    \begin{boxedminipage}{\textwidth}

    \raggedright \textbf{\_\_init\_\_}(\textit{...})

    \vspace{-1.5ex}

    \rule{\textwidth}{0.5\fboxrule}

x.{\_}{\_}init{\_}{\_}(...) initializes x; see x.{\_}{\_}class{\_}{\_}.{\_}{\_}doc{\_}{\_} for signature
    \vspace{1ex}

    \end{boxedminipage}

    \label{object:__new__}
    \index{object.\_\_new\_\_ \textit{(function)}}

    \vspace{0.5ex}

    \begin{boxedminipage}{\textwidth}

    \raggedright \textbf{\_\_new\_\_}(\textit{T}, \textit{S}, \textit{...})

      \textbf{Return Value}
      \begin{quote}
\begin{alltt}
a new object with type S, a subtype of T
\end{alltt}

      \end{quote}

    \vspace{1ex}

    \end{boxedminipage}

    \label{object:__reduce__}
    \index{object.\_\_reduce\_\_ \textit{(function)}}

    \vspace{0.5ex}

    \begin{boxedminipage}{\textwidth}

    \raggedright \textbf{\_\_reduce\_\_}(\textit{...})

    \vspace{-1.5ex}

    \rule{\textwidth}{0.5\fboxrule}

helper for pickle
    \vspace{1ex}

    \end{boxedminipage}

    \label{object:__reduce_ex__}
    \index{object.\_\_reduce\_ex\_\_ \textit{(function)}}

    \vspace{0.5ex}

    \begin{boxedminipage}{\textwidth}

    \raggedright \textbf{\_\_reduce\_ex\_\_}(\textit{...})

    \vspace{-1.5ex}

    \rule{\textwidth}{0.5\fboxrule}

helper for pickle
    \vspace{1ex}

    \end{boxedminipage}

    \label{object:__repr__}
    \index{object.\_\_repr\_\_ \textit{(function)}}

    \vspace{0.5ex}

    \begin{boxedminipage}{\textwidth}

    \raggedright \textbf{\_\_repr\_\_}(\textit{x})

    \vspace{-1.5ex}

    \rule{\textwidth}{0.5\fboxrule}

repr(x)
    \vspace{1ex}

    \end{boxedminipage}

    \label{object:__setattr__}
    \index{object.\_\_setattr\_\_ \textit{(function)}}

    \vspace{0.5ex}

    \begin{boxedminipage}{\textwidth}

    \raggedright \textbf{\_\_setattr\_\_}(\textit{...})

    \vspace{-1.5ex}

    \rule{\textwidth}{0.5\fboxrule}

x.{\_}{\_}setattr{\_}{\_}('name', value) {\textless}=={\textgreater} x.name = value
    \vspace{1ex}

    \end{boxedminipage}

    \label{object:__str__}
    \index{object.\_\_str\_\_ \textit{(function)}}

    \vspace{0.5ex}

    \begin{boxedminipage}{\textwidth}

    \raggedright \textbf{\_\_str\_\_}(\textit{x})

    \vspace{-1.5ex}

    \rule{\textwidth}{0.5\fboxrule}

str(x)
    \vspace{1ex}

    \end{boxedminipage}


%%%%%%%%%%%%%%%%%%%%%%%%%%%%%%%%%%%%%%%%%%%%%%%%%%%%%%%%%%%%%%%%%%%%%%%%%%%
%%                              Properties                               %%
%%%%%%%%%%%%%%%%%%%%%%%%%%%%%%%%%%%%%%%%%%%%%%%%%%%%%%%%%%%%%%%%%%%%%%%%%%%

  \subsubsection{Properties}

\begin{longtable}{|p{.30\textwidth}|p{.62\textwidth}|l}
\cline{1-2}
\cline{1-2} \centering \textbf{Name} & \centering \textbf{Description}& \\
\cline{1-2}
\endhead\cline{1-2}\multicolumn{3}{r}{\small\textit{continued on next page}}\\\endfoot\cline{1-2}
\endlastfoot\raggedright \_\-\_\-c\-l\-a\-s\-s\-\_\-\_\- & \raggedright \textbf{Value:} 
{\tt {\textless}attribute '\_\_class\_\_' of 'object' objects{\textgreater}}&\\
\cline{1-2}
\end{longtable}

    \index{peach \textit{(package)}!peach.ga \textit{(package)}!peach.ga.selection \textit{(module)}!peach.ga.selection.BinaryTournament \textit{(class)}|)}

%%%%%%%%%%%%%%%%%%%%%%%%%%%%%%%%%%%%%%%%%%%%%%%%%%%%%%%%%%%%%%%%%%%%%%%%%%%
%%                           Class Description                           %%
%%%%%%%%%%%%%%%%%%%%%%%%%%%%%%%%%%%%%%%%%%%%%%%%%%%%%%%%%%%%%%%%%%%%%%%%%%%

    \index{peach \textit{(package)}!peach.ga \textit{(package)}!peach.ga.selection \textit{(module)}!peach.ga.selection.Baker \textit{(class)}|(}
\subsection{Class Baker}

    \label{peach:ga:selection:Baker}
\begin{tabular}{cccccccc}
% Line for object, linespec=[False, False]
\multicolumn{2}{r}{\settowidth{\BCL}{object}\multirow{2}{\BCL}{object}}
&&
&&
  \\\cline{3-3}
  &&\multicolumn{1}{c|}{}
&&
&&
  \\
% Line for peach.ga.selection.Selection, linespec=[False]
\multicolumn{4}{r}{\settowidth{\BCL}{peach.ga.selection.Selection}\multirow{2}{\BCL}{peach.ga.selection.Selection}}
&&
  \\\cline{5-5}
  &&&&\multicolumn{1}{c|}{}
&&
  \\
&&&&\multicolumn{2}{l}{\textbf{peach.ga.selection.Baker}}
\end{tabular}


The Baker selection method.

This method is very similar to the Roulette Wheel, but instead or randomly
choosing every new member on the next generation, only the first probability
is randomized. The others are determined as equally spaced numbers from 0 to
1, from this number. Please, note that the selection is done \emph{in loco}, that
is, although the new population is returned, it is not a new list -{}- it is
the same list as before, but with values changed.

%%%%%%%%%%%%%%%%%%%%%%%%%%%%%%%%%%%%%%%%%%%%%%%%%%%%%%%%%%%%%%%%%%%%%%%%%%%
%%                                Methods                                %%
%%%%%%%%%%%%%%%%%%%%%%%%%%%%%%%%%%%%%%%%%%%%%%%%%%%%%%%%%%%%%%%%%%%%%%%%%%%

  \subsubsection{Methods}

    \label{peach:ga:selection:Baker:__call__}
    \index{peach \textit{(package)}!peach.ga \textit{(package)}!peach.ga.selection \textit{(module)}!peach.ga.selection.Baker \textit{(class)}!peach.ga.selection.Baker.\_\_call\_\_ \textit{(method)}}

    \vspace{0.5ex}

    \begin{boxedminipage}{\textwidth}

    \raggedright \textbf{\_\_call\_\_}(\textit{self}, \textit{population})

    \vspace{-1.5ex}

    \rule{\textwidth}{0.5\fboxrule}

Selects the population.
    \vspace{1ex}

      \textbf{Parameters}
      \begin{quote}
        \begin{Ventry}{xxxxxxxxxx}

          \item[population]


The list of chromosomes that should be operated over. The given list
is modified, so be aware that the old generation will not be
available after stepping the GA.
        \end{Ventry}

      \end{quote}

    \vspace{1ex}

      \textbf{Return Value}
      \begin{quote}

The new population.
      \end{quote}

    \vspace{1ex}

    \end{boxedminipage}

    \label{object:__delattr__}
    \index{object.\_\_delattr\_\_ \textit{(function)}}

    \vspace{0.5ex}

    \begin{boxedminipage}{\textwidth}

    \raggedright \textbf{\_\_delattr\_\_}(\textit{...})

    \vspace{-1.5ex}

    \rule{\textwidth}{0.5\fboxrule}

x.{\_}{\_}delattr{\_}{\_}('name') {\textless}=={\textgreater} del x.name
    \vspace{1ex}

    \end{boxedminipage}

    \label{object:__getattribute__}
    \index{object.\_\_getattribute\_\_ \textit{(function)}}

    \vspace{0.5ex}

    \begin{boxedminipage}{\textwidth}

    \raggedright \textbf{\_\_getattribute\_\_}(\textit{...})

    \vspace{-1.5ex}

    \rule{\textwidth}{0.5\fboxrule}

x.{\_}{\_}getattribute{\_}{\_}('name') {\textless}=={\textgreater} x.name
    \vspace{1ex}

    \end{boxedminipage}

    \label{object:__hash__}
    \index{object.\_\_hash\_\_ \textit{(function)}}

    \vspace{0.5ex}

    \begin{boxedminipage}{\textwidth}

    \raggedright \textbf{\_\_hash\_\_}(\textit{x})

    \vspace{-1.5ex}

    \rule{\textwidth}{0.5\fboxrule}

hash(x)
    \vspace{1ex}

    \end{boxedminipage}

    \label{object:__init__}
    \index{object.\_\_init\_\_ \textit{(function)}}

    \vspace{0.5ex}

    \begin{boxedminipage}{\textwidth}

    \raggedright \textbf{\_\_init\_\_}(\textit{...})

    \vspace{-1.5ex}

    \rule{\textwidth}{0.5\fboxrule}

x.{\_}{\_}init{\_}{\_}(...) initializes x; see x.{\_}{\_}class{\_}{\_}.{\_}{\_}doc{\_}{\_} for signature
    \vspace{1ex}

    \end{boxedminipage}

    \label{object:__new__}
    \index{object.\_\_new\_\_ \textit{(function)}}

    \vspace{0.5ex}

    \begin{boxedminipage}{\textwidth}

    \raggedright \textbf{\_\_new\_\_}(\textit{T}, \textit{S}, \textit{...})

      \textbf{Return Value}
      \begin{quote}
\begin{alltt}
a new object with type S, a subtype of T
\end{alltt}

      \end{quote}

    \vspace{1ex}

    \end{boxedminipage}

    \label{object:__reduce__}
    \index{object.\_\_reduce\_\_ \textit{(function)}}

    \vspace{0.5ex}

    \begin{boxedminipage}{\textwidth}

    \raggedright \textbf{\_\_reduce\_\_}(\textit{...})

    \vspace{-1.5ex}

    \rule{\textwidth}{0.5\fboxrule}

helper for pickle
    \vspace{1ex}

    \end{boxedminipage}

    \label{object:__reduce_ex__}
    \index{object.\_\_reduce\_ex\_\_ \textit{(function)}}

    \vspace{0.5ex}

    \begin{boxedminipage}{\textwidth}

    \raggedright \textbf{\_\_reduce\_ex\_\_}(\textit{...})

    \vspace{-1.5ex}

    \rule{\textwidth}{0.5\fboxrule}

helper for pickle
    \vspace{1ex}

    \end{boxedminipage}

    \label{object:__repr__}
    \index{object.\_\_repr\_\_ \textit{(function)}}

    \vspace{0.5ex}

    \begin{boxedminipage}{\textwidth}

    \raggedright \textbf{\_\_repr\_\_}(\textit{x})

    \vspace{-1.5ex}

    \rule{\textwidth}{0.5\fboxrule}

repr(x)
    \vspace{1ex}

    \end{boxedminipage}

    \label{object:__setattr__}
    \index{object.\_\_setattr\_\_ \textit{(function)}}

    \vspace{0.5ex}

    \begin{boxedminipage}{\textwidth}

    \raggedright \textbf{\_\_setattr\_\_}(\textit{...})

    \vspace{-1.5ex}

    \rule{\textwidth}{0.5\fboxrule}

x.{\_}{\_}setattr{\_}{\_}('name', value) {\textless}=={\textgreater} x.name = value
    \vspace{1ex}

    \end{boxedminipage}

    \label{object:__str__}
    \index{object.\_\_str\_\_ \textit{(function)}}

    \vspace{0.5ex}

    \begin{boxedminipage}{\textwidth}

    \raggedright \textbf{\_\_str\_\_}(\textit{x})

    \vspace{-1.5ex}

    \rule{\textwidth}{0.5\fboxrule}

str(x)
    \vspace{1ex}

    \end{boxedminipage}


%%%%%%%%%%%%%%%%%%%%%%%%%%%%%%%%%%%%%%%%%%%%%%%%%%%%%%%%%%%%%%%%%%%%%%%%%%%
%%                              Properties                               %%
%%%%%%%%%%%%%%%%%%%%%%%%%%%%%%%%%%%%%%%%%%%%%%%%%%%%%%%%%%%%%%%%%%%%%%%%%%%

  \subsubsection{Properties}

\begin{longtable}{|p{.30\textwidth}|p{.62\textwidth}|l}
\cline{1-2}
\cline{1-2} \centering \textbf{Name} & \centering \textbf{Description}& \\
\cline{1-2}
\endhead\cline{1-2}\multicolumn{3}{r}{\small\textit{continued on next page}}\\\endfoot\cline{1-2}
\endlastfoot\raggedright \_\-\_\-c\-l\-a\-s\-s\-\_\-\_\- & \raggedright \textbf{Value:} 
{\tt {\textless}attribute '\_\_class\_\_' of 'object' objects{\textgreater}}&\\
\cline{1-2}
\end{longtable}

    \index{peach \textit{(package)}!peach.ga \textit{(package)}!peach.ga.selection \textit{(module)}!peach.ga.selection.Baker \textit{(class)}|)}
    \index{peach \textit{(package)}!peach.ga \textit{(package)}!peach.ga.selection \textit{(module)}|)}
