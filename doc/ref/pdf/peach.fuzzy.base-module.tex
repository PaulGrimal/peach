%
% API Documentation for Peach - Computational Intelligence for Python
% Module peach.fuzzy.base
%
% Generated by epydoc 3.0.1
% [Mon Jan 24 15:39:50 2011]
%

%%%%%%%%%%%%%%%%%%%%%%%%%%%%%%%%%%%%%%%%%%%%%%%%%%%%%%%%%%%%%%%%%%%%%%%%%%%
%%                          Module Description                           %%
%%%%%%%%%%%%%%%%%%%%%%%%%%%%%%%%%%%%%%%%%%%%%%%%%%%%%%%%%%%%%%%%%%%%%%%%%%%

    \index{peach \textit{(package)}!peach.fuzzy \textit{(package)}!peach.fuzzy.base \textit{(module)}|(}
\section{Module peach.fuzzy.base}

    \label{peach:fuzzy:base}

This package implements basic definitions for fuzzy logic

%%%%%%%%%%%%%%%%%%%%%%%%%%%%%%%%%%%%%%%%%%%%%%%%%%%%%%%%%%%%%%%%%%%%%%%%%%%
%%                               Variables                               %%
%%%%%%%%%%%%%%%%%%%%%%%%%%%%%%%%%%%%%%%%%%%%%%%%%%%%%%%%%%%%%%%%%%%%%%%%%%%

  \subsection{Variables}

    \vspace{-1cm}
\hspace{\varindent}\begin{longtable}{|p{\varnamewidth}|p{\vardescrwidth}|l}
\cline{1-2}
\cline{1-2} \centering \textbf{Name} & \centering \textbf{Description}& \\
\cline{1-2}
\endhead\cline{1-2}\multicolumn{3}{r}{\small\textit{continued on next page}}\\\endfoot\cline{1-2}
\endlastfoot\raggedright \_\-\_\-d\-o\-c\-\_\-\_\- & \raggedright \textbf{Value:} 
{\tt \texttt{...}}&\\
\cline{1-2}
\raggedright \_\-\_\-p\-a\-c\-k\-a\-g\-e\-\_\-\_\- & \raggedright \textbf{Value:} 
{\tt \texttt{'}\texttt{peach.fuzzy}\texttt{'}}&\\
\cline{1-2}
\end{longtable}


%%%%%%%%%%%%%%%%%%%%%%%%%%%%%%%%%%%%%%%%%%%%%%%%%%%%%%%%%%%%%%%%%%%%%%%%%%%
%%                           Class Description                           %%
%%%%%%%%%%%%%%%%%%%%%%%%%%%%%%%%%%%%%%%%%%%%%%%%%%%%%%%%%%%%%%%%%%%%%%%%%%%

    \index{peach \textit{(package)}!peach.fuzzy \textit{(package)}!peach.fuzzy.base \textit{(module)}!peach.fuzzy.base.FuzzySet \textit{(class)}|(}
\subsection{Class FuzzySet}

    \label{peach:fuzzy:base:FuzzySet}
\begin{tabular}{cccccccc}
% Line for object, linespec=[False, False]
\multicolumn{2}{r}{\settowidth{\BCL}{object}\multirow{2}{\BCL}{object}}
&&
&&
  \\\cline{3-3}
  &&\multicolumn{1}{c|}{}
&&
&&
  \\
% Line for numpy.ndarray, linespec=[False]
\multicolumn{4}{r}{\settowidth{\BCL}{numpy.ndarray}\multirow{2}{\BCL}{numpy.ndarray}}
&&
  \\\cline{5-5}
  &&&&\multicolumn{1}{c|}{}
&&
  \\
&&&&\multicolumn{2}{l}{\textbf{peach.fuzzy.base.FuzzySet}}
\end{tabular}


Array containing fuzzy values for a set.

This class defines the behavior of a fuzzy set. It is an array of values in
the range from 0 to 1, and the basic operations of the logic -{}- and (using
the \texttt{\&} operator); or (using the \texttt{|} operator); not (using \texttt{\textasciitilde{}}
operator) -{}- can be defined according to a set of norms. The norms can be
redefined using the appropriated methods.

To create a FuzzySet, instantiate this class with a sequence as argument,
for example:
%
\begin{quote}{\ttfamily \raggedright \noindent
fuzzy\_set~=~FuzzySet({[}~0.,~0.25,~0.5,~0.75,~1.0~{]})
}
\end{quote}

%%%%%%%%%%%%%%%%%%%%%%%%%%%%%%%%%%%%%%%%%%%%%%%%%%%%%%%%%%%%%%%%%%%%%%%%%%%
%%                                Methods                                %%
%%%%%%%%%%%%%%%%%%%%%%%%%%%%%%%%%%%%%%%%%%%%%%%%%%%%%%%%%%%%%%%%%%%%%%%%%%%

  \subsubsection{Methods}

    \label{peach:fuzzy:norms:ZadehAnd}
    \index{peach \textit{(package)}!peach.fuzzy \textit{(package)}!peach.fuzzy.norms \textit{(module)}!peach.fuzzy.norms.ZadehAnd \textit{(function)}}

    \vspace{0.5ex}

\hspace{.8\funcindent}\begin{boxedminipage}{\funcwidth}

    \raggedright \textbf{\_\_AND\_\_}(\textit{x}, \textit{y})

    \vspace{-1.5ex}

    \rule{\textwidth}{0.5\fboxrule}
\setlength{\parskip}{2ex}

And operation as defined by Lofti Zadeh.

And operation is the minimum of the two values.
\setlength{\parskip}{1ex}
      \textbf{Return Value}
    \vspace{-1ex}

      \begin{quote}

The result of the and operation.
      \end{quote}

    \end{boxedminipage}

    \label{peach:fuzzy:norms:ZadehOr}
    \index{peach \textit{(package)}!peach.fuzzy \textit{(package)}!peach.fuzzy.norms \textit{(module)}!peach.fuzzy.norms.ZadehOr \textit{(function)}}

    \vspace{0.5ex}

\hspace{.8\funcindent}\begin{boxedminipage}{\funcwidth}

    \raggedright \textbf{\_\_OR\_\_}(\textit{x}, \textit{y})

    \vspace{-1.5ex}

    \rule{\textwidth}{0.5\fboxrule}
\setlength{\parskip}{2ex}

Or operation as defined by Lofti Zadeh.

Or operation is the maximum of the two values.
\setlength{\parskip}{1ex}
      \textbf{Return Value}
    \vspace{-1ex}

      \begin{quote}

The result of the or operation.
      \end{quote}

    \end{boxedminipage}

    \label{peach:fuzzy:norms:ZadehNot}
    \index{peach \textit{(package)}!peach.fuzzy \textit{(package)}!peach.fuzzy.norms \textit{(module)}!peach.fuzzy.norms.ZadehNot \textit{(function)}}

    \vspace{0.5ex}

\hspace{.8\funcindent}\begin{boxedminipage}{\funcwidth}

    \raggedright \textbf{\_\_NOT\_\_}(\textit{x})

    \vspace{-1.5ex}

    \rule{\textwidth}{0.5\fboxrule}
\setlength{\parskip}{2ex}

Not operation as defined by Lofti Zadeh.

Not operation is the complement to 1 of the given value, that is, \texttt{1 - x}.
\setlength{\parskip}{1ex}
      \textbf{Return Value}
    \vspace{-1ex}

      \begin{quote}

The result of the not operation.
      \end{quote}

    \end{boxedminipage}

    \vspace{0.5ex}

\hspace{.8\funcindent}\begin{boxedminipage}{\funcwidth}

    \raggedright \textbf{\_\_new\_\_}(\textit{cls}, \textit{data})

    \vspace{-1.5ex}

    \rule{\textwidth}{0.5\fboxrule}
\setlength{\parskip}{2ex}

Allocates space for the array.

A fuzzy set is derived from the basic NumPy array, so the appropriate
functions and methods are called to allocate the space. In theory, the
values for a fuzzy set should be in the range \texttt{0.0 <= x <= 1.0}, but
to increase efficiency, no verification is made.
\setlength{\parskip}{1ex}
      \textbf{Return Value}
    \vspace{-1ex}

      \begin{quote}

A new array object with the fuzzy set definitions.
      {\it (type=a new object with type S, a subtype of T)}

      \end{quote}

      Overrides: object.\_\_new\_\_

    \end{boxedminipage}

    \vspace{0.5ex}

\hspace{.8\funcindent}\begin{boxedminipage}{\funcwidth}

    \raggedright \textbf{\_\_init\_\_}(\textit{self}, \textit{data}={\tt \texttt{[}\texttt{]}})

    \vspace{-1.5ex}

    \rule{\textwidth}{0.5\fboxrule}
\setlength{\parskip}{2ex}

Initializes the object.

Operations are defaulted to Zadeh norms \texttt{(max, min, 1-x)}
\setlength{\parskip}{1ex}
      Overrides: object.\_\_init\_\_

    \end{boxedminipage}

    \vspace{0.5ex}

\hspace{.8\funcindent}\begin{boxedminipage}{\funcwidth}

    \raggedright \textbf{\_\_and\_\_}(\textit{self}, \textit{a})

    \vspace{-1.5ex}

    \rule{\textwidth}{0.5\fboxrule}
\setlength{\parskip}{2ex}

Fuzzy and (\texttt{\&}) operation.
\setlength{\parskip}{1ex}
      Overrides: numpy.ndarray.\_\_and\_\_

    \end{boxedminipage}

    \vspace{0.5ex}

\hspace{.8\funcindent}\begin{boxedminipage}{\funcwidth}

    \raggedright \textbf{\_\_or\_\_}(\textit{self}, \textit{a})

    \vspace{-1.5ex}

    \rule{\textwidth}{0.5\fboxrule}
\setlength{\parskip}{2ex}

Fuzzy or (\texttt{|}) operation.
\setlength{\parskip}{1ex}
      Overrides: numpy.ndarray.\_\_or\_\_

    \end{boxedminipage}

    \vspace{0.5ex}

\hspace{.8\funcindent}\begin{boxedminipage}{\funcwidth}

    \raggedright \textbf{\_\_invert\_\_}(\textit{self})

    \vspace{-1.5ex}

    \rule{\textwidth}{0.5\fboxrule}
\setlength{\parskip}{2ex}

Fuzzy not (\texttt{\textasciitilde{}}) operation.
\setlength{\parskip}{1ex}
      Overrides: numpy.ndarray.\_\_invert\_\_

    \end{boxedminipage}

    \label{peach:fuzzy:base:FuzzySet:set_norm}
    \index{peach \textit{(package)}!peach.fuzzy \textit{(package)}!peach.fuzzy.base \textit{(module)}!peach.fuzzy.base.FuzzySet \textit{(class)}!peach.fuzzy.base.FuzzySet.set\_norm \textit{(class method)}}

    \vspace{0.5ex}

\hspace{.8\funcindent}\begin{boxedminipage}{\funcwidth}

    \raggedright \textbf{set\_norm}(\textit{cls}, \textit{f})

    \vspace{-1.5ex}

    \rule{\textwidth}{0.5\fboxrule}
\setlength{\parskip}{2ex}

Selects a t-norm (and operation)

Use this method to change the behaviour of the and operation.
\setlength{\parskip}{1ex}
      \textbf{Parameters}
      \vspace{-1ex}

      \begin{quote}
        \begin{Ventry}{x}

          \item[f]


A function of two parameters which must return the \texttt{and} of the
values.
        \end{Ventry}

      \end{quote}

    \end{boxedminipage}

    \label{peach:fuzzy:base:FuzzySet:set_conorm}
    \index{peach \textit{(package)}!peach.fuzzy \textit{(package)}!peach.fuzzy.base \textit{(module)}!peach.fuzzy.base.FuzzySet \textit{(class)}!peach.fuzzy.base.FuzzySet.set\_conorm \textit{(class method)}}

    \vspace{0.5ex}

\hspace{.8\funcindent}\begin{boxedminipage}{\funcwidth}

    \raggedright \textbf{set\_conorm}(\textit{cls}, \textit{f})

    \vspace{-1.5ex}

    \rule{\textwidth}{0.5\fboxrule}
\setlength{\parskip}{2ex}

Selects a t-conorm (or operation)

Use this method to change the behaviour of the or operation.
\setlength{\parskip}{1ex}
      \textbf{Parameters}
      \vspace{-1ex}

      \begin{quote}
        \begin{Ventry}{x}

          \item[f]


A function of two parameters which must return the \texttt{or} of the
values.
        \end{Ventry}

      \end{quote}

    \end{boxedminipage}

    \label{peach:fuzzy:base:FuzzySet:set_negation}
    \index{peach \textit{(package)}!peach.fuzzy \textit{(package)}!peach.fuzzy.base \textit{(module)}!peach.fuzzy.base.FuzzySet \textit{(class)}!peach.fuzzy.base.FuzzySet.set\_negation \textit{(class method)}}

    \vspace{0.5ex}

\hspace{.8\funcindent}\begin{boxedminipage}{\funcwidth}

    \raggedright \textbf{set\_negation}(\textit{cls}, \textit{f})

    \vspace{-1.5ex}

    \rule{\textwidth}{0.5\fboxrule}
\setlength{\parskip}{2ex}

Selects a negation (not operation)

Use this method to change the behaviour of the not operation.
\setlength{\parskip}{1ex}
      \textbf{Parameters}
      \vspace{-1ex}

      \begin{quote}
        \begin{Ventry}{x}

          \item[f]


A function of one parameter which must return the \texttt{not} of the
value.
        \end{Ventry}

      \end{quote}

    \end{boxedminipage}


\large{\textbf{\textit{Inherited from numpy.ndarray}}}

\begin{quote}
\_\_abs\_\_(), \_\_add\_\_(), \_\_array\_\_(), \_\_array\_wrap\_\_(), \_\_contains\_\_(), \_\_copy\_\_(), \_\_deepcopy\_\_(), \_\_delitem\_\_(), \_\_delslice\_\_(), \_\_div\_\_(), \_\_divmod\_\_(), \_\_eq\_\_(), \_\_float\_\_(), \_\_floordiv\_\_(), \_\_ge\_\_(), \_\_getitem\_\_(), \_\_getslice\_\_(), \_\_gt\_\_(), \_\_hex\_\_(), \_\_iadd\_\_(), \_\_iand\_\_(), \_\_idiv\_\_(), \_\_ifloordiv\_\_(), \_\_ilshift\_\_(), \_\_imod\_\_(), \_\_imul\_\_(), \_\_index\_\_(), \_\_int\_\_(), \_\_ior\_\_(), \_\_ipow\_\_(), \_\_irshift\_\_(), \_\_isub\_\_(), \_\_iter\_\_(), \_\_itruediv\_\_(), \_\_ixor\_\_(), \_\_le\_\_(), \_\_len\_\_(), \_\_long\_\_(), \_\_lshift\_\_(), \_\_lt\_\_(), \_\_mod\_\_(), \_\_mul\_\_(), \_\_ne\_\_(), \_\_neg\_\_(), \_\_nonzero\_\_(), \_\_oct\_\_(), \_\_pos\_\_(), \_\_pow\_\_(), \_\_radd\_\_(), \_\_rand\_\_(), \_\_rdiv\_\_(), \_\_rdivmod\_\_(), \_\_reduce\_\_(), \_\_repr\_\_(), \_\_rfloordiv\_\_(), \_\_rlshift\_\_(), \_\_rmod\_\_(), \_\_rmul\_\_(), \_\_ror\_\_(), \_\_rpow\_\_(), \_\_rrshift\_\_(), \_\_rshift\_\_(), \_\_rsub\_\_(), \_\_rtruediv\_\_(), \_\_rxor\_\_(), \_\_setitem\_\_(), \_\_setslice\_\_(), \_\_setstate\_\_(), \_\_str\_\_(), \_\_sub\_\_(), \_\_truediv\_\_(), \_\_xor\_\_(), all(), any(), argmax(), argmin(), argsort(), astype(), byteswap(), choose(), clip(), compress(), conj(), conjugate(), copy(), cumprod(), cumsum(), diagonal(), dump(), dumps(), fill(), flatten(), getfield(), item(), itemset(), max(), mean(), min(), newbyteorder(), nonzero(), prod(), ptp(), put(), ravel(), repeat(), reshape(), resize(), round(), searchsorted(), setfield(), setflags(), sort(), squeeze(), std(), sum(), swapaxes(), take(), tofile(), tolist(), tostring(), trace(), transpose(), var(), view()
\end{quote}

\large{\textbf{\textit{Inherited from object}}}

\begin{quote}
\_\_delattr\_\_(), \_\_format\_\_(), \_\_getattribute\_\_(), \_\_hash\_\_(), \_\_reduce\_ex\_\_(), \_\_setattr\_\_(), \_\_sizeof\_\_(), \_\_subclasshook\_\_()
\end{quote}

%%%%%%%%%%%%%%%%%%%%%%%%%%%%%%%%%%%%%%%%%%%%%%%%%%%%%%%%%%%%%%%%%%%%%%%%%%%
%%                              Properties                               %%
%%%%%%%%%%%%%%%%%%%%%%%%%%%%%%%%%%%%%%%%%%%%%%%%%%%%%%%%%%%%%%%%%%%%%%%%%%%

  \subsubsection{Properties}

    \vspace{-1cm}
\hspace{\varindent}\begin{longtable}{|p{\varnamewidth}|p{\vardescrwidth}|l}
\cline{1-2}
\cline{1-2} \centering \textbf{Name} & \centering \textbf{Description}& \\
\cline{1-2}
\endhead\cline{1-2}\multicolumn{3}{r}{\small\textit{continued on next page}}\\\endfoot\cline{1-2}
\endlastfoot\multicolumn{2}{|l|}{\textit{Inherited from numpy.ndarray}}\\
\multicolumn{2}{|p{\varwidth}|}{\raggedright T, \_\_array\_finalize\_\_, \_\_array\_interface\_\_, \_\_array\_priority\_\_, \_\_array\_struct\_\_, base, ctypes, data, dtype, flags, flat, imag, itemsize, nbytes, ndim, real, shape, size, strides}\\
\cline{1-2}
\multicolumn{2}{|l|}{\textit{Inherited from object}}\\
\multicolumn{2}{|p{\varwidth}|}{\raggedright \_\_class\_\_}\\
\cline{1-2}
\end{longtable}

    \index{peach \textit{(package)}!peach.fuzzy \textit{(package)}!peach.fuzzy.base \textit{(module)}!peach.fuzzy.base.FuzzySet \textit{(class)}|)}
    \index{peach \textit{(package)}!peach.fuzzy \textit{(package)}!peach.fuzzy.base \textit{(module)}|)}
