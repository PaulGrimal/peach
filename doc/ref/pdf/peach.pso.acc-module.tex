%
% API Documentation for Peach - Computational Intelligence for Python
% Module peach.pso.acc
%
% Generated by epydoc 3.0.1
% [Fri Feb  4 17:21:22 2011]
%

%%%%%%%%%%%%%%%%%%%%%%%%%%%%%%%%%%%%%%%%%%%%%%%%%%%%%%%%%%%%%%%%%%%%%%%%%%%
%%                          Module Description                           %%
%%%%%%%%%%%%%%%%%%%%%%%%%%%%%%%%%%%%%%%%%%%%%%%%%%%%%%%%%%%%%%%%%%%%%%%%%%%

    \index{peach \textit{(package)}!peach.pso \textit{(package)}!peach.pso.acc \textit{(module)}|(}
\section{Module peach.pso.acc}

    \label{peach:pso:acc}

Functions to update the velocity (ie, accelerate) of the particles in a swarm.

Acceleration of a particle is an important concept in the theory of particle
swarm optimizers. By choosing an adequate acceleration, particle velocity is
changed so that they can search the domain of definition of the objective
function such that there is a greater probability that a global minimum is
found. Since particle swarm optimizers are derived from genetic algorithms, it
can be said that this is what creates diversity in a swarm, such that the space
is more thoroughly searched.

%%%%%%%%%%%%%%%%%%%%%%%%%%%%%%%%%%%%%%%%%%%%%%%%%%%%%%%%%%%%%%%%%%%%%%%%%%%
%%                               Variables                               %%
%%%%%%%%%%%%%%%%%%%%%%%%%%%%%%%%%%%%%%%%%%%%%%%%%%%%%%%%%%%%%%%%%%%%%%%%%%%

  \subsection{Variables}

    \vspace{-1cm}
\hspace{\varindent}\begin{longtable}{|p{\varnamewidth}|p{\vardescrwidth}|l}
\cline{1-2}
\cline{1-2} \centering \textbf{Name} & \centering \textbf{Description}& \\
\cline{1-2}
\endhead\cline{1-2}\multicolumn{3}{r}{\small\textit{continued on next page}}\\\endfoot\cline{1-2}
\endlastfoot\raggedright \_\-\_\-d\-o\-c\-\_\-\_\- & \raggedright \textbf{Value:} 
{\tt \texttt{...}}&\\
\cline{1-2}
\raggedright \_\-\_\-p\-a\-c\-k\-a\-g\-e\-\_\-\_\- & \raggedright \textbf{Value:} 
{\tt \texttt{'}\texttt{peach.pso}\texttt{'}}&\\
\cline{1-2}
\end{longtable}


%%%%%%%%%%%%%%%%%%%%%%%%%%%%%%%%%%%%%%%%%%%%%%%%%%%%%%%%%%%%%%%%%%%%%%%%%%%
%%                           Class Description                           %%
%%%%%%%%%%%%%%%%%%%%%%%%%%%%%%%%%%%%%%%%%%%%%%%%%%%%%%%%%%%%%%%%%%%%%%%%%%%

    \index{peach \textit{(package)}!peach.pso \textit{(package)}!peach.pso.acc \textit{(module)}!peach.pso.acc.Accelerator \textit{(class)}|(}
\subsection{Class Accelerator}

    \label{peach:pso:acc:Accelerator}
\begin{tabular}{cccccc}
% Line for object, linespec=[False]
\multicolumn{2}{r}{\settowidth{\BCL}{object}\multirow{2}{\BCL}{object}}
&&
  \\\cline{3-3}
  &&\multicolumn{1}{c|}{}
&&
  \\
&&\multicolumn{2}{l}{\textbf{peach.pso.acc.Accelerator}}
\end{tabular}

\textbf{Known Subclasses:} peach.pso.acc.StandardPSO


Base class for accelerators.

This class should be derived to implement a function which computes the
acceleration of a vector of particles in a swarm. Every accelerator function
should implement at least two methods, defined below:
%
\begin{quote}
%
\begin{description}
\item[{\_\_init\_\_(self, %
\raisebox{1em}{\hypertarget{id2}{}}\hyperlink{id1}{\textbf{\color{red}*}}cnf, %
\raisebox{1em}{\hypertarget{id4}{}}\hyperlink{id3}{\textbf{\color{red}**}}kw)}] \leavevmode 
Initializes the object. There are no mandatory arguments, but any
parameters can be used here to configure the operator. For example, a
class can define a variance for randomly chose the acceleration -{}- this
should be defined here:
%
\begin{quote}{\ttfamily \raggedright \noindent
\_\_init\_\_(self,~variance=1.0)
}
\end{quote}

A default value should always be offered, if possible.

\item[{\_\_call\_\_(self, v):}] \leavevmode 
The \texttt{\_\_call\_\_} interface should be programmed to actually compute the
new velocity of a vector of particles. This method should receive a
velocity in \texttt{v} and use whatever parameters from the instantiation to
compute the new velocities. Notice that this function should operate
over a vector of velocities, not on a single velocity. This class,
however, can be instantiated with a single function that is adapted to
perform over a vector.

\end{description}

\end{quote}

%%%%%%%%%%%%%%%%%%%%%%%%%%%%%%%%%%%%%%%%%%%%%%%%%%%%%%%%%%%%%%%%%%%%%%%%%%%
%%                                Methods                                %%
%%%%%%%%%%%%%%%%%%%%%%%%%%%%%%%%%%%%%%%%%%%%%%%%%%%%%%%%%%%%%%%%%%%%%%%%%%%

  \subsubsection{Methods}

    \vspace{0.5ex}

\hspace{.8\funcindent}\begin{boxedminipage}{\funcwidth}

    \raggedright \textbf{\_\_init\_\_}(\textit{self}, \textit{f})

    \vspace{-1.5ex}

    \rule{\textwidth}{0.5\fboxrule}
\setlength{\parskip}{2ex}

Initializes an accelerator object.

This method initializes an accelerator. It receives as argument a simple
function that is adapted to operate over a vector of velocities.
\setlength{\parskip}{1ex}
      \textbf{Parameters}
      \vspace{-1ex}

      \begin{quote}
        \begin{Ventry}{x}

          \item[f]


The function to be used as acceleration. This function can be simple
function that receives a \texttt{n}-dimensional vector representing the
velocity of a single particle, where \texttt{n} is the dimensionality of
the objective function. The object then wraps the function such that
it can receive a list of velocities and applies the acceleration on
every one of them.
        \end{Ventry}

      \end{quote}

      Overrides: object.\_\_init\_\_

    \end{boxedminipage}

    \label{peach:pso:acc:Accelerator:__call__}
    \index{peach \textit{(package)}!peach.pso \textit{(package)}!peach.pso.acc \textit{(module)}!peach.pso.acc.Accelerator \textit{(class)}!peach.pso.acc.Accelerator.\_\_call\_\_ \textit{(method)}}

    \vspace{0.5ex}

\hspace{.8\funcindent}\begin{boxedminipage}{\funcwidth}

    \raggedright \textbf{\_\_call\_\_}(\textit{self}, \textit{v})

    \vspace{-1.5ex}

    \rule{\textwidth}{0.5\fboxrule}
\setlength{\parskip}{2ex}

Computes new velocities for every particle.

This method should be overloaded in implementations of different
accelerators. This method receives the velocities as a list or a vector
of the velocities (a \texttt{n}-dimensional vector in each line) or each
particle in a swarm and computes, for each one of them, a new velocity.
\setlength{\parskip}{1ex}
      \textbf{Parameters}
      \vspace{-1ex}

      \begin{quote}
        \begin{Ventry}{x}

          \item[v]


A list or a vector of velocities, where each velocity is one line of
the vector or one element of the list.
        \end{Ventry}

      \end{quote}

      \textbf{Return Value}
    \vspace{-1ex}

      \begin{quote}

A vector of the same size as the argument with the updated velocities.
The returned vector is returned as a bidimensional array.
      \end{quote}

    \end{boxedminipage}


\large{\textbf{\textit{Inherited from object}}}

\begin{quote}
\_\_delattr\_\_(), \_\_format\_\_(), \_\_getattribute\_\_(), \_\_hash\_\_(), \_\_new\_\_(), \_\_reduce\_\_(), \_\_reduce\_ex\_\_(), \_\_repr\_\_(), \_\_setattr\_\_(), \_\_sizeof\_\_(), \_\_str\_\_(), \_\_subclasshook\_\_()
\end{quote}

%%%%%%%%%%%%%%%%%%%%%%%%%%%%%%%%%%%%%%%%%%%%%%%%%%%%%%%%%%%%%%%%%%%%%%%%%%%
%%                              Properties                               %%
%%%%%%%%%%%%%%%%%%%%%%%%%%%%%%%%%%%%%%%%%%%%%%%%%%%%%%%%%%%%%%%%%%%%%%%%%%%

  \subsubsection{Properties}

    \vspace{-1cm}
\hspace{\varindent}\begin{longtable}{|p{\varnamewidth}|p{\vardescrwidth}|l}
\cline{1-2}
\cline{1-2} \centering \textbf{Name} & \centering \textbf{Description}& \\
\cline{1-2}
\endhead\cline{1-2}\multicolumn{3}{r}{\small\textit{continued on next page}}\\\endfoot\cline{1-2}
\endlastfoot\multicolumn{2}{|l|}{\textit{Inherited from object}}\\
\multicolumn{2}{|p{\varwidth}|}{\raggedright \_\_class\_\_}\\
\cline{1-2}
\end{longtable}

    \index{peach \textit{(package)}!peach.pso \textit{(package)}!peach.pso.acc \textit{(module)}!peach.pso.acc.Accelerator \textit{(class)}|)}

%%%%%%%%%%%%%%%%%%%%%%%%%%%%%%%%%%%%%%%%%%%%%%%%%%%%%%%%%%%%%%%%%%%%%%%%%%%
%%                           Class Description                           %%
%%%%%%%%%%%%%%%%%%%%%%%%%%%%%%%%%%%%%%%%%%%%%%%%%%%%%%%%%%%%%%%%%%%%%%%%%%%

    \index{peach \textit{(package)}!peach.pso \textit{(package)}!peach.pso.acc \textit{(module)}!peach.pso.acc.StandardPSO \textit{(class)}|(}
\subsection{Class StandardPSO}

    \label{peach:pso:acc:StandardPSO}
\begin{tabular}{cccccccc}
% Line for object, linespec=[False, False]
\multicolumn{2}{r}{\settowidth{\BCL}{object}\multirow{2}{\BCL}{object}}
&&
&&
  \\\cline{3-3}
  &&\multicolumn{1}{c|}{}
&&
&&
  \\
% Line for peach.pso.acc.Accelerator, linespec=[False]
\multicolumn{4}{r}{\settowidth{\BCL}{peach.pso.acc.Accelerator}\multirow{2}{\BCL}{peach.pso.acc.Accelerator}}
&&
  \\\cline{5-5}
  &&&&\multicolumn{1}{c|}{}
&&
  \\
&&&&\multicolumn{2}{l}{\textbf{peach.pso.acc.StandardPSO}}
\end{tabular}


Standard PSO Accelerator

This class implements a method for changing the velocities of particles in
a particle swarm. The standard way is to retain information on local bests
and the global bests, and update the velocity based on that.

%%%%%%%%%%%%%%%%%%%%%%%%%%%%%%%%%%%%%%%%%%%%%%%%%%%%%%%%%%%%%%%%%%%%%%%%%%%
%%                                Methods                                %%
%%%%%%%%%%%%%%%%%%%%%%%%%%%%%%%%%%%%%%%%%%%%%%%%%%%%%%%%%%%%%%%%%%%%%%%%%%%

  \subsubsection{Methods}

    \vspace{0.5ex}

\hspace{.8\funcindent}\begin{boxedminipage}{\funcwidth}

    \raggedright \textbf{\_\_init\_\_}(\textit{self}, \textit{ps}, \textit{vmax}={\tt None}, \textit{cp}={\tt 2.05}, \textit{cg}={\tt 2.05})

    \vspace{-1.5ex}

    \rule{\textwidth}{0.5\fboxrule}
\setlength{\parskip}{2ex}

Initializes the accelerator.
\setlength{\parskip}{1ex}
      \textbf{Parameters}
      \vspace{-1ex}

      \begin{quote}
        \begin{Ventry}{xx}

          \item[ps]


A reference to the Particle Swarm that should be updated. This
class, in instantiation, will assume that the position of the
particles in the moment of creation are the local best. The
objective function is computed for all particles, and the values
saved for reference in the future. Also, at the same time, the
global best is computed.
          \item[cp]


The velocity adjustment constant associated with the particle best
values. Defaults to 2.05.
          \item[cg]


The velocity adjustment constant associated with the global best
values. Defaults to 2.05. The defaults in the \texttt{cp} and \texttt{cg}
parameters are such that the inertia weight in the constrition
method satisfies \texttt{cp + cg > 4}. Please, look in the bibliography
for more information.
        \end{Ventry}

      \end{quote}

      Overrides: object.\_\_init\_\_

    \end{boxedminipage}

    \vspace{0.5ex}

\hspace{.8\funcindent}\begin{boxedminipage}{\funcwidth}

    \raggedright \textbf{\_\_call\_\_}(\textit{self}, \textit{v})

    \vspace{-1.5ex}

    \rule{\textwidth}{0.5\fboxrule}
\setlength{\parskip}{2ex}

Computes the new velocities for every particle in the swarm. This method
receives the velocities as a list or a vector of the velocities (a
\texttt{n}-dimensional vector in each line) or each particle in a swarm and
computes, for each one of them, a new velocity.
\setlength{\parskip}{1ex}
      \textbf{Parameters}
      \vspace{-1ex}

      \begin{quote}
        \begin{Ventry}{x}

          \item[v]


A list or a vector of velocities, where each velocity is one line of
the vector or one element of the list.
        \end{Ventry}

      \end{quote}

      \textbf{Return Value}
    \vspace{-1ex}

      \begin{quote}

A vector of the same size as the argument with the updated velocities.
The returned vector is returned as a bidimensional array.
      \end{quote}

      Overrides: peach.pso.acc.Accelerator.\_\_call\_\_

    \end{boxedminipage}


\large{\textbf{\textit{Inherited from object}}}

\begin{quote}
\_\_delattr\_\_(), \_\_format\_\_(), \_\_getattribute\_\_(), \_\_hash\_\_(), \_\_new\_\_(), \_\_reduce\_\_(), \_\_reduce\_ex\_\_(), \_\_repr\_\_(), \_\_setattr\_\_(), \_\_sizeof\_\_(), \_\_str\_\_(), \_\_subclasshook\_\_()
\end{quote}

%%%%%%%%%%%%%%%%%%%%%%%%%%%%%%%%%%%%%%%%%%%%%%%%%%%%%%%%%%%%%%%%%%%%%%%%%%%
%%                              Properties                               %%
%%%%%%%%%%%%%%%%%%%%%%%%%%%%%%%%%%%%%%%%%%%%%%%%%%%%%%%%%%%%%%%%%%%%%%%%%%%

  \subsubsection{Properties}

    \vspace{-1cm}
\hspace{\varindent}\begin{longtable}{|p{\varnamewidth}|p{\vardescrwidth}|l}
\cline{1-2}
\cline{1-2} \centering \textbf{Name} & \centering \textbf{Description}& \\
\cline{1-2}
\endhead\cline{1-2}\multicolumn{3}{r}{\small\textit{continued on next page}}\\\endfoot\cline{1-2}
\endlastfoot\multicolumn{2}{|l|}{\textit{Inherited from object}}\\
\multicolumn{2}{|p{\varwidth}|}{\raggedright \_\_class\_\_}\\
\cline{1-2}
\end{longtable}


%%%%%%%%%%%%%%%%%%%%%%%%%%%%%%%%%%%%%%%%%%%%%%%%%%%%%%%%%%%%%%%%%%%%%%%%%%%
%%                          Instance Variables                           %%
%%%%%%%%%%%%%%%%%%%%%%%%%%%%%%%%%%%%%%%%%%%%%%%%%%%%%%%%%%%%%%%%%%%%%%%%%%%

  \subsubsection{Instance Variables}

    \vspace{-1cm}
\hspace{\varindent}\begin{longtable}{|p{\varnamewidth}|p{\vardescrwidth}|l}
\cline{1-2}
\cline{1-2} \centering \textbf{Name} & \centering \textbf{Description}& \\
\cline{1-2}
\endhead\cline{1-2}\multicolumn{3}{r}{\small\textit{continued on next page}}\\\endfoot\cline{1-2}
\endlastfoot\raggedright c\-p\- & Velocity adjustment constant associated with the particle best values.&\\
\cline{1-2}
\raggedright c\-g\- & Velocity adjustment constant associated with the global best values.&\\
\cline{1-2}
\end{longtable}

    \index{peach \textit{(package)}!peach.pso \textit{(package)}!peach.pso.acc \textit{(module)}!peach.pso.acc.StandardPSO \textit{(class)}|)}
    \index{peach \textit{(package)}!peach.pso \textit{(package)}!peach.pso.acc \textit{(module)}|)}
