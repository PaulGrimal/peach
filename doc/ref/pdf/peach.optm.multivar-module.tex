%
% API Documentation for Peach - Computational Intelligence for Python
% Module peach.optm.multivar
%
% Generated by epydoc 3.0beta1
% [Mon Dec 21 08:51:38 2009]
%

%%%%%%%%%%%%%%%%%%%%%%%%%%%%%%%%%%%%%%%%%%%%%%%%%%%%%%%%%%%%%%%%%%%%%%%%%%%
%%                          Module Description                           %%
%%%%%%%%%%%%%%%%%%%%%%%%%%%%%%%%%%%%%%%%%%%%%%%%%%%%%%%%%%%%%%%%%%%%%%%%%%%

    \index{peach \textit{(package)}!peach.optm \textit{(package)}!peach.optm.multivar \textit{(module)}|(}
\section{Module peach.optm.multivar}

    \label{peach:optm:multivar}

This package implements basic multivariable optimizers, including gradient and
Newton searches.

%%%%%%%%%%%%%%%%%%%%%%%%%%%%%%%%%%%%%%%%%%%%%%%%%%%%%%%%%%%%%%%%%%%%%%%%%%%
%%                               Variables                               %%
%%%%%%%%%%%%%%%%%%%%%%%%%%%%%%%%%%%%%%%%%%%%%%%%%%%%%%%%%%%%%%%%%%%%%%%%%%%

  \subsection{Variables}

\begin{longtable}{|p{.30\textwidth}|p{.62\textwidth}|l}
\cline{1-2}
\cline{1-2} \centering \textbf{Name} & \centering \textbf{Description}& \\
\cline{1-2}
\endhead\cline{1-2}\multicolumn{3}{r}{\small\textit{continued on next page}}\\\endfoot\cline{1-2}
\endlastfoot\raggedright \_\-\_\-d\-o\-c\-\_\-\_\- & \raggedright \textbf{Value:} 
{\tt \texttt{...}}&\\
\cline{1-2}
\end{longtable}


%%%%%%%%%%%%%%%%%%%%%%%%%%%%%%%%%%%%%%%%%%%%%%%%%%%%%%%%%%%%%%%%%%%%%%%%%%%
%%                           Class Description                           %%
%%%%%%%%%%%%%%%%%%%%%%%%%%%%%%%%%%%%%%%%%%%%%%%%%%%%%%%%%%%%%%%%%%%%%%%%%%%

    \index{peach \textit{(package)}!peach.optm \textit{(package)}!peach.optm.multivar \textit{(module)}!peach.optm.multivar.Direct \textit{(class)}|(}
\subsection{Class Direct}

    \label{peach:optm:multivar:Direct}
\begin{tabular}{cccccccc}
% Line for object, linespec=[False, False]
\multicolumn{2}{r}{\settowidth{\BCL}{object}\multirow{2}{\BCL}{object}}
&&
&&
  \\\cline{3-3}
  &&\multicolumn{1}{c|}{}
&&
&&
  \\
% Line for peach.optm.optm.Optimizer, linespec=[False]
\multicolumn{4}{r}{\settowidth{\BCL}{peach.optm.optm.Optimizer}\multirow{2}{\BCL}{peach.optm.optm.Optimizer}}
&&
  \\\cline{5-5}
  &&&&\multicolumn{1}{c|}{}
&&
  \\
&&&&\multicolumn{2}{l}{\textbf{peach.optm.multivar.Direct}}
\end{tabular}


Multidimensional direct search

This optimization method is a generalization of the 1D method, using
variable swap as search direction. This results in a very simplistic and
inefficient method that should be used only when any other method fails.

%%%%%%%%%%%%%%%%%%%%%%%%%%%%%%%%%%%%%%%%%%%%%%%%%%%%%%%%%%%%%%%%%%%%%%%%%%%
%%                                Methods                                %%
%%%%%%%%%%%%%%%%%%%%%%%%%%%%%%%%%%%%%%%%%%%%%%%%%%%%%%%%%%%%%%%%%%%%%%%%%%%

  \subsubsection{Methods}

    \vspace{0.5ex}

    \begin{boxedminipage}{\textwidth}

    \raggedright \textbf{\_\_init\_\_}(\textit{self}, \textit{f}, \textit{h}=\texttt{0.5}, \textit{emax}=\texttt{1e-08}, \textit{imax}=\texttt{1000})

    \vspace{-1.5ex}

    \rule{\textwidth}{0.5\fboxrule}

Initializes the optimizer.

To create an optimizer of this type, instantiate the class with the
parameters given below:
    \vspace{1ex}

      \textbf{Parameters}
      \begin{quote}
        \begin{Ventry}{xxxx}

          \item[f]


A multivariable function to be optimized. The function should have
only one parameter, a multidimensional line-vector, and return the
function value, a scalar.
          \item[dx]


The initial step of the search. Defaults to 0.5
          \item[emax]


Maximum allowed error. The algorithm stops as soon as the error is
below this level. The error is absolute.
          \item[imax]


Maximum number of iterations, the algorithm stops as soon this
number of iterations are executed, no matter what the error is at
the moment.
        \end{Ventry}

      \end{quote}

    \vspace{1ex}

      Overrides: peach.optm.optm.Optimizer.\_\_init\_\_

    \end{boxedminipage}

    \vspace{0.5ex}

    \begin{boxedminipage}{\textwidth}

    \raggedright \textbf{step}(\textit{self}, \textit{x})

    \vspace{-1.5ex}

    \rule{\textwidth}{0.5\fboxrule}

One step of the search.

In this method, the result of the step is highly dependent of the steps
executed before, as the search step is updated at each call to this
method.

One characteristic of this method is that it uses the dimensions of the
input vector to initialize the updating matrix -{}- this is necessary to
mantain coherence with the interface of other methods. The matrix
dimensions are calculated and used in the first call to this method and
used in future calls. In general, there is no need to worry about this,
since everything is taken care of automatically. But, if future calls to
this method are done with incoherent dimensions, then an exception will
be raised.
    \vspace{1ex}

      \textbf{Parameters}
      \begin{quote}
        \begin{Ventry}{x}

          \item[x]


The value from where the new estimate should be calculated. This can
of course be the result of a previous iteration of the algorithm.
        \end{Ventry}

      \end{quote}

    \vspace{1ex}

      \textbf{Return Value}
      \begin{quote}

This method returns a tuple \texttt{(x, e)}, where \texttt{x} is the updated
estimate of the minimum, and \texttt{e} is the estimated error.
      \end{quote}

    \vspace{1ex}

      Overrides: peach.optm.optm.Optimizer.step

    \end{boxedminipage}

    \vspace{0.5ex}

    \begin{boxedminipage}{\textwidth}

    \raggedright \textbf{\_\_call\_\_}(\textit{self}, \textit{x})

    \vspace{-1.5ex}

    \rule{\textwidth}{0.5\fboxrule}

Transparently executes the search until the minimum is found. The stop
criteria are the maximum error or the maximum number of iterations,
whichever is reached first. Note that this is a \texttt{{\_}{\_}call{\_}{\_}} method, so
the object is called as a function. This method returns a tuple
\texttt{(x, e)}, with the best estimate of the minimum and the error.
    \vspace{1ex}

      \textbf{Parameters}
      \begin{quote}
        \begin{Ventry}{x}

          \item[x]


The value from where the search must start.
        \end{Ventry}

      \end{quote}

    \vspace{1ex}

      \textbf{Return Value}
      \begin{quote}

This method returns a tuple \texttt{(x, e)}, where \texttt{x} is the best
estimate of the minimum, and \texttt{e} is the estimated error.
      \end{quote}

    \vspace{1ex}

      Overrides: peach.optm.optm.Optimizer.\_\_call\_\_

    \end{boxedminipage}

    \label{object:__delattr__}
    \index{object.\_\_delattr\_\_ \textit{(function)}}

    \vspace{0.5ex}

    \begin{boxedminipage}{\textwidth}

    \raggedright \textbf{\_\_delattr\_\_}(\textit{...})

    \vspace{-1.5ex}

    \rule{\textwidth}{0.5\fboxrule}

x.{\_}{\_}delattr{\_}{\_}('name') {\textless}=={\textgreater} del x.name
    \vspace{1ex}

    \end{boxedminipage}

    \label{object:__getattribute__}
    \index{object.\_\_getattribute\_\_ \textit{(function)}}

    \vspace{0.5ex}

    \begin{boxedminipage}{\textwidth}

    \raggedright \textbf{\_\_getattribute\_\_}(\textit{...})

    \vspace{-1.5ex}

    \rule{\textwidth}{0.5\fboxrule}

x.{\_}{\_}getattribute{\_}{\_}('name') {\textless}=={\textgreater} x.name
    \vspace{1ex}

    \end{boxedminipage}

    \label{object:__hash__}
    \index{object.\_\_hash\_\_ \textit{(function)}}

    \vspace{0.5ex}

    \begin{boxedminipage}{\textwidth}

    \raggedright \textbf{\_\_hash\_\_}(\textit{x})

    \vspace{-1.5ex}

    \rule{\textwidth}{0.5\fboxrule}

hash(x)
    \vspace{1ex}

    \end{boxedminipage}

    \label{object:__new__}
    \index{object.\_\_new\_\_ \textit{(function)}}

    \vspace{0.5ex}

    \begin{boxedminipage}{\textwidth}

    \raggedright \textbf{\_\_new\_\_}(\textit{T}, \textit{S}, \textit{...})

      \textbf{Return Value}
      \begin{quote}
\begin{alltt}
a new object with type S, a subtype of T
\end{alltt}

      \end{quote}

    \vspace{1ex}

    \end{boxedminipage}

    \label{object:__reduce__}
    \index{object.\_\_reduce\_\_ \textit{(function)}}

    \vspace{0.5ex}

    \begin{boxedminipage}{\textwidth}

    \raggedright \textbf{\_\_reduce\_\_}(\textit{...})

    \vspace{-1.5ex}

    \rule{\textwidth}{0.5\fboxrule}

helper for pickle
    \vspace{1ex}

    \end{boxedminipage}

    \label{object:__reduce_ex__}
    \index{object.\_\_reduce\_ex\_\_ \textit{(function)}}

    \vspace{0.5ex}

    \begin{boxedminipage}{\textwidth}

    \raggedright \textbf{\_\_reduce\_ex\_\_}(\textit{...})

    \vspace{-1.5ex}

    \rule{\textwidth}{0.5\fboxrule}

helper for pickle
    \vspace{1ex}

    \end{boxedminipage}

    \label{object:__repr__}
    \index{object.\_\_repr\_\_ \textit{(function)}}

    \vspace{0.5ex}

    \begin{boxedminipage}{\textwidth}

    \raggedright \textbf{\_\_repr\_\_}(\textit{x})

    \vspace{-1.5ex}

    \rule{\textwidth}{0.5\fboxrule}

repr(x)
    \vspace{1ex}

    \end{boxedminipage}

    \label{object:__setattr__}
    \index{object.\_\_setattr\_\_ \textit{(function)}}

    \vspace{0.5ex}

    \begin{boxedminipage}{\textwidth}

    \raggedright \textbf{\_\_setattr\_\_}(\textit{...})

    \vspace{-1.5ex}

    \rule{\textwidth}{0.5\fboxrule}

x.{\_}{\_}setattr{\_}{\_}('name', value) {\textless}=={\textgreater} x.name = value
    \vspace{1ex}

    \end{boxedminipage}

    \label{object:__str__}
    \index{object.\_\_str\_\_ \textit{(function)}}

    \vspace{0.5ex}

    \begin{boxedminipage}{\textwidth}

    \raggedright \textbf{\_\_str\_\_}(\textit{x})

    \vspace{-1.5ex}

    \rule{\textwidth}{0.5\fboxrule}

str(x)
    \vspace{1ex}

    \end{boxedminipage}


%%%%%%%%%%%%%%%%%%%%%%%%%%%%%%%%%%%%%%%%%%%%%%%%%%%%%%%%%%%%%%%%%%%%%%%%%%%
%%                              Properties                               %%
%%%%%%%%%%%%%%%%%%%%%%%%%%%%%%%%%%%%%%%%%%%%%%%%%%%%%%%%%%%%%%%%%%%%%%%%%%%

  \subsubsection{Properties}

\begin{longtable}{|p{.30\textwidth}|p{.62\textwidth}|l}
\cline{1-2}
\cline{1-2} \centering \textbf{Name} & \centering \textbf{Description}& \\
\cline{1-2}
\endhead\cline{1-2}\multicolumn{3}{r}{\small\textit{continued on next page}}\\\endfoot\cline{1-2}
\endlastfoot\raggedright \_\-\_\-c\-l\-a\-s\-s\-\_\-\_\- & \raggedright \textbf{Value:} 
{\tt {\textless}attribute '\_\_class\_\_' of 'object' objects{\textgreater}}&\\
\cline{1-2}
\end{longtable}

    \index{peach \textit{(package)}!peach.optm \textit{(package)}!peach.optm.multivar \textit{(module)}!peach.optm.multivar.Direct \textit{(class)}|)}

%%%%%%%%%%%%%%%%%%%%%%%%%%%%%%%%%%%%%%%%%%%%%%%%%%%%%%%%%%%%%%%%%%%%%%%%%%%
%%                           Class Description                           %%
%%%%%%%%%%%%%%%%%%%%%%%%%%%%%%%%%%%%%%%%%%%%%%%%%%%%%%%%%%%%%%%%%%%%%%%%%%%

    \index{peach \textit{(package)}!peach.optm \textit{(package)}!peach.optm.multivar \textit{(module)}!peach.optm.multivar.Gradient \textit{(class)}|(}
\subsection{Class Gradient}

    \label{peach:optm:multivar:Gradient}
\begin{tabular}{cccccccc}
% Line for object, linespec=[False, False]
\multicolumn{2}{r}{\settowidth{\BCL}{object}\multirow{2}{\BCL}{object}}
&&
&&
  \\\cline{3-3}
  &&\multicolumn{1}{c|}{}
&&
&&
  \\
% Line for peach.optm.optm.Optimizer, linespec=[False]
\multicolumn{4}{r}{\settowidth{\BCL}{peach.optm.optm.Optimizer}\multirow{2}{\BCL}{peach.optm.optm.Optimizer}}
&&
  \\\cline{5-5}
  &&&&\multicolumn{1}{c|}{}
&&
  \\
&&&&\multicolumn{2}{l}{\textbf{peach.optm.multivar.Gradient}}
\end{tabular}


Gradient search

This method uses the fact that the gradient of a function points to the
direction of largest increase in the function (in general called \emph{uphill}
direction). So, the contrary direction (\emph{downhill}) is used as search
direction.

%%%%%%%%%%%%%%%%%%%%%%%%%%%%%%%%%%%%%%%%%%%%%%%%%%%%%%%%%%%%%%%%%%%%%%%%%%%
%%                                Methods                                %%
%%%%%%%%%%%%%%%%%%%%%%%%%%%%%%%%%%%%%%%%%%%%%%%%%%%%%%%%%%%%%%%%%%%%%%%%%%%

  \subsubsection{Methods}

    \vspace{0.5ex}

    \begin{boxedminipage}{\textwidth}

    \raggedright \textbf{\_\_init\_\_}(\textit{self}, \textit{f}, \textit{df}=\texttt{None}, \textit{h}=\texttt{0.1}, \textit{emax}=\texttt{1e-05}, \textit{imax}=\texttt{1000})

    \vspace{-1.5ex}

    \rule{\textwidth}{0.5\fboxrule}

Initializes the optimizer.

To create an optimizer of this type, instantiate the class with the
parameters given below:
    \vspace{1ex}

      \textbf{Parameters}
      \begin{quote}
        \begin{Ventry}{xxxx}

          \item[f]


A multivariable function to be optimized. The function should have
only one parameter, a multidimensional line-vector, and return the
function value, a scalar.
          \item[df]


A function to calculate the gradient vector of the cost function
\texttt{f}. Defaults to \texttt{None}, if no gradient is supplied, then it is
estimated from the cost function using Euler equations.
          \item[h]


Convergence step. This method does not takes into consideration the
possibility of varying the convergence step, to avoid Stiefel cages.
          \item[emax]


Maximum allowed error. The algorithm stops as soon as the error is
below this level. The error is absolute.
          \item[imax]


Maximum number of iterations, the algorithm stops as soon this
number of iterations are executed, no matter what the error is at
the moment.
        \end{Ventry}

      \end{quote}

    \vspace{1ex}

      Overrides: peach.optm.optm.Optimizer.\_\_init\_\_

    \end{boxedminipage}

    \vspace{0.5ex}

    \begin{boxedminipage}{\textwidth}

    \raggedright \textbf{step}(\textit{self}, \textit{x})

    \vspace{-1.5ex}

    \rule{\textwidth}{0.5\fboxrule}

One step of the search.

In this method, the result of the step is dependent only of the given
estimated, so it can be used for different kind of investigations on the
same cost function.
    \vspace{1ex}

      \textbf{Parameters}
      \begin{quote}
        \begin{Ventry}{x}

          \item[x]


The value from where the new estimate should be calculated. This can
of course be the result of a previous iteration of the algorithm.
        \end{Ventry}

      \end{quote}

    \vspace{1ex}

      \textbf{Return Value}
      \begin{quote}

This method returns a tuple \texttt{(x, e)}, where \texttt{x} is the updated
estimate of the minimum, and \texttt{e} is the estimated error.
      \end{quote}

    \vspace{1ex}

      Overrides: peach.optm.optm.Optimizer.step

    \end{boxedminipage}

    \vspace{0.5ex}

    \begin{boxedminipage}{\textwidth}

    \raggedright \textbf{\_\_call\_\_}(\textit{self}, \textit{x})

    \vspace{-1.5ex}

    \rule{\textwidth}{0.5\fboxrule}

Transparently executes the search until the minimum is found. The stop
criteria are the maximum error or the maximum number of iterations,
whichever is reached first. Note that this is a \texttt{{\_}{\_}call{\_}{\_}} method, so
the object is called as a function. This method returns a tuple
\texttt{(x, e)}, with the best estimate of the minimum and the error.
    \vspace{1ex}

      \textbf{Parameters}
      \begin{quote}
        \begin{Ventry}{x}

          \item[x]


The initial triplet of values from where the search must start.
        \end{Ventry}

      \end{quote}

    \vspace{1ex}

      \textbf{Return Value}
      \begin{quote}

This method returns a tuple \texttt{(x, e)}, where \texttt{x} is the best
estimate of the minimum, and \texttt{e} is the estimated error.
      \end{quote}

    \vspace{1ex}

      Overrides: peach.optm.optm.Optimizer.\_\_call\_\_

    \end{boxedminipage}

    \label{object:__delattr__}
    \index{object.\_\_delattr\_\_ \textit{(function)}}

    \vspace{0.5ex}

    \begin{boxedminipage}{\textwidth}

    \raggedright \textbf{\_\_delattr\_\_}(\textit{...})

    \vspace{-1.5ex}

    \rule{\textwidth}{0.5\fboxrule}

x.{\_}{\_}delattr{\_}{\_}('name') {\textless}=={\textgreater} del x.name
    \vspace{1ex}

    \end{boxedminipage}

    \label{object:__getattribute__}
    \index{object.\_\_getattribute\_\_ \textit{(function)}}

    \vspace{0.5ex}

    \begin{boxedminipage}{\textwidth}

    \raggedright \textbf{\_\_getattribute\_\_}(\textit{...})

    \vspace{-1.5ex}

    \rule{\textwidth}{0.5\fboxrule}

x.{\_}{\_}getattribute{\_}{\_}('name') {\textless}=={\textgreater} x.name
    \vspace{1ex}

    \end{boxedminipage}

    \label{object:__hash__}
    \index{object.\_\_hash\_\_ \textit{(function)}}

    \vspace{0.5ex}

    \begin{boxedminipage}{\textwidth}

    \raggedright \textbf{\_\_hash\_\_}(\textit{x})

    \vspace{-1.5ex}

    \rule{\textwidth}{0.5\fboxrule}

hash(x)
    \vspace{1ex}

    \end{boxedminipage}

    \label{object:__new__}
    \index{object.\_\_new\_\_ \textit{(function)}}

    \vspace{0.5ex}

    \begin{boxedminipage}{\textwidth}

    \raggedright \textbf{\_\_new\_\_}(\textit{T}, \textit{S}, \textit{...})

      \textbf{Return Value}
      \begin{quote}
\begin{alltt}
a new object with type S, a subtype of T
\end{alltt}

      \end{quote}

    \vspace{1ex}

    \end{boxedminipage}

    \label{object:__reduce__}
    \index{object.\_\_reduce\_\_ \textit{(function)}}

    \vspace{0.5ex}

    \begin{boxedminipage}{\textwidth}

    \raggedright \textbf{\_\_reduce\_\_}(\textit{...})

    \vspace{-1.5ex}

    \rule{\textwidth}{0.5\fboxrule}

helper for pickle
    \vspace{1ex}

    \end{boxedminipage}

    \label{object:__reduce_ex__}
    \index{object.\_\_reduce\_ex\_\_ \textit{(function)}}

    \vspace{0.5ex}

    \begin{boxedminipage}{\textwidth}

    \raggedright \textbf{\_\_reduce\_ex\_\_}(\textit{...})

    \vspace{-1.5ex}

    \rule{\textwidth}{0.5\fboxrule}

helper for pickle
    \vspace{1ex}

    \end{boxedminipage}

    \label{object:__repr__}
    \index{object.\_\_repr\_\_ \textit{(function)}}

    \vspace{0.5ex}

    \begin{boxedminipage}{\textwidth}

    \raggedright \textbf{\_\_repr\_\_}(\textit{x})

    \vspace{-1.5ex}

    \rule{\textwidth}{0.5\fboxrule}

repr(x)
    \vspace{1ex}

    \end{boxedminipage}

    \label{object:__setattr__}
    \index{object.\_\_setattr\_\_ \textit{(function)}}

    \vspace{0.5ex}

    \begin{boxedminipage}{\textwidth}

    \raggedright \textbf{\_\_setattr\_\_}(\textit{...})

    \vspace{-1.5ex}

    \rule{\textwidth}{0.5\fboxrule}

x.{\_}{\_}setattr{\_}{\_}('name', value) {\textless}=={\textgreater} x.name = value
    \vspace{1ex}

    \end{boxedminipage}

    \label{object:__str__}
    \index{object.\_\_str\_\_ \textit{(function)}}

    \vspace{0.5ex}

    \begin{boxedminipage}{\textwidth}

    \raggedright \textbf{\_\_str\_\_}(\textit{x})

    \vspace{-1.5ex}

    \rule{\textwidth}{0.5\fboxrule}

str(x)
    \vspace{1ex}

    \end{boxedminipage}


%%%%%%%%%%%%%%%%%%%%%%%%%%%%%%%%%%%%%%%%%%%%%%%%%%%%%%%%%%%%%%%%%%%%%%%%%%%
%%                              Properties                               %%
%%%%%%%%%%%%%%%%%%%%%%%%%%%%%%%%%%%%%%%%%%%%%%%%%%%%%%%%%%%%%%%%%%%%%%%%%%%

  \subsubsection{Properties}

\begin{longtable}{|p{.30\textwidth}|p{.62\textwidth}|l}
\cline{1-2}
\cline{1-2} \centering \textbf{Name} & \centering \textbf{Description}& \\
\cline{1-2}
\endhead\cline{1-2}\multicolumn{3}{r}{\small\textit{continued on next page}}\\\endfoot\cline{1-2}
\endlastfoot\raggedright \_\-\_\-c\-l\-a\-s\-s\-\_\-\_\- & \raggedright \textbf{Value:} 
{\tt {\textless}attribute '\_\_class\_\_' of 'object' objects{\textgreater}}&\\
\cline{1-2}
\end{longtable}

    \index{peach \textit{(package)}!peach.optm \textit{(package)}!peach.optm.multivar \textit{(module)}!peach.optm.multivar.Gradient \textit{(class)}|)}

%%%%%%%%%%%%%%%%%%%%%%%%%%%%%%%%%%%%%%%%%%%%%%%%%%%%%%%%%%%%%%%%%%%%%%%%%%%
%%                           Class Description                           %%
%%%%%%%%%%%%%%%%%%%%%%%%%%%%%%%%%%%%%%%%%%%%%%%%%%%%%%%%%%%%%%%%%%%%%%%%%%%

    \index{peach \textit{(package)}!peach.optm \textit{(package)}!peach.optm.multivar \textit{(module)}!peach.optm.multivar.Newton \textit{(class)}|(}
\subsection{Class Newton}

    \label{peach:optm:multivar:Newton}
\begin{tabular}{cccccccc}
% Line for object, linespec=[False, False]
\multicolumn{2}{r}{\settowidth{\BCL}{object}\multirow{2}{\BCL}{object}}
&&
&&
  \\\cline{3-3}
  &&\multicolumn{1}{c|}{}
&&
&&
  \\
% Line for peach.optm.optm.Optimizer, linespec=[False]
\multicolumn{4}{r}{\settowidth{\BCL}{peach.optm.optm.Optimizer}\multirow{2}{\BCL}{peach.optm.optm.Optimizer}}
&&
  \\\cline{5-5}
  &&&&\multicolumn{1}{c|}{}
&&
  \\
&&&&\multicolumn{2}{l}{\textbf{peach.optm.multivar.Newton}}
\end{tabular}


Newton search

This is a very effective method to find minimum points in functions. In a
very basic fashion, this method corresponds to using Newton root finding
method on f'(x). Converges \emph{very} fast if the cost function is quadratic
of simmilar to it.

%%%%%%%%%%%%%%%%%%%%%%%%%%%%%%%%%%%%%%%%%%%%%%%%%%%%%%%%%%%%%%%%%%%%%%%%%%%
%%                                Methods                                %%
%%%%%%%%%%%%%%%%%%%%%%%%%%%%%%%%%%%%%%%%%%%%%%%%%%%%%%%%%%%%%%%%%%%%%%%%%%%

  \subsubsection{Methods}

    \vspace{0.5ex}

    \begin{boxedminipage}{\textwidth}

    \raggedright \textbf{\_\_init\_\_}(\textit{self}, \textit{f}, \textit{df}=\texttt{None}, \textit{hf}=\texttt{None}, \textit{h}=\texttt{0.1}, \textit{emax}=\texttt{1e-05}, \textit{imax}=\texttt{1000})

    \vspace{-1.5ex}

    \rule{\textwidth}{0.5\fboxrule}

Initializes the optimizer.

To create an optimizer of this type, instantiate the class with the
parameters given below:
    \vspace{1ex}

      \textbf{Parameters}
      \begin{quote}
        \begin{Ventry}{xxxx}

          \item[f]


A multivariable function to be optimized. The function should have
only one parameter, a multidimensional line-vector, and return the
function value, a scalar.
          \item[df]


A function to calculate the gradient vector of the cost function
\texttt{f}. Defaults to \texttt{None}, if no gradient is supplied, then it is
estimated from the cost function using Euler equations.
          \item[hf]


A function to calculate the hessian matrix of the cost function
\texttt{f}. Defaults to \texttt{None}, if no hessian is supplied, then it is
estimated from the cost function using Euler equations.
          \item[h]


Convergence step. This method does not takes into consideration the
possibility of varying the convergence step, to avoid Stiefel cages.
          \item[emax]


Maximum allowed error. The algorithm stops as soon as the error is
below this level. The error is absolute.
          \item[imax]


Maximum number of iterations, the algorithm stops as soon this
number of iterations are executed, no matter what the error is at
the moment.
        \end{Ventry}

      \end{quote}

    \vspace{1ex}

      Overrides: peach.optm.optm.Optimizer.\_\_init\_\_

    \end{boxedminipage}

    \vspace{0.5ex}

    \begin{boxedminipage}{\textwidth}

    \raggedright \textbf{step}(\textit{self}, \textit{x})

    \vspace{-1.5ex}

    \rule{\textwidth}{0.5\fboxrule}

One step of the search.

In this method, the result of the step is dependent only of the given
estimated, so it can be used for different kind of investigations on the
same cost function.
    \vspace{1ex}

      \textbf{Parameters}
      \begin{quote}
        \begin{Ventry}{x}

          \item[x]


The value from where the new estimate should be calculated. This can
of course be the result of a previous iteration of the algorithm.
        \end{Ventry}

      \end{quote}

    \vspace{1ex}

      \textbf{Return Value}
      \begin{quote}

This method returns a tuple \texttt{(x, e)}, where \texttt{x} is the updated
estimate of the minimum, and \texttt{e} is the estimated error.
      \end{quote}

    \vspace{1ex}

      Overrides: peach.optm.optm.Optimizer.step

    \end{boxedminipage}

    \vspace{0.5ex}

    \begin{boxedminipage}{\textwidth}

    \raggedright \textbf{\_\_call\_\_}(\textit{self}, \textit{x})

    \vspace{-1.5ex}

    \rule{\textwidth}{0.5\fboxrule}

Transparently executes the search until the minimum is found. The stop
criteria are the maximum error or the maximum number of iterations,
whichever is reached first. Note that this is a \texttt{{\_}{\_}call{\_}{\_}} method, so
the object is called as a function. This method returns a tuple
\texttt{(x, e)}, with the best estimate of the minimum and the error.
    \vspace{1ex}

      \textbf{Parameters}
      \begin{quote}
        \begin{Ventry}{x}

          \item[x]


The initial triplet of values from where the search must start.
        \end{Ventry}

      \end{quote}

    \vspace{1ex}

      \textbf{Return Value}
      \begin{quote}

This method returns a tuple \texttt{(x, e)}, where \texttt{x} is the best
estimate of the minimum, and \texttt{e} is the estimated error.
      \end{quote}

    \vspace{1ex}

      Overrides: peach.optm.optm.Optimizer.\_\_call\_\_

    \end{boxedminipage}

    \label{object:__delattr__}
    \index{object.\_\_delattr\_\_ \textit{(function)}}

    \vspace{0.5ex}

    \begin{boxedminipage}{\textwidth}

    \raggedright \textbf{\_\_delattr\_\_}(\textit{...})

    \vspace{-1.5ex}

    \rule{\textwidth}{0.5\fboxrule}

x.{\_}{\_}delattr{\_}{\_}('name') {\textless}=={\textgreater} del x.name
    \vspace{1ex}

    \end{boxedminipage}

    \label{object:__getattribute__}
    \index{object.\_\_getattribute\_\_ \textit{(function)}}

    \vspace{0.5ex}

    \begin{boxedminipage}{\textwidth}

    \raggedright \textbf{\_\_getattribute\_\_}(\textit{...})

    \vspace{-1.5ex}

    \rule{\textwidth}{0.5\fboxrule}

x.{\_}{\_}getattribute{\_}{\_}('name') {\textless}=={\textgreater} x.name
    \vspace{1ex}

    \end{boxedminipage}

    \label{object:__hash__}
    \index{object.\_\_hash\_\_ \textit{(function)}}

    \vspace{0.5ex}

    \begin{boxedminipage}{\textwidth}

    \raggedright \textbf{\_\_hash\_\_}(\textit{x})

    \vspace{-1.5ex}

    \rule{\textwidth}{0.5\fboxrule}

hash(x)
    \vspace{1ex}

    \end{boxedminipage}

    \label{object:__new__}
    \index{object.\_\_new\_\_ \textit{(function)}}

    \vspace{0.5ex}

    \begin{boxedminipage}{\textwidth}

    \raggedright \textbf{\_\_new\_\_}(\textit{T}, \textit{S}, \textit{...})

      \textbf{Return Value}
      \begin{quote}
\begin{alltt}
a new object with type S, a subtype of T
\end{alltt}

      \end{quote}

    \vspace{1ex}

    \end{boxedminipage}

    \label{object:__reduce__}
    \index{object.\_\_reduce\_\_ \textit{(function)}}

    \vspace{0.5ex}

    \begin{boxedminipage}{\textwidth}

    \raggedright \textbf{\_\_reduce\_\_}(\textit{...})

    \vspace{-1.5ex}

    \rule{\textwidth}{0.5\fboxrule}

helper for pickle
    \vspace{1ex}

    \end{boxedminipage}

    \label{object:__reduce_ex__}
    \index{object.\_\_reduce\_ex\_\_ \textit{(function)}}

    \vspace{0.5ex}

    \begin{boxedminipage}{\textwidth}

    \raggedright \textbf{\_\_reduce\_ex\_\_}(\textit{...})

    \vspace{-1.5ex}

    \rule{\textwidth}{0.5\fboxrule}

helper for pickle
    \vspace{1ex}

    \end{boxedminipage}

    \label{object:__repr__}
    \index{object.\_\_repr\_\_ \textit{(function)}}

    \vspace{0.5ex}

    \begin{boxedminipage}{\textwidth}

    \raggedright \textbf{\_\_repr\_\_}(\textit{x})

    \vspace{-1.5ex}

    \rule{\textwidth}{0.5\fboxrule}

repr(x)
    \vspace{1ex}

    \end{boxedminipage}

    \label{object:__setattr__}
    \index{object.\_\_setattr\_\_ \textit{(function)}}

    \vspace{0.5ex}

    \begin{boxedminipage}{\textwidth}

    \raggedright \textbf{\_\_setattr\_\_}(\textit{...})

    \vspace{-1.5ex}

    \rule{\textwidth}{0.5\fboxrule}

x.{\_}{\_}setattr{\_}{\_}('name', value) {\textless}=={\textgreater} x.name = value
    \vspace{1ex}

    \end{boxedminipage}

    \label{object:__str__}
    \index{object.\_\_str\_\_ \textit{(function)}}

    \vspace{0.5ex}

    \begin{boxedminipage}{\textwidth}

    \raggedright \textbf{\_\_str\_\_}(\textit{x})

    \vspace{-1.5ex}

    \rule{\textwidth}{0.5\fboxrule}

str(x)
    \vspace{1ex}

    \end{boxedminipage}


%%%%%%%%%%%%%%%%%%%%%%%%%%%%%%%%%%%%%%%%%%%%%%%%%%%%%%%%%%%%%%%%%%%%%%%%%%%
%%                              Properties                               %%
%%%%%%%%%%%%%%%%%%%%%%%%%%%%%%%%%%%%%%%%%%%%%%%%%%%%%%%%%%%%%%%%%%%%%%%%%%%

  \subsubsection{Properties}

\begin{longtable}{|p{.30\textwidth}|p{.62\textwidth}|l}
\cline{1-2}
\cline{1-2} \centering \textbf{Name} & \centering \textbf{Description}& \\
\cline{1-2}
\endhead\cline{1-2}\multicolumn{3}{r}{\small\textit{continued on next page}}\\\endfoot\cline{1-2}
\endlastfoot\raggedright \_\-\_\-c\-l\-a\-s\-s\-\_\-\_\- & \raggedright \textbf{Value:} 
{\tt {\textless}attribute '\_\_class\_\_' of 'object' objects{\textgreater}}&\\
\cline{1-2}
\end{longtable}

    \index{peach \textit{(package)}!peach.optm \textit{(package)}!peach.optm.multivar \textit{(module)}!peach.optm.multivar.Newton \textit{(class)}|)}
    \index{peach \textit{(package)}!peach.optm \textit{(package)}!peach.optm.multivar \textit{(module)}|)}
