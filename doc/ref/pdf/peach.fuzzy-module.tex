%
% API Documentation for Peach - Computational Intelligence for Python
% Package peach.fuzzy
%
% Generated by epydoc 3.0.1
% [Sun Jul 31 17:00:39 2011]
%

%%%%%%%%%%%%%%%%%%%%%%%%%%%%%%%%%%%%%%%%%%%%%%%%%%%%%%%%%%%%%%%%%%%%%%%%%%%
%%                          Module Description                           %%
%%%%%%%%%%%%%%%%%%%%%%%%%%%%%%%%%%%%%%%%%%%%%%%%%%%%%%%%%%%%%%%%%%%%%%%%%%%

    \index{peach \textit{(package)}!peach.fuzzy \textit{(package)}|(}
\section{Package peach.fuzzy}

    \label{peach:fuzzy}

This package implements fuzzy logic. Consult:
%
\begin{quote}
%
\begin{description}
\item[{base}] \leavevmode 
Basic definitions, classes and operations in fuzzy logic;

\item[{mf}] \leavevmode 
Membership functions;

\item[{defuzzy}] \leavevmode 
Defuzzification methods;

\item[{control}] \leavevmode 
Fuzzy controllers (FIS - Fuzzy Inference Systems), for Mamdani- and
Sugeno-type controllers and others;

\item[{cmeans}] \leavevmode 
Fuzzy C-Means clustering algorithm;

\end{description}

\end{quote}

%%%%%%%%%%%%%%%%%%%%%%%%%%%%%%%%%%%%%%%%%%%%%%%%%%%%%%%%%%%%%%%%%%%%%%%%%%%
%%                                Modules                                %%
%%%%%%%%%%%%%%%%%%%%%%%%%%%%%%%%%%%%%%%%%%%%%%%%%%%%%%%%%%%%%%%%%%%%%%%%%%%

\subsection{Modules}

\begin{itemize}
\setlength{\parskip}{0ex}
\item \textbf{base}: 
This package implements basic definitions for fuzzy logic


  \textit{(Section \ref{peach:fuzzy:base}, p.~\pageref{peach:fuzzy:base})}

\item \textbf{cmeans}: 
Fuzzy C-Means


  \textit{(Section \ref{peach:fuzzy:cmeans}, p.~\pageref{peach:fuzzy:cmeans})}

\item \textbf{control}: 
This package implements fuzzy controllers, of fuzzy inference systems.


  \textit{(Section \ref{peach:fuzzy:control}, p.~\pageref{peach:fuzzy:control})}

\item \textbf{defuzzy}: 
This package implements defuzzification methods for use with fuzzy controllers.


  \textit{(Section \ref{peach:fuzzy:defuzzy}, p.~\pageref{peach:fuzzy:defuzzy})}

\item \textbf{mf}: 
Membership functions


  \textit{(Section \ref{peach:fuzzy:mf}, p.~\pageref{peach:fuzzy:mf})}

\item \textbf{norms}: 
This package implements operations of fuzzy logic.


  \textit{(Section \ref{peach:fuzzy:norms}, p.~\pageref{peach:fuzzy:norms})}

\end{itemize}

    \index{peach \textit{(package)}!peach.fuzzy \textit{(package)}|)}
