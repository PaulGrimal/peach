%
% API Documentation for Peach - Computational Intelligence for Python
% Module peach.optm.sa
%
% Generated by epydoc 3.0beta1
% [Mon Dec 21 08:51:38 2009]
%

%%%%%%%%%%%%%%%%%%%%%%%%%%%%%%%%%%%%%%%%%%%%%%%%%%%%%%%%%%%%%%%%%%%%%%%%%%%
%%                          Module Description                           %%
%%%%%%%%%%%%%%%%%%%%%%%%%%%%%%%%%%%%%%%%%%%%%%%%%%%%%%%%%%%%%%%%%%%%%%%%%%%

    \index{peach \textit{(package)}!peach.optm \textit{(package)}!peach.optm.sa \textit{(module)}|(}
\section{Module peach.optm.sa}

    \label{peach:optm:sa}

This package implements two versions of simulated annealing optimization. One
works with numeric data, and the other with a codified bit string. This last
method can be used in discrete optimization problems.

%%%%%%%%%%%%%%%%%%%%%%%%%%%%%%%%%%%%%%%%%%%%%%%%%%%%%%%%%%%%%%%%%%%%%%%%%%%
%%                               Variables                               %%
%%%%%%%%%%%%%%%%%%%%%%%%%%%%%%%%%%%%%%%%%%%%%%%%%%%%%%%%%%%%%%%%%%%%%%%%%%%

  \subsection{Variables}

\begin{longtable}{|p{.30\textwidth}|p{.62\textwidth}|l}
\cline{1-2}
\cline{1-2} \centering \textbf{Name} & \centering \textbf{Description}& \\
\cline{1-2}
\endhead\cline{1-2}\multicolumn{3}{r}{\small\textit{continued on next page}}\\\endfoot\cline{1-2}
\endlastfoot\raggedright \_\-\_\-d\-o\-c\-\_\-\_\- & \raggedright \textbf{Value:} 
{\tt \texttt{...}}&\\
\cline{1-2}
\end{longtable}


%%%%%%%%%%%%%%%%%%%%%%%%%%%%%%%%%%%%%%%%%%%%%%%%%%%%%%%%%%%%%%%%%%%%%%%%%%%
%%                           Class Description                           %%
%%%%%%%%%%%%%%%%%%%%%%%%%%%%%%%%%%%%%%%%%%%%%%%%%%%%%%%%%%%%%%%%%%%%%%%%%%%

    \index{peach \textit{(package)}!peach.optm \textit{(package)}!peach.optm.sa \textit{(module)}!peach.optm.sa.ContinuousSA \textit{(class)}|(}
\subsection{Class ContinuousSA}

    \label{peach:optm:sa:ContinuousSA}
\begin{tabular}{cccccccc}
% Line for object, linespec=[False, False]
\multicolumn{2}{r}{\settowidth{\BCL}{object}\multirow{2}{\BCL}{object}}
&&
&&
  \\\cline{3-3}
  &&\multicolumn{1}{c|}{}
&&
&&
  \\
% Line for peach.optm.optm.Optimizer, linespec=[False]
\multicolumn{4}{r}{\settowidth{\BCL}{peach.optm.optm.Optimizer}\multirow{2}{\BCL}{peach.optm.optm.Optimizer}}
&&
  \\\cline{5-5}
  &&&&\multicolumn{1}{c|}{}
&&
  \\
&&&&\multicolumn{2}{l}{\textbf{peach.optm.sa.ContinuousSA}}
\end{tabular}


Simulated Annealing continuous optimization.

This is a simulated annealing optimizer implemented to work with vectors of
continuous variables (obviouslly, implemented as floating point numbers). In
general, simulated annealing methods searches for neighbors of one estimate,
which makes a lot more sense in discrete problems. While in this class the
method is implemented in a different way (to deal with continuous
variables), the principle is pretty much the same -{}- the neighbor is found
based on a gaussian neighborhood.

A simulated annealing algorithm adapted to deal with continuous variables
has an enhancement that can be used: a gradient vector can be given and, in
case the neighbor is not accepted, the estimate is updated in the downhill
direction.

%%%%%%%%%%%%%%%%%%%%%%%%%%%%%%%%%%%%%%%%%%%%%%%%%%%%%%%%%%%%%%%%%%%%%%%%%%%
%%                                Methods                                %%
%%%%%%%%%%%%%%%%%%%%%%%%%%%%%%%%%%%%%%%%%%%%%%%%%%%%%%%%%%%%%%%%%%%%%%%%%%%

  \subsubsection{Methods}

    \vspace{0.5ex}

    \begin{boxedminipage}{\textwidth}

    \raggedright \textbf{\_\_init\_\_}(\textit{self}, \textit{f}, \textit{df}=\texttt{None}, \textit{T0}=\texttt{1000.0}, \textit{rt}=\texttt{0.95}, \textit{h}=\texttt{0.05}, \textit{emax}=\texttt{1e-08}, \textit{imax}=\texttt{1000})

    \vspace{-1.5ex}

    \rule{\textwidth}{0.5\fboxrule}

Initializes the optimizer.

To create an optimizer of this type, instantiate the class with the
parameters given below:
    \vspace{1ex}

      \textbf{Parameters}
      \begin{quote}
        \begin{Ventry}{xxxx}

          \item[f]


A multivariable function to be optimized. The function should have
only one parameter, a multidimensional line-vector, and return the
function value, a scalar.
          \item[df]


A function to calculate the gradient vector of the cost function
\texttt{f}. Defaults to \texttt{None}, if no gradient is supplied, then it is
estimated from the cost function using Euler equations.
          \item[T0]


Initial temperature of the system. The temperature is, of course, an
analogy. Defaults to 1000.
          \item[rt]


Temperature decreasing rate. The temperature must slowly decrease in
simulated annealing algorithms. In this implementation, this is
controlled by this parameter. At each step, the temperature is
multiplied by this value, so it is necessary that \texttt{0 < rt < 1}.
Defaults to 0.95, smaller values make the temperature decay faster,
while larger values make the temperature decay slower.
          \item[h]


Convergence step. In the case that the neighbor estimate is not
accepted, a simple gradient step is executed. This parameter is the
convergence step to the gradient step.
          \item[emax]


Maximum allowed error. The algorithm stops as soon as the error is
below this level. The error is absolute.
          \item[imax]


Maximum number of iterations, the algorithm stops as soon this
number of iterations are executed, no matter what the error is at
the moment.
        \end{Ventry}

      \end{quote}

    \vspace{1ex}

      Overrides: peach.optm.optm.Optimizer.\_\_init\_\_

    \end{boxedminipage}

    \vspace{0.5ex}

    \begin{boxedminipage}{\textwidth}

    \raggedright \textbf{step}(\textit{self}, \textit{x})

    \vspace{-1.5ex}

    \rule{\textwidth}{0.5\fboxrule}

One step of the search.

In this method, a neighbor of the given estimate is chosen at random,
using a gaussian neighborhood. It is accepted as a new estimate if it
performs better in the cost function \emph{or} if the temperature is high
enough. In case it is not accepted, a gradient step is executed.
    \vspace{1ex}

      \textbf{Parameters}
      \begin{quote}
        \begin{Ventry}{x}

          \item[x]


The value from where the new estimate should be calculated. This can
of course be the result of a previous iteration of the algorithm.
        \end{Ventry}

      \end{quote}

    \vspace{1ex}

      \textbf{Return Value}
      \begin{quote}

This method returns a tuple \texttt{(x, e)}, where \texttt{x} is the updated
estimate of the minimum, and \texttt{e} is the estimated error.
      \end{quote}

    \vspace{1ex}

      Overrides: peach.optm.optm.Optimizer.step

    \end{boxedminipage}

    \vspace{0.5ex}

    \begin{boxedminipage}{\textwidth}

    \raggedright \textbf{\_\_call\_\_}(\textit{self}, \textit{x})

    \vspace{-1.5ex}

    \rule{\textwidth}{0.5\fboxrule}

Transparently executes the search until the minimum is found. The stop
criteria are the maximum error or the maximum number of iterations,
whichever is reached first. Note that this is a \texttt{{\_}{\_}call{\_}{\_}} method, so
the object is called as a function. This method returns a tuple
\texttt{(x, e)}, with the best estimate of the minimum and the error.
    \vspace{1ex}

      \textbf{Parameters}
      \begin{quote}
        \begin{Ventry}{x}

          \item[x]


The initial triplet of values from where the search must start.
        \end{Ventry}

      \end{quote}

    \vspace{1ex}

      \textbf{Return Value}
      \begin{quote}

This method returns a tuple \texttt{(x, e)}, where \texttt{x} is the best
estimate of the minimum, and \texttt{e} is the estimated error.
      \end{quote}

    \vspace{1ex}

      Overrides: peach.optm.optm.Optimizer.\_\_call\_\_

    \end{boxedminipage}

    \label{object:__delattr__}
    \index{object.\_\_delattr\_\_ \textit{(function)}}

    \vspace{0.5ex}

    \begin{boxedminipage}{\textwidth}

    \raggedright \textbf{\_\_delattr\_\_}(\textit{...})

    \vspace{-1.5ex}

    \rule{\textwidth}{0.5\fboxrule}

x.{\_}{\_}delattr{\_}{\_}('name') {\textless}=={\textgreater} del x.name
    \vspace{1ex}

    \end{boxedminipage}

    \label{object:__getattribute__}
    \index{object.\_\_getattribute\_\_ \textit{(function)}}

    \vspace{0.5ex}

    \begin{boxedminipage}{\textwidth}

    \raggedright \textbf{\_\_getattribute\_\_}(\textit{...})

    \vspace{-1.5ex}

    \rule{\textwidth}{0.5\fboxrule}

x.{\_}{\_}getattribute{\_}{\_}('name') {\textless}=={\textgreater} x.name
    \vspace{1ex}

    \end{boxedminipage}

    \label{object:__hash__}
    \index{object.\_\_hash\_\_ \textit{(function)}}

    \vspace{0.5ex}

    \begin{boxedminipage}{\textwidth}

    \raggedright \textbf{\_\_hash\_\_}(\textit{x})

    \vspace{-1.5ex}

    \rule{\textwidth}{0.5\fboxrule}

hash(x)
    \vspace{1ex}

    \end{boxedminipage}

    \label{object:__new__}
    \index{object.\_\_new\_\_ \textit{(function)}}

    \vspace{0.5ex}

    \begin{boxedminipage}{\textwidth}

    \raggedright \textbf{\_\_new\_\_}(\textit{T}, \textit{S}, \textit{...})

      \textbf{Return Value}
      \begin{quote}
\begin{alltt}
a new object with type S, a subtype of T
\end{alltt}

      \end{quote}

    \vspace{1ex}

    \end{boxedminipage}

    \label{object:__reduce__}
    \index{object.\_\_reduce\_\_ \textit{(function)}}

    \vspace{0.5ex}

    \begin{boxedminipage}{\textwidth}

    \raggedright \textbf{\_\_reduce\_\_}(\textit{...})

    \vspace{-1.5ex}

    \rule{\textwidth}{0.5\fboxrule}

helper for pickle
    \vspace{1ex}

    \end{boxedminipage}

    \label{object:__reduce_ex__}
    \index{object.\_\_reduce\_ex\_\_ \textit{(function)}}

    \vspace{0.5ex}

    \begin{boxedminipage}{\textwidth}

    \raggedright \textbf{\_\_reduce\_ex\_\_}(\textit{...})

    \vspace{-1.5ex}

    \rule{\textwidth}{0.5\fboxrule}

helper for pickle
    \vspace{1ex}

    \end{boxedminipage}

    \label{object:__repr__}
    \index{object.\_\_repr\_\_ \textit{(function)}}

    \vspace{0.5ex}

    \begin{boxedminipage}{\textwidth}

    \raggedright \textbf{\_\_repr\_\_}(\textit{x})

    \vspace{-1.5ex}

    \rule{\textwidth}{0.5\fboxrule}

repr(x)
    \vspace{1ex}

    \end{boxedminipage}

    \label{object:__setattr__}
    \index{object.\_\_setattr\_\_ \textit{(function)}}

    \vspace{0.5ex}

    \begin{boxedminipage}{\textwidth}

    \raggedright \textbf{\_\_setattr\_\_}(\textit{...})

    \vspace{-1.5ex}

    \rule{\textwidth}{0.5\fboxrule}

x.{\_}{\_}setattr{\_}{\_}('name', value) {\textless}=={\textgreater} x.name = value
    \vspace{1ex}

    \end{boxedminipage}

    \label{object:__str__}
    \index{object.\_\_str\_\_ \textit{(function)}}

    \vspace{0.5ex}

    \begin{boxedminipage}{\textwidth}

    \raggedright \textbf{\_\_str\_\_}(\textit{x})

    \vspace{-1.5ex}

    \rule{\textwidth}{0.5\fboxrule}

str(x)
    \vspace{1ex}

    \end{boxedminipage}


%%%%%%%%%%%%%%%%%%%%%%%%%%%%%%%%%%%%%%%%%%%%%%%%%%%%%%%%%%%%%%%%%%%%%%%%%%%
%%                              Properties                               %%
%%%%%%%%%%%%%%%%%%%%%%%%%%%%%%%%%%%%%%%%%%%%%%%%%%%%%%%%%%%%%%%%%%%%%%%%%%%

  \subsubsection{Properties}

\begin{longtable}{|p{.30\textwidth}|p{.62\textwidth}|l}
\cline{1-2}
\cline{1-2} \centering \textbf{Name} & \centering \textbf{Description}& \\
\cline{1-2}
\endhead\cline{1-2}\multicolumn{3}{r}{\small\textit{continued on next page}}\\\endfoot\cline{1-2}
\endlastfoot\raggedright \_\-\_\-c\-l\-a\-s\-s\-\_\-\_\- & \raggedright \textbf{Value:} 
{\tt {\textless}attribute '\_\_class\_\_' of 'object' objects{\textgreater}}&\\
\cline{1-2}
\end{longtable}

    \index{peach \textit{(package)}!peach.optm \textit{(package)}!peach.optm.sa \textit{(module)}!peach.optm.sa.ContinuousSA \textit{(class)}|)}

%%%%%%%%%%%%%%%%%%%%%%%%%%%%%%%%%%%%%%%%%%%%%%%%%%%%%%%%%%%%%%%%%%%%%%%%%%%
%%                           Class Description                           %%
%%%%%%%%%%%%%%%%%%%%%%%%%%%%%%%%%%%%%%%%%%%%%%%%%%%%%%%%%%%%%%%%%%%%%%%%%%%

    \index{peach \textit{(package)}!peach.optm \textit{(package)}!peach.optm.sa \textit{(module)}!peach.optm.sa.DiscreteSA \textit{(class)}|(}
\subsection{Class DiscreteSA}

    \label{peach:optm:sa:DiscreteSA}
\begin{tabular}{cccccccc}
% Line for object, linespec=[False, False]
\multicolumn{2}{r}{\settowidth{\BCL}{object}\multirow{2}{\BCL}{object}}
&&
&&
  \\\cline{3-3}
  &&\multicolumn{1}{c|}{}
&&
&&
  \\
% Line for peach.optm.optm.Optimizer, linespec=[False]
\multicolumn{4}{r}{\settowidth{\BCL}{peach.optm.optm.Optimizer}\multirow{2}{\BCL}{peach.optm.optm.Optimizer}}
&&
  \\\cline{5-5}
  &&&&\multicolumn{1}{c|}{}
&&
  \\
&&&&\multicolumn{2}{l}{\textbf{peach.optm.sa.DiscreteSA}}
\end{tabular}


Simulated Annealing discrete optimization.

This is a simulated annealing optimizer implemented to work with vectors of
discrete variables, which can be floating point or integer numbers,
characters or anything allowed by the \texttt{struct} module of the Python
standard library. The neighborhood of an estimate is calculated by inverting
a number of bits randomly according to a given rate. Given the nature of
this implementation, no alternate convergence can be used in the case of
rejection of an estimate.

%%%%%%%%%%%%%%%%%%%%%%%%%%%%%%%%%%%%%%%%%%%%%%%%%%%%%%%%%%%%%%%%%%%%%%%%%%%
%%                                Methods                                %%
%%%%%%%%%%%%%%%%%%%%%%%%%%%%%%%%%%%%%%%%%%%%%%%%%%%%%%%%%%%%%%%%%%%%%%%%%%%

  \subsubsection{Methods}

    \vspace{0.5ex}

    \begin{boxedminipage}{\textwidth}

    \raggedright \textbf{\_\_init\_\_}(\textit{self}, \textit{f}, \textit{fmt}, \textit{ranges}=\texttt{\texttt{[}\texttt{]}}, \textit{T0}=\texttt{1000.0}, \textit{rt}=\texttt{0.95}, \textit{nb}=\texttt{1}, \textit{emax}=\texttt{1e-08}, \textit{imax}=\texttt{1000})

    \vspace{-1.5ex}

    \rule{\textwidth}{0.5\fboxrule}

Initializes the optimizer.

To create an optimizer of this type, instantiate the class with the
parameters given below:
    \vspace{1ex}

      \textbf{Parameters}
      \begin{quote}
        \begin{Ventry}{xxxxxx}

          \item[f]


A multivariable function to be optimized. The function should have
only one parameter, a multidimensional line-vector, and return the
function value, a scalar.
          \item[fmt]


A \texttt{struct}-module string with the format of the data used. Please,
consult the \texttt{struct} documentation, since what is explained there
is exactly what is used here. For example, if you are going to use
the optimizer to deal with three-dimensional vectors of continuous
variables, the format would be something like:
\begin{quote}{\ttfamily \raggedright \noindent
fmt~=~'fff'
}\end{quote}

Default value is an empty string. Notice that this is implemented as
a \texttt{bitarray}, so this module must be present.

It is strongly recommended that integer numbers are used! Floating
point numbers can be simulated with long integers. The reason for
this is that random bit sequences can have no representation as
floating point numbers, and that can make the algorithm not perform
adequatelly.
          \item[ranges]


Ranges of values allowed for each component of the input vector. If
given, ranges are checked and a new estimate is generated in case
any of the components fall beyond the value. \texttt{range} can be a
tuple containing the inferior and superior limits of the interval;
in that case, the same range is used for every variable in the input
vector. \texttt{range} can also be a list of tuples of the same format,
inferior and superior limits; in that case, the first tuple is
assumed as the range allowed for the first variable, the second
tuple is assumed as the range allowed for the second variable and so
on.
          \item[T0]


Initial temperature of the system. The temperature is, of course, an
analogy. Defaults to 1000.
          \item[rt]


Temperature decreasing rate. The temperature must slowly decrease in
simulated annealing algorithms. In this implementation, this is
controlled by this parameter. At each step, the temperature is
multiplied by this value, so it is necessary that \texttt{0 < rt < 1}.
Defaults to 0.95, smaller values make the temperature decay faster,
while larger values make the temperature decay slower.
          \item[nb]


The number of bits to be randomly choosen to be inverted in the
calculation of the neighbor. Be very careful while choosing this
parameter. While very large optimizations can benefit from a big
value here, it is not recommended that more than one bit per
variable is inverted at each step -{}- otherwise, the neighbor might
fall very far from the present estimate, which can make the
algorithm not work accordingly. This defaults to 1, that is, at each
step, only one bit will be inverted at most.
          \item[emax]


Maximum allowed error. The algorithm stops as soon as the error is
below this level. The error is absolute.
          \item[imax]


Maximum number of iterations, the algorithm stops as soon this
number of iterations are executed, no matter what the error is at
the moment.
        \end{Ventry}

      \end{quote}

    \vspace{1ex}

      Overrides: peach.optm.optm.Optimizer.\_\_init\_\_

    \end{boxedminipage}

    \vspace{0.5ex}

    \begin{boxedminipage}{\textwidth}

    \raggedright \textbf{step}(\textit{self}, \textit{x})

    \vspace{-1.5ex}

    \rule{\textwidth}{0.5\fboxrule}

One step of the search.

In this method, a neighbor of the given estimate is obtained from the
present estimate by choosing \texttt{nb} bits and inverting them. It is
accepted as a new estimate if it performs better in the cost function
\emph{or} if the temperature is high enough. In case it is not accepted, the
previous estimate is mantained.
    \vspace{1ex}

      \textbf{Parameters}
      \begin{quote}
        \begin{Ventry}{x}

          \item[x]


The value from where the new estimate should be calculated. This can
of course be the result of a previous iteration of the algorithm.
        \end{Ventry}

      \end{quote}

    \vspace{1ex}

      \textbf{Return Value}
      \begin{quote}

This method returns a tuple \texttt{(x, e)}, where \texttt{x} is the updated
estimate of the minimum, and \texttt{e} is the estimated error.
      \end{quote}

    \vspace{1ex}

      Overrides: peach.optm.optm.Optimizer.step

    \end{boxedminipage}

    \vspace{0.5ex}

    \begin{boxedminipage}{\textwidth}

    \raggedright \textbf{\_\_call\_\_}(\textit{self}, \textit{x})

    \vspace{-1.5ex}

    \rule{\textwidth}{0.5\fboxrule}

Transparently executes the search until the minimum is found. The stop
criteria are the maximum error or the maximum number of iterations,
whichever is reached first. Note that this is a \texttt{{\_}{\_}call{\_}{\_}} method, so
the object is called as a function. This method returns a tuple
\texttt{(x, e)}, with the best estimate of the minimum and the error.
    \vspace{1ex}

      \textbf{Parameters}
      \begin{quote}
        \begin{Ventry}{x}

          \item[x]


The initial triplet of values from where the search must start.
        \end{Ventry}

      \end{quote}

    \vspace{1ex}

      \textbf{Return Value}
      \begin{quote}

This method returns a tuple \texttt{(x, e)}, where \texttt{x} is the best
estimate of the minimum, and \texttt{e} is the estimated error.
      \end{quote}

    \vspace{1ex}

      Overrides: peach.optm.optm.Optimizer.\_\_call\_\_

    \end{boxedminipage}

    \label{object:__delattr__}
    \index{object.\_\_delattr\_\_ \textit{(function)}}

    \vspace{0.5ex}

    \begin{boxedminipage}{\textwidth}

    \raggedright \textbf{\_\_delattr\_\_}(\textit{...})

    \vspace{-1.5ex}

    \rule{\textwidth}{0.5\fboxrule}

x.{\_}{\_}delattr{\_}{\_}('name') {\textless}=={\textgreater} del x.name
    \vspace{1ex}

    \end{boxedminipage}

    \label{object:__getattribute__}
    \index{object.\_\_getattribute\_\_ \textit{(function)}}

    \vspace{0.5ex}

    \begin{boxedminipage}{\textwidth}

    \raggedright \textbf{\_\_getattribute\_\_}(\textit{...})

    \vspace{-1.5ex}

    \rule{\textwidth}{0.5\fboxrule}

x.{\_}{\_}getattribute{\_}{\_}('name') {\textless}=={\textgreater} x.name
    \vspace{1ex}

    \end{boxedminipage}

    \label{object:__hash__}
    \index{object.\_\_hash\_\_ \textit{(function)}}

    \vspace{0.5ex}

    \begin{boxedminipage}{\textwidth}

    \raggedright \textbf{\_\_hash\_\_}(\textit{x})

    \vspace{-1.5ex}

    \rule{\textwidth}{0.5\fboxrule}

hash(x)
    \vspace{1ex}

    \end{boxedminipage}

    \label{object:__new__}
    \index{object.\_\_new\_\_ \textit{(function)}}

    \vspace{0.5ex}

    \begin{boxedminipage}{\textwidth}

    \raggedright \textbf{\_\_new\_\_}(\textit{T}, \textit{S}, \textit{...})

      \textbf{Return Value}
      \begin{quote}
\begin{alltt}
a new object with type S, a subtype of T
\end{alltt}

      \end{quote}

    \vspace{1ex}

    \end{boxedminipage}

    \label{object:__reduce__}
    \index{object.\_\_reduce\_\_ \textit{(function)}}

    \vspace{0.5ex}

    \begin{boxedminipage}{\textwidth}

    \raggedright \textbf{\_\_reduce\_\_}(\textit{...})

    \vspace{-1.5ex}

    \rule{\textwidth}{0.5\fboxrule}

helper for pickle
    \vspace{1ex}

    \end{boxedminipage}

    \label{object:__reduce_ex__}
    \index{object.\_\_reduce\_ex\_\_ \textit{(function)}}

    \vspace{0.5ex}

    \begin{boxedminipage}{\textwidth}

    \raggedright \textbf{\_\_reduce\_ex\_\_}(\textit{...})

    \vspace{-1.5ex}

    \rule{\textwidth}{0.5\fboxrule}

helper for pickle
    \vspace{1ex}

    \end{boxedminipage}

    \label{object:__repr__}
    \index{object.\_\_repr\_\_ \textit{(function)}}

    \vspace{0.5ex}

    \begin{boxedminipage}{\textwidth}

    \raggedright \textbf{\_\_repr\_\_}(\textit{x})

    \vspace{-1.5ex}

    \rule{\textwidth}{0.5\fboxrule}

repr(x)
    \vspace{1ex}

    \end{boxedminipage}

    \label{object:__setattr__}
    \index{object.\_\_setattr\_\_ \textit{(function)}}

    \vspace{0.5ex}

    \begin{boxedminipage}{\textwidth}

    \raggedright \textbf{\_\_setattr\_\_}(\textit{...})

    \vspace{-1.5ex}

    \rule{\textwidth}{0.5\fboxrule}

x.{\_}{\_}setattr{\_}{\_}('name', value) {\textless}=={\textgreater} x.name = value
    \vspace{1ex}

    \end{boxedminipage}

    \label{object:__str__}
    \index{object.\_\_str\_\_ \textit{(function)}}

    \vspace{0.5ex}

    \begin{boxedminipage}{\textwidth}

    \raggedright \textbf{\_\_str\_\_}(\textit{x})

    \vspace{-1.5ex}

    \rule{\textwidth}{0.5\fboxrule}

str(x)
    \vspace{1ex}

    \end{boxedminipage}


%%%%%%%%%%%%%%%%%%%%%%%%%%%%%%%%%%%%%%%%%%%%%%%%%%%%%%%%%%%%%%%%%%%%%%%%%%%
%%                              Properties                               %%
%%%%%%%%%%%%%%%%%%%%%%%%%%%%%%%%%%%%%%%%%%%%%%%%%%%%%%%%%%%%%%%%%%%%%%%%%%%

  \subsubsection{Properties}

\begin{longtable}{|p{.30\textwidth}|p{.62\textwidth}|l}
\cline{1-2}
\cline{1-2} \centering \textbf{Name} & \centering \textbf{Description}& \\
\cline{1-2}
\endhead\cline{1-2}\multicolumn{3}{r}{\small\textit{continued on next page}}\\\endfoot\cline{1-2}
\endlastfoot\raggedright \_\-\_\-c\-l\-a\-s\-s\-\_\-\_\- & \raggedright \textbf{Value:} 
{\tt {\textless}attribute '\_\_class\_\_' of 'object' objects{\textgreater}}&\\
\cline{1-2}
\end{longtable}

    \index{peach \textit{(package)}!peach.optm \textit{(package)}!peach.optm.sa \textit{(module)}!peach.optm.sa.DiscreteSA \textit{(class)}|)}
    \index{peach \textit{(package)}!peach.optm \textit{(package)}!peach.optm.sa \textit{(module)}|)}
