%
% API Documentation for Peach - Computational Intelligence for Python
% Package peach.optm
%
% Generated by epydoc 3.0.1
% [Thu Jul 28 16:37:49 2011]
%

%%%%%%%%%%%%%%%%%%%%%%%%%%%%%%%%%%%%%%%%%%%%%%%%%%%%%%%%%%%%%%%%%%%%%%%%%%%
%%                          Module Description                           %%
%%%%%%%%%%%%%%%%%%%%%%%%%%%%%%%%%%%%%%%%%%%%%%%%%%%%%%%%%%%%%%%%%%%%%%%%%%%

    \index{peach \textit{(package)}!peach.optm \textit{(package)}|(}
\section{Package peach.optm}

    \label{peach:optm}

This package implements deterministic optimization methods. Consult:
%
\begin{quote}
%
\begin{description}
\item[{base}] \leavevmode 
Basic definitions and interface with the optimization methods;

\item[{linear}] \leavevmode 
Basic methods for one variable optimization;

\item[{multivar}] \leavevmode 
Gradient, Newton and othe multivariable optimization methods;

\item[{quasinewton}] \leavevmode 
Quasi-Newton methods;

\end{description}

\end{quote}

Every optimizer works in pretty much the same way. Instantiate the respective
class, using as parameter the cost function to be optimized, the first estimate
(a scalar in case of a single variable optimization, and a one-dimensional array
in case of multivariable optimization) and some other parameters. Use \texttt{step()}
to perform one iteration of the method, use the \texttt{\_\_call\_\_()} method to perform
the search until the stop conditions are met. See each method for details.

%%%%%%%%%%%%%%%%%%%%%%%%%%%%%%%%%%%%%%%%%%%%%%%%%%%%%%%%%%%%%%%%%%%%%%%%%%%
%%                                Modules                                %%
%%%%%%%%%%%%%%%%%%%%%%%%%%%%%%%%%%%%%%%%%%%%%%%%%%%%%%%%%%%%%%%%%%%%%%%%%%%

\subsection{Modules}

\begin{itemize}
\setlength{\parskip}{0ex}
\item \textbf{base}: 
Basic definitons and base class for optimizers


  \textit{(Section \ref{peach:optm:base}, p.~\pageref{peach:optm:base})}

\item \textbf{linear}: 
This package implements basic one variable only optimizers.


  \textit{(Section \ref{peach:optm:linear}, p.~\pageref{peach:optm:linear})}

\item \textbf{multivar}: 
This package implements basic multivariable optimizers, including gradient and
Newton searches.


  \textit{(Section \ref{peach:optm:multivar}, p.~\pageref{peach:optm:multivar})}

\item \textbf{quasinewton}: 
This package implements basic quasi-Newton optimizers. Newton optimizer is very
efficient, except that inverse matrices need to be calculated at each
convergence step. These methods try to estimate the hessian inverse iteratively,
thus increasing performance.


  \textit{(Section \ref{peach:optm:quasinewton}, p.~\pageref{peach:optm:quasinewton})}

\item \textbf{stochastic}
  \textit{(Section \ref{peach:optm:stochastic}, p.~\pageref{peach:optm:stochastic})}

\end{itemize}

    \index{peach \textit{(package)}!peach.optm \textit{(package)}|)}
