%
% API Documentation for Peach - Computational Intelligence for Python
% Package peach.optm
%
% Generated by epydoc 3.0beta1
% [Mon Dec 21 08:51:37 2009]
%

%%%%%%%%%%%%%%%%%%%%%%%%%%%%%%%%%%%%%%%%%%%%%%%%%%%%%%%%%%%%%%%%%%%%%%%%%%%
%%                          Module Description                           %%
%%%%%%%%%%%%%%%%%%%%%%%%%%%%%%%%%%%%%%%%%%%%%%%%%%%%%%%%%%%%%%%%%%%%%%%%%%%

    \index{peach \textit{(package)}!peach.optm \textit{(package)}|(}
\section{Package peach.optm}

    \label{peach:optm}

This package implements deterministic optimization methods. Consult:
\begin{quote}
\begin{description}
%[visit_definition_list_item]
\item[{optm}] %[visit_definition]

Basic definitions and interface with the optimization methods;

%[depart_definition]
%[depart_definition_list_item]
%[visit_definition_list_item]
\item[{linear}] %[visit_definition]

Basic methods for one variable optimization;

%[depart_definition]
%[depart_definition_list_item]
%[visit_definition_list_item]
\item[{multivar}] %[visit_definition]

Gradient, Newton and othe multivariable optimization methods;

%[depart_definition]
%[depart_definition_list_item]
%[visit_definition_list_item]
\item[{quasinewton}] %[visit_definition]

Quasi-Newton methods;

%[depart_definition]
%[depart_definition_list_item]
%[visit_definition_list_item]
\item[{stochastic}] %[visit_definition]

General stochastic methods;

%[depart_definition]
%[depart_definition_list_item]
%[visit_definition_list_item]
\item[{sa}] %[visit_definition]

Simulated Annealing methods;

%[depart_definition]
%[depart_definition_list_item]
\end{description}
\end{quote}

Every optimizer works in pretty much the same way. Instantiate the respective
class, using as parameter the cost function to be optimized and some other
parameters. Use \texttt{step()} to perform one iteration of the method, use the
\texttt{{\_}{\_}call{\_}{\_}()} method to perform the search until the stop conditions are met.
See each method for details.

%%%%%%%%%%%%%%%%%%%%%%%%%%%%%%%%%%%%%%%%%%%%%%%%%%%%%%%%%%%%%%%%%%%%%%%%%%%
%%                                Modules                                %%
%%%%%%%%%%%%%%%%%%%%%%%%%%%%%%%%%%%%%%%%%%%%%%%%%%%%%%%%%%%%%%%%%%%%%%%%%%%

\subsection{Modules}

\begin{itemize}
\setlength{\parskip}{0ex}
\item \textbf{linear}: 
This package implements basic one variable only optimizers.


  \textit{(Section \ref{peach:optm:linear}, p.~\pageref{peach:optm:linear})}

\item \textbf{multivar}: 
This package implements basic multivariable optimizers, including gradient and
Newton searches.


  \textit{(Section \ref{peach:optm:multivar}, p.~\pageref{peach:optm:multivar})}

\item \textbf{optm}: 
Basic definitons and base class for optimizers


  \textit{(Section \ref{peach:optm:optm}, p.~\pageref{peach:optm:optm})}

\item \textbf{quasinewton}: 
This package implements basic quasi-Newton optimizers.


  \textit{(Section \ref{peach:optm:quasinewton}, p.~\pageref{peach:optm:quasinewton})}

\item \textbf{sa}: 
This package implements two versions of simulated annealing optimization.


  \textit{(Section \ref{peach:optm:sa}, p.~\pageref{peach:optm:sa})}

\item \textbf{stochastic}: 
General methods of stochastic optimization.


  \textit{(Section \ref{peach:optm:stochastic}, p.~\pageref{peach:optm:stochastic})}

\end{itemize}


%%%%%%%%%%%%%%%%%%%%%%%%%%%%%%%%%%%%%%%%%%%%%%%%%%%%%%%%%%%%%%%%%%%%%%%%%%%
%%                               Variables                               %%
%%%%%%%%%%%%%%%%%%%%%%%%%%%%%%%%%%%%%%%%%%%%%%%%%%%%%%%%%%%%%%%%%%%%%%%%%%%

  \subsection{Variables}

\begin{longtable}{|p{.30\textwidth}|p{.62\textwidth}|l}
\cline{1-2}
\cline{1-2} \centering \textbf{Name} & \centering \textbf{Description}& \\
\cline{1-2}
\endhead\cline{1-2}\multicolumn{3}{r}{\small\textit{continued on next page}}\\\endfoot\cline{1-2}
\endlastfoot\raggedright \_\-\_\-d\-o\-c\-\_\-\_\- & \raggedright \textbf{Value:} 
{\tt \texttt{...}}&\\
\cline{1-2}
\end{longtable}

    \index{peach \textit{(package)}!peach.optm \textit{(package)}|)}
