%
% API Documentation for Peach - Computational Intelligence for Python
% Package peach.sa
%
% Generated by epydoc 3.0.1
% [Mon Jan 24 15:39:53 2011]
%

%%%%%%%%%%%%%%%%%%%%%%%%%%%%%%%%%%%%%%%%%%%%%%%%%%%%%%%%%%%%%%%%%%%%%%%%%%%
%%                          Module Description                           %%
%%%%%%%%%%%%%%%%%%%%%%%%%%%%%%%%%%%%%%%%%%%%%%%%%%%%%%%%%%%%%%%%%%%%%%%%%%%

    \index{peach \textit{(package)}!peach.sa \textit{(package)}|(}
\section{Package peach.sa}

    \label{peach:sa}

This package implements optimization by simulated annealing. Consult:
%
\begin{quote}
%
\begin{description}
\item[{base}] \leavevmode 
Implementation of the basic simulated annealing algorithms;

\item[{neighbor}] \leavevmode 
Some methods for determining the neighbor of the present estimate;

\end{description}

\end{quote}

Simulated Annealing is a meta-heuristic designed for optimization of functions.
It tries to mimic the way that atoms settle in crystal structures of metals. By
slowly cooling the metal, atoms settle in a position of low energy -{}- thus, it
is a natural optimization method.

Two kinds of optimizer are implemented here. The continuous version of the
algorithm can be used for optimization of continuous objective functions; the
discrete (or binary) one, can be used in combinatorial optimization problems.

%%%%%%%%%%%%%%%%%%%%%%%%%%%%%%%%%%%%%%%%%%%%%%%%%%%%%%%%%%%%%%%%%%%%%%%%%%%
%%                                Modules                                %%
%%%%%%%%%%%%%%%%%%%%%%%%%%%%%%%%%%%%%%%%%%%%%%%%%%%%%%%%%%%%%%%%%%%%%%%%%%%

\subsection{Modules}

\begin{itemize}
\setlength{\parskip}{0ex}
\item \textbf{base}: 
This package implements two versions of simulated annealing optimization. One
works with numeric data, and the other with a codified bit string. This last
method can be used in discrete optimization problems.


  \textit{(Section \ref{peach:sa:base}, p.~\pageref{peach:sa:base})}

\item \textbf{neighbor}: 
This module implements a general class to compute neighbors for continuous and
binary simulated annealing algorithms. The continuous neighbor functions return
an array with a neighbor of a given estimate; the binary neighbor functions
return a \texttt{bitarray} object.


  \textit{(Section \ref{peach:sa:neighbor}, p.~\pageref{peach:sa:neighbor})}

\end{itemize}

    \index{peach \textit{(package)}!peach.sa \textit{(package)}|)}
