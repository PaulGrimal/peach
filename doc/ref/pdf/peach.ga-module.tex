%
% API Documentation for Peach - Computational Intelligence for Python
% Package peach.ga
%
% Generated by epydoc 3.0beta1
% [Mon Dec 21 08:51:36 2009]
%

%%%%%%%%%%%%%%%%%%%%%%%%%%%%%%%%%%%%%%%%%%%%%%%%%%%%%%%%%%%%%%%%%%%%%%%%%%%
%%                          Module Description                           %%
%%%%%%%%%%%%%%%%%%%%%%%%%%%%%%%%%%%%%%%%%%%%%%%%%%%%%%%%%%%%%%%%%%%%%%%%%%%

    \index{peach \textit{(package)}!peach.ga \textit{(package)}|(}
\section{Package peach.ga}

    \label{peach:ga}

This package implements genetic algorithms. Consult:
\begin{quote}
\begin{description}
%[visit_definition_list_item]
\item[{chromosome}] %[visit_definition]

Basic definitions to work with chromosomes. Defined as arrays of bits;

%[depart_definition]
%[depart_definition_list_item]
%[visit_definition_list_item]
\item[{crossover}] %[visit_definition]

Defines crossover operators and base classes;

%[depart_definition]
%[depart_definition_list_item]
%[visit_definition_list_item]
\item[{fitness}] %[visit_definition]

Defines fitness functions and base classes;

%[depart_definition]
%[depart_definition_list_item]
%[visit_definition_list_item]
\item[{ga}] %[visit_definition]

Implementation of the basic genetic algorithm;

%[depart_definition]
%[depart_definition_list_item]
%[visit_definition_list_item]
\item[{mutation}] %[visit_definition]

Defines mutation operators and base classes;

%[depart_definition]
%[depart_definition_list_item]
%[visit_definition_list_item]
\item[{selection}] %[visit_definition]

Defines selection operators and base classes;

%[depart_definition]
%[depart_definition_list_item]
\end{description}
\end{quote}

%%%%%%%%%%%%%%%%%%%%%%%%%%%%%%%%%%%%%%%%%%%%%%%%%%%%%%%%%%%%%%%%%%%%%%%%%%%
%%                                Modules                                %%
%%%%%%%%%%%%%%%%%%%%%%%%%%%%%%%%%%%%%%%%%%%%%%%%%%%%%%%%%%%%%%%%%%%%%%%%%%%

\subsection{Modules}

\begin{itemize}
\setlength{\parskip}{0ex}
\item \textbf{chromosome}: 
Basic definitions and classes for manipulating chromosomes


  \textit{(Section \ref{peach:ga:chromosome}, p.~\pageref{peach:ga:chromosome})}

\item \textbf{crossover}: 
Basic definitions for crossover operations and base classes.


  \textit{(Section \ref{peach:ga:crossover}, p.~\pageref{peach:ga:crossover})}

\item \textbf{fitness}: 
Basic definitions and base classes for definition of fitness functions for use
with genetic algorithms.


  \textit{(Section \ref{peach:ga:fitness}, p.~\pageref{peach:ga:fitness})}

\item \textbf{ga}: 
Basic Genetic Algorithm (GA)


  \textit{(Section \ref{peach:ga:ga}, p.~\pageref{peach:ga:ga})}

\item \textbf{mutation}: 
Basic definitions and classes for operating mutation on chromosomes.


  \textit{(Section \ref{peach:ga:mutation}, p.~\pageref{peach:ga:mutation})}

\item \textbf{selection}: 
Basic classes and definitions for selection operator.


  \textit{(Section \ref{peach:ga:selection}, p.~\pageref{peach:ga:selection})}

\end{itemize}


%%%%%%%%%%%%%%%%%%%%%%%%%%%%%%%%%%%%%%%%%%%%%%%%%%%%%%%%%%%%%%%%%%%%%%%%%%%
%%                               Variables                               %%
%%%%%%%%%%%%%%%%%%%%%%%%%%%%%%%%%%%%%%%%%%%%%%%%%%%%%%%%%%%%%%%%%%%%%%%%%%%

  \subsection{Variables}

\begin{longtable}{|p{.30\textwidth}|p{.62\textwidth}|l}
\cline{1-2}
\cline{1-2} \centering \textbf{Name} & \centering \textbf{Description}& \\
\cline{1-2}
\endhead\cline{1-2}\multicolumn{3}{r}{\small\textit{continued on next page}}\\\endfoot\cline{1-2}
\endlastfoot\raggedright \_\-\_\-d\-o\-c\-\_\-\_\- & \raggedright \textbf{Value:} 
{\tt \texttt{...}}&\\
\cline{1-2}
\end{longtable}

    \index{peach \textit{(package)}!peach.ga \textit{(package)}|)}
