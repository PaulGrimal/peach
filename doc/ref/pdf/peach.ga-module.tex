%
% API Documentation for Peach - Computational Intelligence for Python
% Package peach.ga
%
% Generated by epydoc 3.0.1
% [Thu Jul 28 16:37:46 2011]
%

%%%%%%%%%%%%%%%%%%%%%%%%%%%%%%%%%%%%%%%%%%%%%%%%%%%%%%%%%%%%%%%%%%%%%%%%%%%
%%                          Module Description                           %%
%%%%%%%%%%%%%%%%%%%%%%%%%%%%%%%%%%%%%%%%%%%%%%%%%%%%%%%%%%%%%%%%%%%%%%%%%%%

    \index{peach \textit{(package)}!peach.ga \textit{(package)}|(}
\section{Package peach.ga}

    \label{peach:ga}

This package implements genetic algorithms. Consult:
%
\begin{quote}
%
\begin{description}
\item[{base}] \leavevmode 
Implementation of the basic genetic algorithm;

\item[{chromosome}] \leavevmode 
Basic definitions to work with chromosomes. Defined as arrays of bits;

\item[{crossover}] \leavevmode 
Defines crossover operators and base classes;

\item[{fitness}] \leavevmode 
Defines fitness functions and base classes;

\item[{mutation}] \leavevmode 
Defines mutation operators and base classes;

\item[{selection}] \leavevmode 
Defines selection operators and base classes;

\end{description}

\end{quote}

%%%%%%%%%%%%%%%%%%%%%%%%%%%%%%%%%%%%%%%%%%%%%%%%%%%%%%%%%%%%%%%%%%%%%%%%%%%
%%                                Modules                                %%
%%%%%%%%%%%%%%%%%%%%%%%%%%%%%%%%%%%%%%%%%%%%%%%%%%%%%%%%%%%%%%%%%%%%%%%%%%%

\subsection{Modules}

\begin{itemize}
\setlength{\parskip}{0ex}
\item \textbf{base}: 
Basic Genetic Algorithm (GA)


  \textit{(Section \ref{peach:ga:base}, p.~\pageref{peach:ga:base})}

\item \textbf{chromosome}: 
Basic definitions and classes for manipulating chromosomes


  \textit{(Section \ref{peach:ga:chromosome}, p.~\pageref{peach:ga:chromosome})}

\item \textbf{crossover}: 
Basic definitions for crossover operations and base classes.


  \textit{(Section \ref{peach:ga:crossover}, p.~\pageref{peach:ga:crossover})}

\item \textbf{fitness}: 
Basic definitions and base classes for definition of fitness functions for use
with genetic algorithms.


  \textit{(Section \ref{peach:ga:fitness}, p.~\pageref{peach:ga:fitness})}

\item \textbf{mutation}: 
Basic definitions and classes for operating mutation on chromosomes.


  \textit{(Section \ref{peach:ga:mutation}, p.~\pageref{peach:ga:mutation})}

\item \textbf{selection}: 
Basic classes and definitions for selection operator.


  \textit{(Section \ref{peach:ga:selection}, p.~\pageref{peach:ga:selection})}

\end{itemize}

    \index{peach \textit{(package)}!peach.ga \textit{(package)}|)}
