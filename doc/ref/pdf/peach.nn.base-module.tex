%
% API Documentation for Peach - Computational Intelligence for Python
% Module peach.nn.base
%
% Generated by epydoc 3.0.1
% [Thu Jul 28 16:37:48 2011]
%

%%%%%%%%%%%%%%%%%%%%%%%%%%%%%%%%%%%%%%%%%%%%%%%%%%%%%%%%%%%%%%%%%%%%%%%%%%%
%%                          Module Description                           %%
%%%%%%%%%%%%%%%%%%%%%%%%%%%%%%%%%%%%%%%%%%%%%%%%%%%%%%%%%%%%%%%%%%%%%%%%%%%

    \index{peach \textit{(package)}!peach.nn \textit{(package)}!peach.nn.base \textit{(module)}|(}
\section{Module peach.nn.base}

    \label{peach:nn:base}

Basic definitions for layers of neurons.

This subpackage implements the basic classes used with neural networks. A neural
network is basically implemented as a layer of neurons. To speed things up, a
layer is implemented as a array, where each line represents the weight vector
of a neuron. Further definitions and algorithms are based on this definition.

%%%%%%%%%%%%%%%%%%%%%%%%%%%%%%%%%%%%%%%%%%%%%%%%%%%%%%%%%%%%%%%%%%%%%%%%%%%
%%                               Variables                               %%
%%%%%%%%%%%%%%%%%%%%%%%%%%%%%%%%%%%%%%%%%%%%%%%%%%%%%%%%%%%%%%%%%%%%%%%%%%%

  \subsection{Variables}

    \vspace{-1cm}
\hspace{\varindent}\begin{longtable}{|p{\varnamewidth}|p{\vardescrwidth}|l}
\cline{1-2}
\cline{1-2} \centering \textbf{Name} & \centering \textbf{Description}& \\
\cline{1-2}
\endhead\cline{1-2}\multicolumn{3}{r}{\small\textit{continued on next page}}\\\endfoot\cline{1-2}
\endlastfoot\raggedright \_\-\_\-d\-o\-c\-\_\-\_\- & \raggedright \textbf{Value:} 
{\tt \texttt{...}}&\\
\cline{1-2}
\raggedright \_\-\_\-p\-a\-c\-k\-a\-g\-e\-\_\-\_\- & \raggedright \textbf{Value:} 
{\tt \texttt{'}\texttt{peach.nn}\texttt{'}}&\\
\cline{1-2}
\end{longtable}


%%%%%%%%%%%%%%%%%%%%%%%%%%%%%%%%%%%%%%%%%%%%%%%%%%%%%%%%%%%%%%%%%%%%%%%%%%%
%%                           Class Description                           %%
%%%%%%%%%%%%%%%%%%%%%%%%%%%%%%%%%%%%%%%%%%%%%%%%%%%%%%%%%%%%%%%%%%%%%%%%%%%

    \index{peach \textit{(package)}!peach.nn \textit{(package)}!peach.nn.base \textit{(module)}!peach.nn.base.Layer \textit{(class)}|(}
\subsection{Class Layer}

    \label{peach:nn:base:Layer}
\begin{tabular}{cccccc}
% Line for object, linespec=[False]
\multicolumn{2}{r}{\settowidth{\BCL}{object}\multirow{2}{\BCL}{object}}
&&
  \\\cline{3-3}
  &&\multicolumn{1}{c|}{}
&&
  \\
&&\multicolumn{2}{l}{\textbf{peach.nn.base.Layer}}
\end{tabular}

\textbf{Known Subclasses:}
peach.nn.nnet.SOM,
    peach.nn.mem.Hopfield


Base class for neural networks.

This class implements a layer of neurons. It is represented by a array of
real values. Each line of the array represents the weight vector of a
single neuron. If the neurons on the layer are biased, then the first
element of the weight vector is the bias weight, and the bias input is
always valued 1. Also, to each layer is associated an activation function,
that determines if the neuron is fired or not. Please, consult the module
\texttt{af} to see more about activation functions.

In general, this class shoulb be subclassed if you want to use neural nets.
But, as neural nets are very different one from the other, check carefully
the documentation to see if the attributes, properties and methods are
suited to your task.

%%%%%%%%%%%%%%%%%%%%%%%%%%%%%%%%%%%%%%%%%%%%%%%%%%%%%%%%%%%%%%%%%%%%%%%%%%%
%%                                Methods                                %%
%%%%%%%%%%%%%%%%%%%%%%%%%%%%%%%%%%%%%%%%%%%%%%%%%%%%%%%%%%%%%%%%%%%%%%%%%%%

  \subsubsection{Methods}

    \label{peach:nn:base:Layer:__call__}
    \index{peach \textit{(package)}!peach.nn \textit{(package)}!peach.nn.base \textit{(module)}!peach.nn.base.Layer \textit{(class)}!peach.nn.base.Layer.\_\_call\_\_ \textit{(method)}}

    \vspace{0.5ex}

\hspace{.8\funcindent}\begin{boxedminipage}{\funcwidth}

    \raggedright \textbf{\_\_call\_\_}(\textit{self}, \textit{x})

    \vspace{-1.5ex}

    \rule{\textwidth}{0.5\fboxrule}
\setlength{\parskip}{2ex}

The feedforward method to the layer.

The \texttt{\_\_call\_\_} interface should be called if the answer of the neuron
to a given input vector \texttt{x} is desired. \emph{This method has collateral
effects}, so beware. After the calling of this method, the \texttt{v} and
\texttt{y} properties are set with the activation potential and the answer of
the neurons, respectivelly.
\setlength{\parskip}{1ex}
      \textbf{Parameters}
      \vspace{-1ex}

      \begin{quote}
        \begin{Ventry}{x}

          \item[x]


The input vector to the layer.
        \end{Ventry}

      \end{quote}

      \textbf{Return Value}
    \vspace{-1ex}

      \begin{quote}

The vector containing the answer of every neuron in the layer, in the
respective order.
      \end{quote}

    \end{boxedminipage}

    \label{peach:nn:base:Layer:__getitem__}
    \index{peach \textit{(package)}!peach.nn \textit{(package)}!peach.nn.base \textit{(module)}!peach.nn.base.Layer \textit{(class)}!peach.nn.base.Layer.\_\_getitem\_\_ \textit{(method)}}

    \vspace{0.5ex}

\hspace{.8\funcindent}\begin{boxedminipage}{\funcwidth}

    \raggedright \textbf{\_\_getitem\_\_}(\textit{self}, \textit{n})

    \vspace{-1.5ex}

    \rule{\textwidth}{0.5\fboxrule}
\setlength{\parskip}{2ex}

The \texttt{{[} {]}} get interface.

The input to this method is forwarded to the \texttt{weights} property. That
means that it will return the respective line/element of the weight
array.
\setlength{\parskip}{1ex}
      \textbf{Parameters}
      \vspace{-1ex}

      \begin{quote}
        \begin{Ventry}{x}

          \item[n]


A slice object containing the elements referenced. Since it is
forwarded to an array, it behaves exactly as one.
        \end{Ventry}

      \end{quote}

      \textbf{Return Value}
    \vspace{-1ex}

      \begin{quote}

The element or elements in the referenced indices.
      \end{quote}

    \end{boxedminipage}

    \vspace{0.5ex}

\hspace{.8\funcindent}\begin{boxedminipage}{\funcwidth}

    \raggedright \textbf{\_\_init\_\_}(\textit{self}, \textit{shape}, \textit{phi}={\tt {\textless}class 'peach.nn.af.Linear'{\textgreater}}, \textit{bias}={\tt False})

    \vspace{-1.5ex}

    \rule{\textwidth}{0.5\fboxrule}
\setlength{\parskip}{2ex}

Initializes the layer.

A layer is represented by a array where each line is the weight vector
of a single neuron. The first element of the vector is the bias weight,
in case the neuron is biased. Associated with the layer is an activation
function defined in an appropriate way.
\setlength{\parskip}{1ex}
      \textbf{Parameters}
      \vspace{-1ex}

      \begin{quote}
        \begin{Ventry}{xxxxx}

          \item[shape]


Stablishes the size of the layer. It must be a two-tuple of the
format \texttt{(m, n)}, where \texttt{m} is the number of neurons in the
layer, and \texttt{n} is the number of inputs of each neuron. The neurons
in the layer all have the same number of inputs.
          \item[phi]


The activation function. It can be an \texttt{Activation} object (please,
consult the \texttt{af} module) or a standard Python function. In this
case, it must receive a single real value and return a single real
value which determines if the neuron is activated or not. Defaults
to \texttt{Linear}.
          \item[bias]


If \texttt{True}, then the neurons on the layer are biased. That means
that an additional weight is added to each neuron to represent the
bias. If \texttt{False}, no modification is made.
        \end{Ventry}

      \end{quote}

      Overrides: object.\_\_init\_\_

    \end{boxedminipage}

    \label{peach:nn:base:Layer:__setitem__}
    \index{peach \textit{(package)}!peach.nn \textit{(package)}!peach.nn.base \textit{(module)}!peach.nn.base.Layer \textit{(class)}!peach.nn.base.Layer.\_\_setitem\_\_ \textit{(method)}}

    \vspace{0.5ex}

\hspace{.8\funcindent}\begin{boxedminipage}{\funcwidth}

    \raggedright \textbf{\_\_setitem\_\_}(\textit{self}, \textit{n}, \textit{w})

    \vspace{-1.5ex}

    \rule{\textwidth}{0.5\fboxrule}
\setlength{\parskip}{2ex}

The \texttt{{[} {]}} set interface.

The inputs to this method are forwarded to the \texttt{weights} property.
That means that it will set the respective line/element of the weight
array.
\setlength{\parskip}{1ex}
      \textbf{Parameters}
      \vspace{-1ex}

      \begin{quote}
        \begin{Ventry}{x}

          \item[n]


A slice object containing the elements referenced. Since it is
forwarded to an array, it behaves exactly as one.
          \item[w]


A value or array of values to be set in the given indices.
        \end{Ventry}

      \end{quote}

    \end{boxedminipage}


\large{\textbf{\textit{Inherited from object}}}

\begin{quote}
\_\_delattr\_\_(), \_\_format\_\_(), \_\_getattribute\_\_(), \_\_hash\_\_(), \_\_new\_\_(), \_\_reduce\_\_(), \_\_reduce\_ex\_\_(), \_\_repr\_\_(), \_\_setattr\_\_(), \_\_sizeof\_\_(), \_\_str\_\_(), \_\_subclasshook\_\_()
\end{quote}

%%%%%%%%%%%%%%%%%%%%%%%%%%%%%%%%%%%%%%%%%%%%%%%%%%%%%%%%%%%%%%%%%%%%%%%%%%%
%%                              Properties                               %%
%%%%%%%%%%%%%%%%%%%%%%%%%%%%%%%%%%%%%%%%%%%%%%%%%%%%%%%%%%%%%%%%%%%%%%%%%%%

  \subsubsection{Properties}

    \vspace{-1cm}
\hspace{\varindent}\begin{longtable}{|p{\varnamewidth}|p{\vardescrwidth}|l}
\cline{1-2}
\cline{1-2} \centering \textbf{Name} & \centering \textbf{Description}& \\
\cline{1-2}
\endhead\cline{1-2}\multicolumn{3}{r}{\small\textit{continued on next page}}\\\endfoot\cline{1-2}
\endlastfoot\raggedright b\-i\-a\-s\- & &\\
\cline{1-2}
\raggedright i\-n\-p\-u\-t\-s\- & &\\
\cline{1-2}
\raggedright p\-h\-i\- & &\\
\cline{1-2}
\raggedright s\-h\-a\-p\-e\- & &\\
\cline{1-2}
\raggedright s\-i\-z\-e\- & &\\
\cline{1-2}
\raggedright v\- & &\\
\cline{1-2}
\raggedright w\-e\-i\-g\-h\-t\-s\- & &\\
\cline{1-2}
\raggedright y\- & &\\
\cline{1-2}
\multicolumn{2}{|l|}{\textit{Inherited from object}}\\
\multicolumn{2}{|p{\varwidth}|}{\raggedright \_\_class\_\_}\\
\cline{1-2}
\end{longtable}

    \index{peach \textit{(package)}!peach.nn \textit{(package)}!peach.nn.base \textit{(module)}!peach.nn.base.Layer \textit{(class)}|)}
    \index{peach \textit{(package)}!peach.nn \textit{(package)}!peach.nn.base \textit{(module)}|)}
