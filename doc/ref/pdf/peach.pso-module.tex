%
% API Documentation for Peach - Computational Intelligence for Python
% Package peach.pso
%
% Generated by epydoc 3.0.1
% [Sun Jul 31 17:00:42 2011]
%

%%%%%%%%%%%%%%%%%%%%%%%%%%%%%%%%%%%%%%%%%%%%%%%%%%%%%%%%%%%%%%%%%%%%%%%%%%%
%%                          Module Description                           %%
%%%%%%%%%%%%%%%%%%%%%%%%%%%%%%%%%%%%%%%%%%%%%%%%%%%%%%%%%%%%%%%%%%%%%%%%%%%

    \index{peach \textit{(package)}!peach.pso \textit{(package)}|(}
\section{Package peach.pso}

    \label{peach:pso}

Basic Particle Swarm Optimization (PSO)

This sub-package implements traditional particle swarm optimizers as described
in literature. It consists of a very simple algorithm emulating the behaviour
of a flock of birds (though in a very simplified way). A population of particles
is created, each particle with its corresponding velocity. They fly towards the
particle local best and the swarm global best, thus exploring the whole domain.

For consistency purposes, the particles are represented internally as a list of
vectors. The particles can be acessed externally by using the \texttt{{[} {]}} interface.
See the rest of the documentation for more information.

%%%%%%%%%%%%%%%%%%%%%%%%%%%%%%%%%%%%%%%%%%%%%%%%%%%%%%%%%%%%%%%%%%%%%%%%%%%
%%                                Modules                                %%
%%%%%%%%%%%%%%%%%%%%%%%%%%%%%%%%%%%%%%%%%%%%%%%%%%%%%%%%%%%%%%%%%%%%%%%%%%%

\subsection{Modules}

\begin{itemize}
\setlength{\parskip}{0ex}
\item \textbf{acc}: 
Functions to update the velocity (ie, accelerate) of the particles in a swarm.


  \textit{(Section \ref{peach:pso:acc}, p.~\pageref{peach:pso:acc})}

\item \textbf{base}: 
This package implements the simple continuous version of the particle swarm
optimizer. In this implementation, it is possible to specify, besides the
objective function and the first estimates, the ranges of search, which will
influence the max velocity of the particles, and the population size. Other
parameters are available too, please refer to the rest of this documentation for
further details.


  \textit{(Section \ref{peach:pso:base}, p.~\pageref{peach:pso:base})}

\end{itemize}

    \index{peach \textit{(package)}!peach.pso \textit{(package)}|)}
