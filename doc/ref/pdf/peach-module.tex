%
% API Documentation for Peach - Computational Intelligence for Python
% Package peach
%
% Generated by epydoc 3.0.1
% [Thu Jul 28 16:37:45 2011]
%

%%%%%%%%%%%%%%%%%%%%%%%%%%%%%%%%%%%%%%%%%%%%%%%%%%%%%%%%%%%%%%%%%%%%%%%%%%%
%%                          Module Description                           %%
%%%%%%%%%%%%%%%%%%%%%%%%%%%%%%%%%%%%%%%%%%%%%%%%%%%%%%%%%%%%%%%%%%%%%%%%%%%

    \index{peach \textit{(package)}|(}
\section{Package peach}

    \label{peach}

\emph{Peach} is a pure-Python package with aims to implement techniques of machine
learning and computational intelligence. It contains packages for
%
\begin{itemize}

\item Neural Networks, including, but not limited to, multi-layer perceptrons and
self-organizing maps;

\item Fuzzy logic and fuzzy inference systems, including Mamdani-type and
Sugeno-type controllers;

\item Optimization packages, including multidimensional optimization;

\item Stochastic Optimizations, including genetic algorithms, simulated annealing,
particle swarm optimization;

\item A lot more.

\end{itemize}
\textbf{Author:} 
José Alexandre Nalon


\textbf{Version:} 0.1.0




%%%%%%%%%%%%%%%%%%%%%%%%%%%%%%%%%%%%%%%%%%%%%%%%%%%%%%%%%%%%%%%%%%%%%%%%%%%
%%                                Modules                                %%
%%%%%%%%%%%%%%%%%%%%%%%%%%%%%%%%%%%%%%%%%%%%%%%%%%%%%%%%%%%%%%%%%%%%%%%%%%%

\subsection{Modules}

\begin{itemize}
\setlength{\parskip}{0ex}
\item \textbf{fuzzy}: 
This package implements fuzzy logic. Consult:


  \textit{(Section \ref{peach:fuzzy}, p.~\pageref{peach:fuzzy})}

  \begin{itemize}
\setlength{\parskip}{0ex}
    \item \textbf{base}: 
This package implements basic definitions for fuzzy logic


  \textit{(Section \ref{peach:fuzzy:base}, p.~\pageref{peach:fuzzy:base})}

    \item \textbf{cmeans}: 
Fuzzy C-Means


  \textit{(Section \ref{peach:fuzzy:cmeans}, p.~\pageref{peach:fuzzy:cmeans})}

    \item \textbf{control}: 
This package implements fuzzy controllers, of fuzzy inference systems.


  \textit{(Section \ref{peach:fuzzy:control}, p.~\pageref{peach:fuzzy:control})}

    \item \textbf{defuzzy}: 
This package implements defuzzification methods for use with fuzzy controllers.


  \textit{(Section \ref{peach:fuzzy:defuzzy}, p.~\pageref{peach:fuzzy:defuzzy})}

    \item \textbf{mf}: 
Membership functions


  \textit{(Section \ref{peach:fuzzy:mf}, p.~\pageref{peach:fuzzy:mf})}

    \item \textbf{norms}: 
This package implements operations of fuzzy logic.


  \textit{(Section \ref{peach:fuzzy:norms}, p.~\pageref{peach:fuzzy:norms})}

  \end{itemize}
\item \textbf{ga}: 
This package implements genetic algorithms. Consult:


  \textit{(Section \ref{peach:ga}, p.~\pageref{peach:ga})}

  \begin{itemize}
\setlength{\parskip}{0ex}
    \item \textbf{base}: 
Basic Genetic Algorithm (GA)


  \textit{(Section \ref{peach:ga:base}, p.~\pageref{peach:ga:base})}

    \item \textbf{chromosome}: 
Basic definitions and classes for manipulating chromosomes


  \textit{(Section \ref{peach:ga:chromosome}, p.~\pageref{peach:ga:chromosome})}

    \item \textbf{crossover}: 
Basic definitions for crossover operations and base classes.


  \textit{(Section \ref{peach:ga:crossover}, p.~\pageref{peach:ga:crossover})}

    \item \textbf{fitness}: 
Basic definitions and base classes for definition of fitness functions for use
with genetic algorithms.


  \textit{(Section \ref{peach:ga:fitness}, p.~\pageref{peach:ga:fitness})}

    \item \textbf{mutation}: 
Basic definitions and classes for operating mutation on chromosomes.


  \textit{(Section \ref{peach:ga:mutation}, p.~\pageref{peach:ga:mutation})}

    \item \textbf{selection}: 
Basic classes and definitions for selection operator.


  \textit{(Section \ref{peach:ga:selection}, p.~\pageref{peach:ga:selection})}

  \end{itemize}
\item \textbf{nn}: 
This package implements support for neural networks. Consult:


  \textit{(Section \ref{peach:nn}, p.~\pageref{peach:nn})}

  \begin{itemize}
\setlength{\parskip}{0ex}
    \item \textbf{af}: 
Base activation functions and base class


  \textit{(Section \ref{peach:nn:af}, p.~\pageref{peach:nn:af})}

    \item \textbf{base}: 
Basic definitions for layers of neurons.


  \textit{(Section \ref{peach:nn:base}, p.~\pageref{peach:nn:base})}

    \item \textbf{kmeans}: 
K-Means clustering algorithm


  \textit{(Section \ref{peach:nn:kmeans}, p.~\pageref{peach:nn:kmeans})}

    \item \textbf{lrules}: 
Learning rules for neural networks and base classes for custom learning.


  \textit{(Section \ref{peach:nn:lrules}, p.~\pageref{peach:nn:lrules})}

    \item \textbf{mem}: 
Associative memories and Hopfield network model.


  \textit{(Section \ref{peach:nn:mem}, p.~\pageref{peach:nn:mem})}

    \item \textbf{nnet}: 
Basic topologies of neural networks.


  \textit{(Section \ref{peach:nn:nnet}, p.~\pageref{peach:nn:nnet})}

  \end{itemize}
\item \textbf{optm}: 
This package implements deterministic optimization methods. Consult:


  \textit{(Section \ref{peach:optm}, p.~\pageref{peach:optm})}

  \begin{itemize}
\setlength{\parskip}{0ex}
    \item \textbf{base}: 
Basic definitons and base class for optimizers


  \textit{(Section \ref{peach:optm:base}, p.~\pageref{peach:optm:base})}

    \item \textbf{linear}: 
This package implements basic one variable only optimizers.


  \textit{(Section \ref{peach:optm:linear}, p.~\pageref{peach:optm:linear})}

    \item \textbf{multivar}: 
This package implements basic multivariable optimizers, including gradient and
Newton searches.


  \textit{(Section \ref{peach:optm:multivar}, p.~\pageref{peach:optm:multivar})}

    \item \textbf{quasinewton}: 
This package implements basic quasi-Newton optimizers. Newton optimizer is very
efficient, except that inverse matrices need to be calculated at each
convergence step. These methods try to estimate the hessian inverse iteratively,
thus increasing performance.


  \textit{(Section \ref{peach:optm:quasinewton}, p.~\pageref{peach:optm:quasinewton})}

    \item \textbf{stochastic}
  \textit{(Section \ref{peach:optm:stochastic}, p.~\pageref{peach:optm:stochastic})}

  \end{itemize}
\item \textbf{pso}: 
Basic Particle Swarm Optimization (PSO)


  \textit{(Section \ref{peach:pso}, p.~\pageref{peach:pso})}

  \begin{itemize}
\setlength{\parskip}{0ex}
    \item \textbf{acc}: 
Functions to update the velocity (ie, accelerate) of the particles in a swarm.


  \textit{(Section \ref{peach:pso:acc}, p.~\pageref{peach:pso:acc})}

    \item \textbf{base}: 
This package implements the simple continuous version of the particle swarm
optimizer. In this implementation, it is possible to specify, besides the
objective function and the first estimates, the ranges of search, which will
influence the max velocity of the particles, and the population size. Other
parameters are available too, please refer to the rest of this documentation for
further details.


  \textit{(Section \ref{peach:pso:base}, p.~\pageref{peach:pso:base})}

  \end{itemize}
\item \textbf{sa}: 
This package implements optimization by simulated annealing. Consult:


  \textit{(Section \ref{peach:sa}, p.~\pageref{peach:sa})}

  \begin{itemize}
\setlength{\parskip}{0ex}
    \item \textbf{base}: 
This package implements two versions of simulated annealing optimization. One
works with numeric data, and the other with a codified bit string. This last
method can be used in discrete optimization problems.


  \textit{(Section \ref{peach:sa:base}, p.~\pageref{peach:sa:base})}

    \item \textbf{neighbor}: 
This module implements a general class to compute neighbors for continuous and
binary simulated annealing algorithms. The continuous neighbor functions return
an array with a neighbor of a given estimate; the binary neighbor functions
return a \texttt{bitarray} object.


  \textit{(Section \ref{peach:sa:neighbor}, p.~\pageref{peach:sa:neighbor})}

  \end{itemize}
\end{itemize}

    \index{peach \textit{(package)}|)}
