%
% API Documentation for Peach - Computational Intelligence for Python
% Package peach
%
% Generated by epydoc 3.0beta1
% [Mon Dec 21 08:51:35 2009]
%

%%%%%%%%%%%%%%%%%%%%%%%%%%%%%%%%%%%%%%%%%%%%%%%%%%%%%%%%%%%%%%%%%%%%%%%%%%%
%%                          Module Description                           %%
%%%%%%%%%%%%%%%%%%%%%%%%%%%%%%%%%%%%%%%%%%%%%%%%%%%%%%%%%%%%%%%%%%%%%%%%%%%

    \index{peach \textit{(package)}|(}
\section{Package peach}

    \label{peach}

\emph{Peach} is a pure-Python package with aims to implement techniques of machine
learning and computational intelligence. It contains packages for
\begin{itemize}
\item {} 
Neural Networks, including, but not limited to, multi-layer perceptrons and
self-organizing maps;

\item {} 
Fuzzy logic and fuzzy inference systems, including Mamdani-type and
Sugeno-type controllers;

\item {} 
Optimization packages, including multidimensional optimization;

\item {} 
Stochastic Optimizations, including genetic algorithms, simulated annealing,
particle swarm optimization;

\item {} 
A lot more.

\end{itemize}
\textbf{Author:} 
Jos� Alexandre Nalon


\textbf{Version:} 0.1.0




%%%%%%%%%%%%%%%%%%%%%%%%%%%%%%%%%%%%%%%%%%%%%%%%%%%%%%%%%%%%%%%%%%%%%%%%%%%
%%                                Modules                                %%
%%%%%%%%%%%%%%%%%%%%%%%%%%%%%%%%%%%%%%%%%%%%%%%%%%%%%%%%%%%%%%%%%%%%%%%%%%%

\subsection{Modules}

\begin{itemize}
\setlength{\parskip}{0ex}
\item \textbf{fuzzy}: 
This package implements fuzzy logic.


  \textit{(Section \ref{peach:fuzzy}, p.~\pageref{peach:fuzzy})}

  \begin{itemize}
\setlength{\parskip}{0ex}
    \item \textbf{control}: 
This package implements fuzzy controllers, of fuzzy inference systems.


  \textit{(Section \ref{peach:fuzzy:control}, p.~\pageref{peach:fuzzy:control})}

    \item \textbf{defuzzy}: 
This package implements defuzzification methods for use with fuzzy controllers.


  \textit{(Section \ref{peach:fuzzy:defuzzy}, p.~\pageref{peach:fuzzy:defuzzy})}

    \item \textbf{fuzzy}: 
This package implements basic definitions for fuzzy logic


  \textit{(Section \ref{peach:fuzzy:fuzzy}, p.~\pageref{peach:fuzzy:fuzzy})}

    \item \textbf{mf}: 
Membership functions


  \textit{(Section \ref{peach:fuzzy:mf}, p.~\pageref{peach:fuzzy:mf})}

    \item \textbf{norms}: 
This package implements operations of fuzzy logic.


  \textit{(Section \ref{peach:fuzzy:norms}, p.~\pageref{peach:fuzzy:norms})}

  \end{itemize}
\item \textbf{ga}: 
This package implements genetic algorithms.


  \textit{(Section \ref{peach:ga}, p.~\pageref{peach:ga})}

  \begin{itemize}
\setlength{\parskip}{0ex}
    \item \textbf{chromosome}: 
Basic definitions and classes for manipulating chromosomes


  \textit{(Section \ref{peach:ga:chromosome}, p.~\pageref{peach:ga:chromosome})}

    \item \textbf{crossover}: 
Basic definitions for crossover operations and base classes.


  \textit{(Section \ref{peach:ga:crossover}, p.~\pageref{peach:ga:crossover})}

    \item \textbf{fitness}: 
Basic definitions and base classes for definition of fitness functions for use
with genetic algorithms.


  \textit{(Section \ref{peach:ga:fitness}, p.~\pageref{peach:ga:fitness})}

    \item \textbf{ga}: 
Basic Genetic Algorithm (GA)


  \textit{(Section \ref{peach:ga:ga}, p.~\pageref{peach:ga:ga})}

    \item \textbf{mutation}: 
Basic definitions and classes for operating mutation on chromosomes.


  \textit{(Section \ref{peach:ga:mutation}, p.~\pageref{peach:ga:mutation})}

    \item \textbf{selection}: 
Basic classes and definitions for selection operator.


  \textit{(Section \ref{peach:ga:selection}, p.~\pageref{peach:ga:selection})}

  \end{itemize}
\item \textbf{nn}: 
This package implements support for neural networks.


  \textit{(Section \ref{peach:nn}, p.~\pageref{peach:nn})}

  \begin{itemize}
\setlength{\parskip}{0ex}
    \item \textbf{af}: 
Base activation functions and base class


  \textit{(Section \ref{peach:nn:af}, p.~\pageref{peach:nn:af})}

    \item \textbf{base}: 
Basic definitions for layers of neurons.


  \textit{(Section \ref{peach:nn:base}, p.~\pageref{peach:nn:base})}

    \item \textbf{lrules}: 
Learning rules for neural networks and base classes for custom learning.


  \textit{(Section \ref{peach:nn:lrules}, p.~\pageref{peach:nn:lrules})}

    \item \textbf{nn}: 
Basic topologies of neural networks.


  \textit{(Section \ref{peach:nn:nn}, p.~\pageref{peach:nn:nn})}

  \end{itemize}
\item \textbf{optm}: 
This package implements deterministic optimization methods.


  \textit{(Section \ref{peach:optm}, p.~\pageref{peach:optm})}

  \begin{itemize}
\setlength{\parskip}{0ex}
    \item \textbf{linear}: 
This package implements basic one variable only optimizers.


  \textit{(Section \ref{peach:optm:linear}, p.~\pageref{peach:optm:linear})}

    \item \textbf{multivar}: 
This package implements basic multivariable optimizers, including gradient and
Newton searches.


  \textit{(Section \ref{peach:optm:multivar}, p.~\pageref{peach:optm:multivar})}

    \item \textbf{optm}: 
Basic definitons and base class for optimizers


  \textit{(Section \ref{peach:optm:optm}, p.~\pageref{peach:optm:optm})}

    \item \textbf{quasinewton}: 
This package implements basic quasi-Newton optimizers.


  \textit{(Section \ref{peach:optm:quasinewton}, p.~\pageref{peach:optm:quasinewton})}

    \item \textbf{sa}: 
This package implements two versions of simulated annealing optimization.


  \textit{(Section \ref{peach:optm:sa}, p.~\pageref{peach:optm:sa})}

    \item \textbf{stochastic}: 
General methods of stochastic optimization.


  \textit{(Section \ref{peach:optm:stochastic}, p.~\pageref{peach:optm:stochastic})}

  \end{itemize}
\end{itemize}


%%%%%%%%%%%%%%%%%%%%%%%%%%%%%%%%%%%%%%%%%%%%%%%%%%%%%%%%%%%%%%%%%%%%%%%%%%%
%%                               Variables                               %%
%%%%%%%%%%%%%%%%%%%%%%%%%%%%%%%%%%%%%%%%%%%%%%%%%%%%%%%%%%%%%%%%%%%%%%%%%%%

  \subsection{Variables}

\begin{longtable}{|p{.30\textwidth}|p{.62\textwidth}|l}
\cline{1-2}
\cline{1-2} \centering \textbf{Name} & \centering \textbf{Description}& \\
\cline{1-2}
\endhead\cline{1-2}\multicolumn{3}{r}{\small\textit{continued on next page}}\\\endfoot\cline{1-2}
\endlastfoot\raggedright \_\-\_\-d\-o\-c\-\_\-\_\- & \raggedright \textbf{Value:} 
{\tt \texttt{...}}&\\
\cline{1-2}
\end{longtable}

    \index{peach \textit{(package)}|)}
