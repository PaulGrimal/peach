%
% API Documentation for Peach - Computational Intelligence for Python
% Module peach.ga.mutation
%
% Generated by epydoc 3.0.1
% [Thu Jul 28 16:37:47 2011]
%

%%%%%%%%%%%%%%%%%%%%%%%%%%%%%%%%%%%%%%%%%%%%%%%%%%%%%%%%%%%%%%%%%%%%%%%%%%%
%%                          Module Description                           %%
%%%%%%%%%%%%%%%%%%%%%%%%%%%%%%%%%%%%%%%%%%%%%%%%%%%%%%%%%%%%%%%%%%%%%%%%%%%

    \index{peach \textit{(package)}!peach.ga \textit{(package)}!peach.ga.mutation \textit{(module)}|(}
\section{Module peach.ga.mutation}

    \label{peach:ga:mutation}

Basic definitions and classes for operating mutation on chromosomes.

The mutation operator changes selected bits in the array corresponding to the
chromosome. This operation is not as common as the others, but some genetic
algorithms still implement it.

%%%%%%%%%%%%%%%%%%%%%%%%%%%%%%%%%%%%%%%%%%%%%%%%%%%%%%%%%%%%%%%%%%%%%%%%%%%
%%                               Variables                               %%
%%%%%%%%%%%%%%%%%%%%%%%%%%%%%%%%%%%%%%%%%%%%%%%%%%%%%%%%%%%%%%%%%%%%%%%%%%%

  \subsection{Variables}

    \vspace{-1cm}
\hspace{\varindent}\begin{longtable}{|p{\varnamewidth}|p{\vardescrwidth}|l}
\cline{1-2}
\cline{1-2} \centering \textbf{Name} & \centering \textbf{Description}& \\
\cline{1-2}
\endhead\cline{1-2}\multicolumn{3}{r}{\small\textit{continued on next page}}\\\endfoot\cline{1-2}
\endlastfoot\raggedright \_\-\_\-d\-o\-c\-\_\-\_\- & \raggedright \textbf{Value:} 
{\tt \texttt{...}}&\\
\cline{1-2}
\raggedright \_\-\_\-p\-a\-c\-k\-a\-g\-e\-\_\-\_\- & \raggedright \textbf{Value:} 
{\tt \texttt{'}\texttt{peach.ga}\texttt{'}}&\\
\cline{1-2}
\end{longtable}


%%%%%%%%%%%%%%%%%%%%%%%%%%%%%%%%%%%%%%%%%%%%%%%%%%%%%%%%%%%%%%%%%%%%%%%%%%%
%%                           Class Description                           %%
%%%%%%%%%%%%%%%%%%%%%%%%%%%%%%%%%%%%%%%%%%%%%%%%%%%%%%%%%%%%%%%%%%%%%%%%%%%

    \index{peach \textit{(package)}!peach.ga \textit{(package)}!peach.ga.mutation \textit{(module)}!peach.ga.mutation.Mutation \textit{(class)}|(}
\subsection{Class Mutation}

    \label{peach:ga:mutation:Mutation}
\begin{tabular}{cccccc}
% Line for object, linespec=[False]
\multicolumn{2}{r}{\settowidth{\BCL}{object}\multirow{2}{\BCL}{object}}
&&
  \\\cline{3-3}
  &&\multicolumn{1}{c|}{}
&&
  \\
&&\multicolumn{2}{l}{\textbf{peach.ga.mutation.Mutation}}
\end{tabular}

\textbf{Known Subclasses:} peach.ga.mutation.BitToBit


Base class for mutation operators.

This class should be subclassed if you want to create your own mutation
operator. The base class doesn't do much, it is only a prototype. As is done
with all the base classes within this library, use the \texttt{\_\_init\_\_} method
to configure your mutation behaviour -{}- if needed -{}- and the \texttt{\_\_call\_\_}
method to operate over a population.

A class derived from this one should implement at least 2 methods, defined
below:
%
\begin{quote}
%
\begin{description}
\item[{\_\_init\_\_(self, %
\raisebox{1em}{\hypertarget{id2}{}}\hyperlink{id1}{\textbf{\color{red}*}}cnf, %
\raisebox{1em}{\hypertarget{id4}{}}\hyperlink{id3}{\textbf{\color{red}**}}kw)}] \leavevmode 
Initializes the object. There is no mandatory arguments, but any
parameters can be used here to configure the operator. For example, a
class can define a mutation rate -{}- this should be defined here:
%
\begin{quote}{\ttfamily \raggedright \noindent
\_\_init\_\_(self,~rate=0.75)
}
\end{quote}

A default value should always be offered, if possible.

\item[{\_\_call\_\_(self, population)}] \leavevmode 
The \texttt{\_\_call\_\_} implementation should receive a population and operate
over it. Please, consult the \texttt{ga} module to see more information on
populations. It should return the processed population. No recomendation
on the internals of the method is made.

\end{description}

\end{quote}

Please, note that the GA implementations relies on this behaviour: it will
pass a population to your \texttt{\_\_call\_\_} method and expects to received the
result back.

%%%%%%%%%%%%%%%%%%%%%%%%%%%%%%%%%%%%%%%%%%%%%%%%%%%%%%%%%%%%%%%%%%%%%%%%%%%
%%                                Methods                                %%
%%%%%%%%%%%%%%%%%%%%%%%%%%%%%%%%%%%%%%%%%%%%%%%%%%%%%%%%%%%%%%%%%%%%%%%%%%%

  \subsubsection{Methods}


\large{\textbf{\textit{Inherited from object}}}

\begin{quote}
\_\_delattr\_\_(), \_\_format\_\_(), \_\_getattribute\_\_(), \_\_hash\_\_(), \_\_init\_\_(), \_\_new\_\_(), \_\_reduce\_\_(), \_\_reduce\_ex\_\_(), \_\_repr\_\_(), \_\_setattr\_\_(), \_\_sizeof\_\_(), \_\_str\_\_(), \_\_subclasshook\_\_()
\end{quote}

%%%%%%%%%%%%%%%%%%%%%%%%%%%%%%%%%%%%%%%%%%%%%%%%%%%%%%%%%%%%%%%%%%%%%%%%%%%
%%                              Properties                               %%
%%%%%%%%%%%%%%%%%%%%%%%%%%%%%%%%%%%%%%%%%%%%%%%%%%%%%%%%%%%%%%%%%%%%%%%%%%%

  \subsubsection{Properties}

    \vspace{-1cm}
\hspace{\varindent}\begin{longtable}{|p{\varnamewidth}|p{\vardescrwidth}|l}
\cline{1-2}
\cline{1-2} \centering \textbf{Name} & \centering \textbf{Description}& \\
\cline{1-2}
\endhead\cline{1-2}\multicolumn{3}{r}{\small\textit{continued on next page}}\\\endfoot\cline{1-2}
\endlastfoot\multicolumn{2}{|l|}{\textit{Inherited from object}}\\
\multicolumn{2}{|p{\varwidth}|}{\raggedright \_\_class\_\_}\\
\cline{1-2}
\end{longtable}

    \index{peach \textit{(package)}!peach.ga \textit{(package)}!peach.ga.mutation \textit{(module)}!peach.ga.mutation.Mutation \textit{(class)}|)}

%%%%%%%%%%%%%%%%%%%%%%%%%%%%%%%%%%%%%%%%%%%%%%%%%%%%%%%%%%%%%%%%%%%%%%%%%%%
%%                           Class Description                           %%
%%%%%%%%%%%%%%%%%%%%%%%%%%%%%%%%%%%%%%%%%%%%%%%%%%%%%%%%%%%%%%%%%%%%%%%%%%%

    \index{peach \textit{(package)}!peach.ga \textit{(package)}!peach.ga.mutation \textit{(module)}!peach.ga.mutation.BitToBit \textit{(class)}|(}
\subsection{Class BitToBit}

    \label{peach:ga:mutation:BitToBit}
\begin{tabular}{cccccccc}
% Line for object, linespec=[False, False]
\multicolumn{2}{r}{\settowidth{\BCL}{object}\multirow{2}{\BCL}{object}}
&&
&&
  \\\cline{3-3}
  &&\multicolumn{1}{c|}{}
&&
&&
  \\
% Line for peach.ga.mutation.Mutation, linespec=[False]
\multicolumn{4}{r}{\settowidth{\BCL}{peach.ga.mutation.Mutation}\multirow{2}{\BCL}{peach.ga.mutation.Mutation}}
&&
  \\\cline{5-5}
  &&&&\multicolumn{1}{c|}{}
&&
  \\
&&&&\multicolumn{2}{l}{\textbf{peach.ga.mutation.BitToBit}}
\end{tabular}


A simple bit-to-bit mutation operator.

This operator scans every individual in the population, in a bit-to-bit
fashion. If a uniformly random number is less than the mutation rate (see
below), then the bit is inverted. The mutation should be made very small,
since large populations will represent a big number of bits; it should never
be more than 0.5.

%%%%%%%%%%%%%%%%%%%%%%%%%%%%%%%%%%%%%%%%%%%%%%%%%%%%%%%%%%%%%%%%%%%%%%%%%%%
%%                                Methods                                %%
%%%%%%%%%%%%%%%%%%%%%%%%%%%%%%%%%%%%%%%%%%%%%%%%%%%%%%%%%%%%%%%%%%%%%%%%%%%

  \subsubsection{Methods}

    \vspace{0.5ex}

\hspace{.8\funcindent}\begin{boxedminipage}{\funcwidth}

    \raggedright \textbf{\_\_init\_\_}(\textit{self}, \textit{rate}={\tt 0.05})

    \vspace{-1.5ex}

    \rule{\textwidth}{0.5\fboxrule}
\setlength{\parskip}{2ex}

Initialize the mutation operator.
\setlength{\parskip}{1ex}
      \textbf{Parameters}
      \vspace{-1ex}

      \begin{quote}
        \begin{Ventry}{xxxx}

          \item[rate]


Probability that a single bit in an individual will be inverted.
        \end{Ventry}

      \end{quote}

      Overrides: object.\_\_init\_\_

    \end{boxedminipage}

    \label{peach:ga:mutation:BitToBit:__call__}
    \index{peach \textit{(package)}!peach.ga \textit{(package)}!peach.ga.mutation \textit{(module)}!peach.ga.mutation.BitToBit \textit{(class)}!peach.ga.mutation.BitToBit.\_\_call\_\_ \textit{(method)}}

    \vspace{0.5ex}

\hspace{.8\funcindent}\begin{boxedminipage}{\funcwidth}

    \raggedright \textbf{\_\_call\_\_}(\textit{self}, \textit{population})

    \vspace{-1.5ex}

    \rule{\textwidth}{0.5\fboxrule}
\setlength{\parskip}{2ex}

Applies the operator over a population.

The behaviour of this operator is as described above: it scans every bit
in every individual, and if a random number is less than the mutation
rate, the bit is inverted.
\setlength{\parskip}{1ex}
      \textbf{Parameters}
      \vspace{-1ex}

      \begin{quote}
        \begin{Ventry}{xxxxxxxxxx}

          \item[population]


A list of \texttt{Chromosomes} containing the present population of the
algorithm. It is processed and the results of the exchange are
returned to the caller.
        \end{Ventry}

      \end{quote}

      \textbf{Return Value}
    \vspace{-1ex}

      \begin{quote}

The processed population, a list of \texttt{Chromosomes}.
      \end{quote}

    \end{boxedminipage}


\large{\textbf{\textit{Inherited from object}}}

\begin{quote}
\_\_delattr\_\_(), \_\_format\_\_(), \_\_getattribute\_\_(), \_\_hash\_\_(), \_\_new\_\_(), \_\_reduce\_\_(), \_\_reduce\_ex\_\_(), \_\_repr\_\_(), \_\_setattr\_\_(), \_\_sizeof\_\_(), \_\_str\_\_(), \_\_subclasshook\_\_()
\end{quote}

%%%%%%%%%%%%%%%%%%%%%%%%%%%%%%%%%%%%%%%%%%%%%%%%%%%%%%%%%%%%%%%%%%%%%%%%%%%
%%                              Properties                               %%
%%%%%%%%%%%%%%%%%%%%%%%%%%%%%%%%%%%%%%%%%%%%%%%%%%%%%%%%%%%%%%%%%%%%%%%%%%%

  \subsubsection{Properties}

    \vspace{-1cm}
\hspace{\varindent}\begin{longtable}{|p{\varnamewidth}|p{\vardescrwidth}|l}
\cline{1-2}
\cline{1-2} \centering \textbf{Name} & \centering \textbf{Description}& \\
\cline{1-2}
\endhead\cline{1-2}\multicolumn{3}{r}{\small\textit{continued on next page}}\\\endfoot\cline{1-2}
\endlastfoot\multicolumn{2}{|l|}{\textit{Inherited from object}}\\
\multicolumn{2}{|p{\varwidth}|}{\raggedright \_\_class\_\_}\\
\cline{1-2}
\end{longtable}


%%%%%%%%%%%%%%%%%%%%%%%%%%%%%%%%%%%%%%%%%%%%%%%%%%%%%%%%%%%%%%%%%%%%%%%%%%%
%%                          Instance Variables                           %%
%%%%%%%%%%%%%%%%%%%%%%%%%%%%%%%%%%%%%%%%%%%%%%%%%%%%%%%%%%%%%%%%%%%%%%%%%%%

  \subsubsection{Instance Variables}

    \vspace{-1cm}
\hspace{\varindent}\begin{longtable}{|p{\varnamewidth}|p{\vardescrwidth}|l}
\cline{1-2}
\cline{1-2} \centering \textbf{Name} & \centering \textbf{Description}& \\
\cline{1-2}
\endhead\cline{1-2}\multicolumn{3}{r}{\small\textit{continued on next page}}\\\endfoot\cline{1-2}
\endlastfoot\raggedright r\-a\-t\-e\- & Property that contains the mutation rate.&\\
\cline{1-2}
\end{longtable}

    \index{peach \textit{(package)}!peach.ga \textit{(package)}!peach.ga.mutation \textit{(module)}!peach.ga.mutation.BitToBit \textit{(class)}|)}
    \index{peach \textit{(package)}!peach.ga \textit{(package)}!peach.ga.mutation \textit{(module)}|)}
