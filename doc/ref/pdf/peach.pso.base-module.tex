%
% API Documentation for Peach - Computational Intelligence for Python
% Module peach.pso.base
%
% Generated by epydoc 3.0.1
% [Thu Jul 28 16:37:50 2011]
%

%%%%%%%%%%%%%%%%%%%%%%%%%%%%%%%%%%%%%%%%%%%%%%%%%%%%%%%%%%%%%%%%%%%%%%%%%%%
%%                          Module Description                           %%
%%%%%%%%%%%%%%%%%%%%%%%%%%%%%%%%%%%%%%%%%%%%%%%%%%%%%%%%%%%%%%%%%%%%%%%%%%%

    \index{peach \textit{(package)}!peach.pso \textit{(package)}!peach.pso.base \textit{(module)}|(}
\section{Module peach.pso.base}

    \label{peach:pso:base}

This package implements the simple continuous version of the particle swarm
optimizer. In this implementation, it is possible to specify, besides the
objective function and the first estimates, the ranges of search, which will
influence the max velocity of the particles, and the population size. Other
parameters are available too, please refer to the rest of this documentation for
further details.

%%%%%%%%%%%%%%%%%%%%%%%%%%%%%%%%%%%%%%%%%%%%%%%%%%%%%%%%%%%%%%%%%%%%%%%%%%%
%%                               Variables                               %%
%%%%%%%%%%%%%%%%%%%%%%%%%%%%%%%%%%%%%%%%%%%%%%%%%%%%%%%%%%%%%%%%%%%%%%%%%%%

  \subsection{Variables}

    \vspace{-1cm}
\hspace{\varindent}\begin{longtable}{|p{\varnamewidth}|p{\vardescrwidth}|l}
\cline{1-2}
\cline{1-2} \centering \textbf{Name} & \centering \textbf{Description}& \\
\cline{1-2}
\endhead\cline{1-2}\multicolumn{3}{r}{\small\textit{continued on next page}}\\\endfoot\cline{1-2}
\endlastfoot\raggedright \_\-\_\-d\-o\-c\-\_\-\_\- & \raggedright \textbf{Value:} 
{\tt \texttt{...}}&\\
\cline{1-2}
\raggedright \_\-\_\-p\-a\-c\-k\-a\-g\-e\-\_\-\_\- & \raggedright \textbf{Value:} 
{\tt \texttt{'}\texttt{peach.pso}\texttt{'}}&\\
\cline{1-2}
\raggedright a\-b\-s\- & \raggedright \textbf{Value:} 
{\tt {\textless}ufunc 'absolute'{\textgreater}}&\\
\cline{1-2}
\raggedright s\-i\-g\-n\- & \raggedright \textbf{Value:} 
{\tt {\textless}ufunc 'sign'{\textgreater}}&\\
\cline{1-2}
\raggedright s\-q\-r\-t\- & \raggedright \textbf{Value:} 
{\tt {\textless}ufunc 'sqrt'{\textgreater}}&\\
\cline{1-2}
\end{longtable}


%%%%%%%%%%%%%%%%%%%%%%%%%%%%%%%%%%%%%%%%%%%%%%%%%%%%%%%%%%%%%%%%%%%%%%%%%%%
%%                           Class Description                           %%
%%%%%%%%%%%%%%%%%%%%%%%%%%%%%%%%%%%%%%%%%%%%%%%%%%%%%%%%%%%%%%%%%%%%%%%%%%%

    \index{peach \textit{(package)}!peach.pso \textit{(package)}!peach.pso.base \textit{(module)}!peach.pso.base.ParticleSwarmOptimizer \textit{(class)}|(}
\subsection{Class ParticleSwarmOptimizer}

    \label{peach:pso:base:ParticleSwarmOptimizer}
\begin{tabular}{cccccccc}
% Line for object, linespec=[False, False]
\multicolumn{2}{r}{\settowidth{\BCL}{object}\multirow{2}{\BCL}{object}}
&&
&&
  \\\cline{3-3}
  &&\multicolumn{1}{c|}{}
&&
&&
  \\
% Line for list, linespec=[False]
\multicolumn{4}{r}{\settowidth{\BCL}{list}\multirow{2}{\BCL}{list}}
&&
  \\\cline{5-5}
  &&&&\multicolumn{1}{c|}{}
&&
  \\
&&&&\multicolumn{2}{l}{\textbf{peach.pso.base.ParticleSwarmOptimizer}}
\end{tabular}

\textbf{Known Subclasses:} peach.pso.base.PSO


A standard Particle Swarm Optimizer

This class implements a particle swarm optimization (PSO) procedure. A
swarm is a list of estimates, and should answer to every \texttt{list} method. A
population of particles is created to travel through the search domain with
a certain velocity. At each point, the objective function is evaluated for
each particle, and the positions are adjusted correspondingly. The velocity
is then modified (ie, the particles are accelerated) towards its 'personal'
best (the best value found by that particle at the moment) and a global best
(the best value found overall at the moment).

%%%%%%%%%%%%%%%%%%%%%%%%%%%%%%%%%%%%%%%%%%%%%%%%%%%%%%%%%%%%%%%%%%%%%%%%%%%
%%                                Methods                                %%
%%%%%%%%%%%%%%%%%%%%%%%%%%%%%%%%%%%%%%%%%%%%%%%%%%%%%%%%%%%%%%%%%%%%%%%%%%%

  \subsubsection{Methods}

    \vspace{0.5ex}

\hspace{.8\funcindent}\begin{boxedminipage}{\funcwidth}

    \raggedright \textbf{\_\_init\_\_}(\textit{self}, \textit{f}, \textit{x0}, \textit{ranges}={\tt None}, \textit{accelerator}={\tt {\textless}class 'peach.pso.acc.StandardPSO'{\textgreater}}, \textit{emax}={\tt 1e-05}, \textit{imax}={\tt 1000})

    \vspace{-1.5ex}

    \rule{\textwidth}{0.5\fboxrule}
\setlength{\parskip}{2ex}

Initializes the optimizer.
\setlength{\parskip}{1ex}
      \textbf{Parameters}
      \vspace{-1ex}

      \begin{quote}
        \begin{Ventry}{xxxxxxxxxxx}

          \item[f]


A multivariable function to be evaluated. It must receive only one
parameter, a multidimensional line-vector with the same dimensions
of the range list (see below) and return a real value, a scalar.
          \item[x0]


A population of first estimates. This is a list, array or tuple of
one-dimension arrays, each one corresponding to an estimate of the
position of the minimum. The population size of the algorithm will
be the same as the number of estimates in this list. Each component
of the vectors in this list are one of the variables in the function
to be optimized.
          \item[ranges]


A range of values might be passed to the algorithm, but it is not
necessary. If this parameter is not supplied, then the ranges will
be computed from the estimates, but be aware that this might not
represent the complete search space. If supplied, this parameter
should be a list of ranges for each variable of the objective
function. It is specified as a list of tuples of two values,
\texttt{(x0, x1)}, where \texttt{x0} is the start of the interval, and \texttt{x1}
its end. Obviously, \texttt{x0} should be smaller than \texttt{x1}. It can
also be given as a list with a simple tuple in the same format. In
that case, the same range will be applied for every variable in the
optimization.
          \item[accelerator]


An acceleration method, please consult the documentation on \texttt{acc}
module. Defaults to StandardPSO, that is, velocities change based on
local and global bests.
          \item[emax]


Maximum allowed error. The algorithm stops as soon as the error is
below this level. The error is absolute.
          \item[imax]


Maximum number of iterations, the algorithm stops as soon this
number of iterations are executed, no matter what the error is at
the moment.
        \end{Ventry}

      \end{quote}

      \textbf{Return Value}
    \vspace{-1ex}

      \begin{quote}

new empty list
      \end{quote}

      Overrides: object.\_\_init\_\_

    \end{boxedminipage}

    \label{peach:pso:base:ParticleSwarmOptimizer:restart}
    \index{peach \textit{(package)}!peach.pso \textit{(package)}!peach.pso.base \textit{(module)}!peach.pso.base.ParticleSwarmOptimizer \textit{(class)}!peach.pso.base.ParticleSwarmOptimizer.restart \textit{(method)}}

    \vspace{0.5ex}

\hspace{.8\funcindent}\begin{boxedminipage}{\funcwidth}

    \raggedright \textbf{restart}(\textit{self}, \textit{x0})

    \vspace{-1.5ex}

    \rule{\textwidth}{0.5\fboxrule}
\setlength{\parskip}{2ex}

Resets the optimizer, allowing the use of a new set of estimates. This
can be used to avoid stagnation
\setlength{\parskip}{1ex}
      \textbf{Parameters}
      \vspace{-1ex}

      \begin{quote}
        \begin{Ventry}{xx}

          \item[x0]


A new set of estimates. It doesn't need to have the same size of the
original swarm, but it must be a list of estimates in the same
format as in the object instantiation. Please, see the documentation
on the instantiation of the class. New velocities will be computed.
        \end{Ventry}

      \end{quote}

    \end{boxedminipage}

    \label{peach:pso:base:ParticleSwarmOptimizer:step}
    \index{peach \textit{(package)}!peach.pso \textit{(package)}!peach.pso.base \textit{(module)}!peach.pso.base.ParticleSwarmOptimizer \textit{(class)}!peach.pso.base.ParticleSwarmOptimizer.step \textit{(method)}}

    \vspace{0.5ex}

\hspace{.8\funcindent}\begin{boxedminipage}{\funcwidth}

    \raggedright \textbf{step}(\textit{self})

    \vspace{-1.5ex}

    \rule{\textwidth}{0.5\fboxrule}
\setlength{\parskip}{2ex}

Computes the new positions of the particles, a step of the algorithm.

This method updates the velocity given the constants associated with the
particle and global bests; and then updates the positions accordingly.

This method has no parameters and returns no values. The particles
positions can be consulted with the \texttt{{[}{]}} interface (as a swarm of
particles is a list of estimates), \texttt{best} property, to find the global
best, and \texttt{fbest} property to find the minimum (see above).
\setlength{\parskip}{1ex}
    \end{boxedminipage}

    \label{peach:pso:base:ParticleSwarmOptimizer:__call__}
    \index{peach \textit{(package)}!peach.pso \textit{(package)}!peach.pso.base \textit{(module)}!peach.pso.base.ParticleSwarmOptimizer \textit{(class)}!peach.pso.base.ParticleSwarmOptimizer.\_\_call\_\_ \textit{(method)}}

    \vspace{0.5ex}

\hspace{.8\funcindent}\begin{boxedminipage}{\funcwidth}

    \raggedright \textbf{\_\_call\_\_}(\textit{self})

    \vspace{-1.5ex}

    \rule{\textwidth}{0.5\fboxrule}
\setlength{\parskip}{2ex}

Transparently executes the search until the minimum is found. The stop
criteria are the maximum error or the maximum number of iterations,
whichever is reached first. Note that this is a \texttt{\_\_call\_\_} method, so
the object is called as a function. This method returns a tuple
\texttt{(x, e)}, with the best estimate of the minimum and the error.
\setlength{\parskip}{1ex}
      \textbf{Return Value}
    \vspace{-1ex}

      \begin{quote}

This method returns a tuple \texttt{(x, e)}, where \texttt{x} is the best
estimate of the minimum, and \texttt{e} is the estimated error.
      \end{quote}

    \end{boxedminipage}


\large{\textbf{\textit{Inherited from list}}}

\begin{quote}
\_\_add\_\_(), \_\_contains\_\_(), \_\_delitem\_\_(), \_\_delslice\_\_(), \_\_eq\_\_(), \_\_ge\_\_(), \_\_getattribute\_\_(), \_\_getitem\_\_(), \_\_getslice\_\_(), \_\_gt\_\_(), \_\_iadd\_\_(), \_\_imul\_\_(), \_\_iter\_\_(), \_\_le\_\_(), \_\_len\_\_(), \_\_lt\_\_(), \_\_mul\_\_(), \_\_ne\_\_(), \_\_new\_\_(), \_\_repr\_\_(), \_\_reversed\_\_(), \_\_rmul\_\_(), \_\_setitem\_\_(), \_\_setslice\_\_(), \_\_sizeof\_\_(), append(), count(), extend(), index(), insert(), pop(), remove(), reverse(), sort()
\end{quote}

\large{\textbf{\textit{Inherited from object}}}

\begin{quote}
\_\_delattr\_\_(), \_\_format\_\_(), \_\_reduce\_\_(), \_\_reduce\_ex\_\_(), \_\_setattr\_\_(), \_\_str\_\_(), \_\_subclasshook\_\_()
\end{quote}

%%%%%%%%%%%%%%%%%%%%%%%%%%%%%%%%%%%%%%%%%%%%%%%%%%%%%%%%%%%%%%%%%%%%%%%%%%%
%%                              Properties                               %%
%%%%%%%%%%%%%%%%%%%%%%%%%%%%%%%%%%%%%%%%%%%%%%%%%%%%%%%%%%%%%%%%%%%%%%%%%%%

  \subsubsection{Properties}

    \vspace{-1cm}
\hspace{\varindent}\begin{longtable}{|p{\varnamewidth}|p{\vardescrwidth}|l}
\cline{1-2}
\cline{1-2} \centering \textbf{Name} & \centering \textbf{Description}& \\
\cline{1-2}
\endhead\cline{1-2}\multicolumn{3}{r}{\small\textit{continued on next page}}\\\endfoot\cline{1-2}
\endlastfoot\raggedright f\-x\- & &\\
\cline{1-2}
\raggedright b\-e\-s\-t\- & &\\
\cline{1-2}
\raggedright f\-b\-e\-s\-t\- & &\\
\cline{1-2}
\multicolumn{2}{|l|}{\textit{Inherited from object}}\\
\multicolumn{2}{|p{\varwidth}|}{\raggedright \_\_class\_\_}\\
\cline{1-2}
\end{longtable}


%%%%%%%%%%%%%%%%%%%%%%%%%%%%%%%%%%%%%%%%%%%%%%%%%%%%%%%%%%%%%%%%%%%%%%%%%%%
%%                            Class Variables                            %%
%%%%%%%%%%%%%%%%%%%%%%%%%%%%%%%%%%%%%%%%%%%%%%%%%%%%%%%%%%%%%%%%%%%%%%%%%%%

  \subsubsection{Class Variables}

    \vspace{-1cm}
\hspace{\varindent}\begin{longtable}{|p{\varnamewidth}|p{\vardescrwidth}|l}
\cline{1-2}
\cline{1-2} \centering \textbf{Name} & \centering \textbf{Description}& \\
\cline{1-2}
\endhead\cline{1-2}\multicolumn{3}{r}{\small\textit{continued on next page}}\\\endfoot\cline{1-2}
\endlastfoot\multicolumn{2}{|l|}{\textit{Inherited from list}}\\
\multicolumn{2}{|p{\varwidth}|}{\raggedright \_\_hash\_\_}\\
\cline{1-2}
\end{longtable}


%%%%%%%%%%%%%%%%%%%%%%%%%%%%%%%%%%%%%%%%%%%%%%%%%%%%%%%%%%%%%%%%%%%%%%%%%%%
%%                          Instance Variables                           %%
%%%%%%%%%%%%%%%%%%%%%%%%%%%%%%%%%%%%%%%%%%%%%%%%%%%%%%%%%%%%%%%%%%%%%%%%%%%

  \subsubsection{Instance Variables}

    \vspace{-1cm}
\hspace{\varindent}\begin{longtable}{|p{\varnamewidth}|p{\vardescrwidth}|l}
\cline{1-2}
\cline{1-2} \centering \textbf{Name} & \centering \textbf{Description}& \\
\cline{1-2}
\endhead\cline{1-2}\multicolumn{3}{r}{\small\textit{continued on next page}}\\\endfoot\cline{1-2}
\endlastfoot\raggedright r\-a\-n\-g\-e\-s\- & Holds the ranges for every variable. Although it is a writable
property, care should be taken in changing parameters before ending the
convergence.&\\
\cline{1-2}
\end{longtable}

    \index{peach \textit{(package)}!peach.pso \textit{(package)}!peach.pso.base \textit{(module)}!peach.pso.base.ParticleSwarmOptimizer \textit{(class)}|)}

%%%%%%%%%%%%%%%%%%%%%%%%%%%%%%%%%%%%%%%%%%%%%%%%%%%%%%%%%%%%%%%%%%%%%%%%%%%
%%                           Class Description                           %%
%%%%%%%%%%%%%%%%%%%%%%%%%%%%%%%%%%%%%%%%%%%%%%%%%%%%%%%%%%%%%%%%%%%%%%%%%%%

    \index{peach \textit{(package)}!peach.pso \textit{(package)}!peach.pso.base \textit{(module)}!peach.pso.base.PSO \textit{(class)}|(}
\subsection{Class PSO}

    \label{peach:pso:base:PSO}
\begin{tabular}{cccccccccc}
% Line for object, linespec=[False, False, False]
\multicolumn{2}{r}{\settowidth{\BCL}{object}\multirow{2}{\BCL}{object}}
&&
&&
&&
  \\\cline{3-3}
  &&\multicolumn{1}{c|}{}
&&
&&
&&
  \\
% Line for list, linespec=[False, False]
\multicolumn{4}{r}{\settowidth{\BCL}{list}\multirow{2}{\BCL}{list}}
&&
&&
  \\\cline{5-5}
  &&&&\multicolumn{1}{c|}{}
&&
&&
  \\
% Line for peach.pso.base.ParticleSwarmOptimizer, linespec=[False]
\multicolumn{6}{r}{\settowidth{\BCL}{peach.pso.base.ParticleSwarmOptimizer}\multirow{2}{\BCL}{peach.pso.base.ParticleSwarmOptimizer}}
&&
  \\\cline{7-7}
  &&&&&&\multicolumn{1}{c|}{}
&&
  \\
&&&&&&\multicolumn{2}{l}{\textbf{peach.pso.base.PSO}}
\end{tabular}


PSO is an alias to \texttt{ParticleSwarmOptimizer}

%%%%%%%%%%%%%%%%%%%%%%%%%%%%%%%%%%%%%%%%%%%%%%%%%%%%%%%%%%%%%%%%%%%%%%%%%%%
%%                                Methods                                %%
%%%%%%%%%%%%%%%%%%%%%%%%%%%%%%%%%%%%%%%%%%%%%%%%%%%%%%%%%%%%%%%%%%%%%%%%%%%

  \subsubsection{Methods}


\large{\textbf{\textit{Inherited from peach.pso.base.ParticleSwarmOptimizer\textit{(Section \ref{peach:pso:base:ParticleSwarmOptimizer})}}}}

\begin{quote}
\_\_call\_\_(), \_\_init\_\_(), restart(), step()
\end{quote}

\large{\textbf{\textit{Inherited from list}}}

\begin{quote}
\_\_add\_\_(), \_\_contains\_\_(), \_\_delitem\_\_(), \_\_delslice\_\_(), \_\_eq\_\_(), \_\_ge\_\_(), \_\_getattribute\_\_(), \_\_getitem\_\_(), \_\_getslice\_\_(), \_\_gt\_\_(), \_\_iadd\_\_(), \_\_imul\_\_(), \_\_iter\_\_(), \_\_le\_\_(), \_\_len\_\_(), \_\_lt\_\_(), \_\_mul\_\_(), \_\_ne\_\_(), \_\_new\_\_(), \_\_repr\_\_(), \_\_reversed\_\_(), \_\_rmul\_\_(), \_\_setitem\_\_(), \_\_setslice\_\_(), \_\_sizeof\_\_(), append(), count(), extend(), index(), insert(), pop(), remove(), reverse(), sort()
\end{quote}

\large{\textbf{\textit{Inherited from object}}}

\begin{quote}
\_\_delattr\_\_(), \_\_format\_\_(), \_\_reduce\_\_(), \_\_reduce\_ex\_\_(), \_\_setattr\_\_(), \_\_str\_\_(), \_\_subclasshook\_\_()
\end{quote}

%%%%%%%%%%%%%%%%%%%%%%%%%%%%%%%%%%%%%%%%%%%%%%%%%%%%%%%%%%%%%%%%%%%%%%%%%%%
%%                              Properties                               %%
%%%%%%%%%%%%%%%%%%%%%%%%%%%%%%%%%%%%%%%%%%%%%%%%%%%%%%%%%%%%%%%%%%%%%%%%%%%

  \subsubsection{Properties}

    \vspace{-1cm}
\hspace{\varindent}\begin{longtable}{|p{\varnamewidth}|p{\vardescrwidth}|l}
\cline{1-2}
\cline{1-2} \centering \textbf{Name} & \centering \textbf{Description}& \\
\cline{1-2}
\endhead\cline{1-2}\multicolumn{3}{r}{\small\textit{continued on next page}}\\\endfoot\cline{1-2}
\endlastfoot\multicolumn{2}{|l|}{\textit{Inherited from peach.pso.base.ParticleSwarmOptimizer \textit{(Section \ref{peach:pso:base:ParticleSwarmOptimizer})}}}\\
\multicolumn{2}{|p{\varwidth}|}{\raggedright best, fbest, fx}\\
\cline{1-2}
\multicolumn{2}{|l|}{\textit{Inherited from object}}\\
\multicolumn{2}{|p{\varwidth}|}{\raggedright \_\_class\_\_}\\
\cline{1-2}
\end{longtable}


%%%%%%%%%%%%%%%%%%%%%%%%%%%%%%%%%%%%%%%%%%%%%%%%%%%%%%%%%%%%%%%%%%%%%%%%%%%
%%                            Class Variables                            %%
%%%%%%%%%%%%%%%%%%%%%%%%%%%%%%%%%%%%%%%%%%%%%%%%%%%%%%%%%%%%%%%%%%%%%%%%%%%

  \subsubsection{Class Variables}

    \vspace{-1cm}
\hspace{\varindent}\begin{longtable}{|p{\varnamewidth}|p{\vardescrwidth}|l}
\cline{1-2}
\cline{1-2} \centering \textbf{Name} & \centering \textbf{Description}& \\
\cline{1-2}
\endhead\cline{1-2}\multicolumn{3}{r}{\small\textit{continued on next page}}\\\endfoot\cline{1-2}
\endlastfoot\multicolumn{2}{|l|}{\textit{Inherited from list}}\\
\multicolumn{2}{|p{\varwidth}|}{\raggedright \_\_hash\_\_}\\
\cline{1-2}
\end{longtable}


%%%%%%%%%%%%%%%%%%%%%%%%%%%%%%%%%%%%%%%%%%%%%%%%%%%%%%%%%%%%%%%%%%%%%%%%%%%
%%                          Instance Variables                           %%
%%%%%%%%%%%%%%%%%%%%%%%%%%%%%%%%%%%%%%%%%%%%%%%%%%%%%%%%%%%%%%%%%%%%%%%%%%%

  \subsubsection{Instance Variables}

    \vspace{-1cm}
\hspace{\varindent}\begin{longtable}{|p{\varnamewidth}|p{\vardescrwidth}|l}
\cline{1-2}
\cline{1-2} \centering \textbf{Name} & \centering \textbf{Description}& \\
\cline{1-2}
\endhead\cline{1-2}\multicolumn{3}{r}{\small\textit{continued on next page}}\\\endfoot\cline{1-2}
\endlastfoot\multicolumn{2}{|l|}{\textit{Inherited from peach.pso.base.ParticleSwarmOptimizer \textit{(Section \ref{peach:pso:base:ParticleSwarmOptimizer})}}}\\
\multicolumn{2}{|p{\varwidth}|}{\raggedright ranges}\\
\cline{1-2}
\end{longtable}

    \index{peach \textit{(package)}!peach.pso \textit{(package)}!peach.pso.base \textit{(module)}!peach.pso.base.PSO \textit{(class)}|)}
    \index{peach \textit{(package)}!peach.pso \textit{(package)}!peach.pso.base \textit{(module)}|)}
